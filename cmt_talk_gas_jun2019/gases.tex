% *{{ Actual preamble
% *{{ preamble
\documentclass[10pt, pdf, hyperref={draft}, usenames, dvipsnames]{beamer}
\usepackage{libertine,textcomp,amssymb,wasysym,color,ulem,verbatim,calc,tikz}
\usepackage{floatrow,subcaption,appendixnumberbeamer,dcolumn,booktabs,multirow}
\usenavigationsymbolstemplate{}
\usepackage[T1]{fontenc}
\setbeamertemplate{caption}[numbered]

\newcolumntype{d}[1]{D{.}{.}{#1}}
\newcommand\mc[1]{\multicolumn{1}{c}{#1}}

% }}*

% *{{ item colouring and themes
\usecolortheme{beaver}
\setbeamercolor{item}{fg=LancsRed}
\setbeamercolor{structure}{fg=black}
\definecolor{LancsRed}{HTML}{B5121B}
\setbeamercolor{subsection in toc}{fg=gray}
\setbeamercolor{bibliography entry note}{fg=black}
\setbeamercolor{bibliography entry title}{fg=black}
\setbeamercolor{bibliography entry author}{fg=LancsRed}



\setbeamertemplate{frametitle}{\nointerlineskip
    \begin{beamercolorbox}[wd=\paperwidth,ht=2.75ex,dp=1.375ex]{frametitle}
        \hspace*{2ex}\insertframetitle \hfill
        {\raisebox{1mm}{\textcolor{gray}{\small$\langle$\ \insertframenumber\
        of
        \inserttotalframenumber\ $\rangle$}}} \hspace*{1ex}%
    \end{beamercolorbox}}

\newcommand{\gitem}[1]{\setbeamercolor{item}{fg=ForestGreen}\item[$\checkmark$] #1}
\newcommand{\bitem}[1]{\setbeamercolor{item}{fg=LancsRed}\item[$\times$] #1}
\newcommand{\nitem}[1]{\setbeamercolor{item}{fg=NavyBlue}\item[$-$] #1}
% }}*

% *{{ margins
\setbeamersize{text margin left=5mm, text margin right=5mm}
% }}*

% *{{ niceties
\newcommand{\dd}{\mathrm{d}}
\newcommand{\e}{\mathrm{e}}
\newcommand{\ket}[1]{\lvert{#1}\rangle}
\newcommand{\bra}[1]{\langle{#1}\rvert}
\newcommand{\expt}[1]{\langle{#1}\rangle}
\newcommand{\red}[1]{{\bf\color{LancsRed}{#1}}}
\newcommand{\blue}[1]{{\bf\color{NavyBlue}{#1}}}
\newcommand{\green}[1]{{\bf\color{ForestGreen}{#1}}}
\newcommand{\orange}[1]{{\bf\color{BurntOrange}{#1}}}
\newcommand{\mred}[1]{{\color{LancsRed}{#1}}}
\newcommand{\mblue}[1]{{\color{NavyBlue}{#1}}}
\newcommand{\mgreen}[1]{{\color{ForestGreen}{#1}}}
\newcommand{\morange}[1]{{\color{BurntOrange}{#1}}}
\newcommand{\kfr}{{\bf k}_{\rm f}}
\newcommand{\kto}{{\bf k}_{\rm t}}
\let\OLDitemize\itemize
\renewcommand\itemize{\OLDitemize\addtolength{\itemsep}{5pt}}
% }}*

% *{{ title, subtitle, inst., date
\title{{\LARGE The electron gas in doped 2D semiconductors}}

\institute[]{\normalsize
  Ryan Hunt \\
  CMT Seminar \\
  20$^{\text{th}}$ June
}

\date{}

% }}*

% *{{ bibliography setup
\usepackage[backend=biber, style=phys]{biblatex}
\bibliography{refs}
\renewcommand{\footnotesize}{\scriptsize}
\AtEveryCitekey{\clearfield{title}
                \clearfield{pagetotal}
                \clearfield{pages}
                \clearfield{doi}}
\renewcommand*{\bibfont}{\footnotesize}

% }}*

% *{{ graphics on front page
\titlegraphic{\includegraphics[width=4cm]{figs/lan_logo.png}
\hfill\includegraphics[width=4cm]{figs/nownano_logo.png}}

% }}*

% *{{ titlepage + toc

% Section and subsections will appear in the presentation overview
% and table of contents.

\begin{document}

\begin{frame}[plain]
  \titlepage
\end{frame}
% }}*

% }}* End Preamble

% *{{ Problem

\begin{frame}{Problem}
\begin{itemize}
  \item Consider the conventional 2DEG. The electrostatic field at the
  interface between two similar materials w/ different single-particle
  energetics binds electrons.
\end{itemize}

\begin{figure}[H]
  \floatbox[{\capbeside\thisfloatsetup{capbesideposition={left, center},
  capbesidewidth=0.4\textwidth}}]{figure}[\FBwidth]
  {\caption{An AlGaAs/GaAS MODFET. Blue and purple are supposed to represent
  material \textit{similarities}.}\label{fig:modfet}}
  {\includegraphics[width=0.4\textwidth]{figs/modfet.pdf}}
\end{figure}

\begin{itemize}
  \item Point is that the ``layer'' and the ``bulk'' are similar.
  \item e$^{-}$ move in 2D, interact as 3D
  ($V(r)\sim1/r$).\footfullcite{Ando1982}$^{,}$\footnote{~~The blue and purple
  bits are similar for a wide range of aluminium fractions - in $a$, and in
  $\epsilon_r$.}
\end{itemize}

\end{frame}

\begin{frame}{Problem (cont.)}

\begin{itemize}

  \item What if the e$^{-}$ behaved a little bit more as if they \textit{were
  really} in 2D\ldots?
  \item Solve $\nabla^{2}\phi = 4\pi\rho $ in 2D, and find
  $V(r)\sim-\text{ln}{(r)}$.\footnote{~~Hartree atomic units, throughout.} But
  this is unphysical.
  \item What IS physical is a finitely-thick (ultimately, atomically thin)
  slab.

  \begin{figure}[H]
    \centering
    \includegraphics[width=0.6\linewidth]{figs/kel.png}
  \end{figure}

\end{itemize}
\end{frame}

% }}*

% *{{ Keldysh interaction
\begin{frame}{Keldysh Interaction}
  \begin{itemize}
    \item In the atomically flat limit, one can show that
  \end{itemize}
\begin{equation}
  V({\boldsymbol\rho}) = V_{\text{K}}({\boldsymbol\rho}) =
  \dfrac{\pi}{2r_{\star}} \left[ H_{0}\left(\frac{r}{r_{\star}}\right) -
  Y_{0}\left(\frac{r}{r_{\star}}\right) \right].
\end{equation}
\begin{itemize}
  \item The parameter $r_{\star} = \kappa/(2\epsilon_{r})$ is a kind of
  screening length.\footnote{~~$\kappa=d(\epsilon+1)$ is the in-plane
  susceptibility.}
\end{itemize}

\begin{figure}[H]
  \floatbox[{\capbeside\thisfloatsetup{capbesideposition={left, center},
  capbesidewidth=0.3\textwidth}}]{figure}[\FBwidth]
  {\caption{The Keldysh and Coulomb interactions compared.}\label{fig:kelint}}
  {\includegraphics[width=0.4\textwidth]{figs/kelint.pdf}}
\end{figure}
\end{frame}
% }}*

% *{{ Physics and experiments / studies
\begin{frame}{Physical Picture}
Consider a 2D TMDC;
\begin{figure}[H]
  \centering
  \includegraphics[width=0.8\linewidth]{figs/tmd.pdf}
\end{figure}
\begin{center}
  (+ the remaining valence!)
\end{center}
\end{frame}
% }}*

% *{{ Theory

\begin{frame}{Our Model - Keldysh Jellium - ``Kellium''}
  \begin{itemize}
    \item Model the conduction electrons in a doped 2D semiconductor as being
    quasiparticles whose interactions are of Keldysh form,
    \begin{equation}
      \mathcal{\hat H} = -\dfrac{1}{2m} \sum_{i} \nabla_{i}^{2} + \sum_{\langle
      i,j \rangle} V_{\text{K}}(\lvert {\boldsymbol\rho}_{i}-{\boldsymbol\rho}_{j} \rvert).
    \end{equation}
    \item The effect of the core electrons, and the nuclei, are then tied up in
    $r_{\star}$ and the effective mass $m$ (parameters!).
    \item Could do this \textit{ab initio} - but at huge cost.\footnote{~~Not
    just huge because QMC - but rather electron density fixing means big
    supercells.}
  \end{itemize}
\end{frame}

% }}*

% *{{ Solution
\begin{frame}{Solving our model}

\begin{itemize}
  \item As written, $\mathcal{\hat H}$ defines an infinite system of electrons.
  \item We're going to have to put finitely many of them in a finite box,
  instead, and invoke periodic boundary conditions.
  \begin{equation}
    \sum_{\langle i,j \rangle} \rightarrow \sum_{\bf R}\sum_{\langle i,j \rangle}
  \end{equation}
  \item Incur a finite-size effect, but one which is controllable. More
  worryingly, lattice sums over long-ranged sums are \textbf{conditionally}
  convergent.
  \item The Ewald method cures this problem.
\end{itemize}

\end{frame}

\begin{frame}{Ewaldised Keldysh interaction}
  \begin{itemize}
    \item The Keldysh interaction is not the same as the Coulomb interaction.
    The Ewald method isn't directly applicable. Time to add zero:
    \begin{equation}
      V_{\text{K}}({\boldsymbol\rho})=V_{\text{C}}({\boldsymbol\rho}) + \left[
      V_{\text{K}}({\boldsymbol\rho})-V_{\text{C}}({\boldsymbol\rho}) \right].
    \end{equation}
    \item First term summed by Ewald method. Second term is
    $\mathcal{O}(r^{-3})$ at long range. Has an \textbf{absolutely} convergent
    lattice sum. Also need to \orange{maintain $\langle V \rangle_{A}$}, so finally
    \begin{align}
      \sum_{\bf R}
      V_{\text{K}}(|{\boldsymbol\rho}_i -{\boldsymbol\rho}_j-{\bf R} |) =
      \orange{\dfrac{2\pi r_{\star}}{A}} +
      V^{\text{Ewald}}_{\text{C}}({\boldsymbol\rho}_i,{\boldsymbol\rho}_j)
      +\nonumber \\ \sum_{\bf R}\left[ V_{\text{K}}(
      |{\boldsymbol\rho}_i -{\boldsymbol\rho}_j-{\bf R} |)-
      V_{\text{C}}(|{\boldsymbol\rho}_i-{\boldsymbol\rho}_j-{\bf R} |)\right].
    \end{align}
    \item Can just pre-compute the big sum on a B-spline
    grid,\footnote{~~Spanning possible range of ${\boldsymbol\rho}_{i,j}$.} and
    interpolate at calculation runtime.\footnote{~~Almost, we need to think
    about short-range.}
  \end{itemize}
\end{frame}

\begin{frame}{What will we study in particular?}
\begin{block}{For now: Wigner Crystallization}
\begin{itemize}
  \item In a 2D semiconductor, what is the expected crystallization density?
  \begin{itemize}
    \item Hamiltonian qualitatively same as 2D HEG. Same argument of Wigner
    applies, ${\hat T}$, ${\hat V}$ compete as fn.~of $r_s$.
  \end{itemize}
\end{itemize}
\end{block}

\begin{figure}[H]
  \floatbox[{\capbeside\thisfloatsetup{capbesideposition={left, center},
  capbesidewidth=0.3\textwidth}}]{figure}[\FBwidth]
  {\caption{Spin-valley locked spin density in a striped antiferromagnetic
  WC. (left) up-spin, K valley quasielectrons, (right) down-spin, K' valley
  quasielectrons.}\label{fig:test}}
  {\includegraphics[width=0.6\textwidth]{figs/test.png}}
\end{figure}

\end{frame}

\begin{frame}{Quantum Monte Carlo calculations}


We use Variational and Diffusion (VMC/DMC) Quantum Monte Carlo methods, with
trial wave functions of form:

\vfill

\begin{itemize}
  \item Wigner Crystal Phases: \green{determinants} of a series of
  site-centered Gaussian orbitals.

  \item Fluid Phases: \green{determinants} of plane waves, with wave vectors defined by
  system size (quantization conditions).
\end{itemize}

\vfill

These are multiplied by \blue{Jastrow} correlation factors, and we also make use of
\orange{backflow} transformations of coordinates,

\begin{equation}
  \Psi({\bf R}) = \blue{\exp{\left[\mathcal{J}({\bf R})\right]}}\cdot
  \green{\mathcal{D}\left[\orange{{\boldsymbol\xi({\bf R})}}\right]}.
\end{equation}

\textbf{Aside}: $\exp{\left[\mathcal{J}({\bf R})\right]}$ allows us to satisfy the cusp
conditions on particle pairs (when ${\bf r}_{ij} \rightarrow 0$, wfn.~must go
to 0 in a particular way). These conditions are \textit{different} under the
Keldysh interaction and require a \textit{bespoke} periodic Jastrow term.

\end{frame}

% }}*

% *{{ To-do list, wish list

\begin{frame}{To Do}

\begin{block}{Lots\ldots}
 \begin{figure}[H]
   \centering
   \includegraphics[width=0.5\linewidth]{figs/wip.jpg}
 \end{figure}
\end{block}

\end{frame}

\begin{frame}{To Do (contd.)}

  \begin{block}{DMC calculations: latest tech}
    \begin{itemize}
      \item Ruggeri \textit{et al.} tricks? (Recent seminar by PLR on electron gas to
      meV accuracy\ldots)
      \item 2DEG-like Fermi surface necessitates twist-averaging.
      \item Can we physically motivate new FS extrapolation formulae?
      \begin{itemize}
        \item E.g.~does Kellium have a Gell-Mann--Brueckner formula? (does it need to?)
      \end{itemize}
    \end{itemize}
  \end{block}

  \begin{block}{Holes at finite density (probably someone elses job)}
    \begin{itemize}
      \item If had $\epsilon(r_s|r_{\star})$, could evaluate e-h correlation
      energy at finite e-density (finite-density excitonic effects!).
    \end{itemize}
  \end{block}
\end{frame}

% }}*

% *{{ The breakdown of our model
\begin{frame}{Failure of the Model}
Seeing as we're all friends, let's talk about where this model will
fail.
\begin{minipage}[t]{0.60\textwidth}

\begin{itemize}
  \item $\exists$ a critical density (at $T=0$K) where the spin-split band
  will fill. This is the onset of true 2 valley, 2 spin dynamics, on two energy
  scales (separated by $\Delta_{\sigma C}$).\footnotemark
  \item $\exists$ a density beyond this where ultra-short-range (inter-valley)
  effects are important, and \ldots
  \item The effective mass approximation itself breaks down when we talk about
  inter-particle distances of order a lattice constant.
\end{itemize}

\end{minipage}%
\hfill
\begin{minipage}[t]{0.3\textwidth}

\begin{figure}[H]
  \centering
  \includegraphics[width=\linewidth]{figs/laundry.png}
\end{figure}

\end{minipage}%
\footnotetext{~~Or, a given $T$, for some density, where thermal fluctuations
might populate these levels significantly anyway\ldots $\sim
4\times10^{12}\text{cm}^{-2}$ in \textbf{encapsulated} MoS$_2$, Pisoni \textit{et al.}}
\vfill
$\implies$ model reasonable at intermediate/low density.
\end{frame}

\begin{frame}{Failure of the Model (cont.)}
\begin{block}{Questions}
\begin{itemize}
  \item Will there ever be a clean enough sample to observe a crystallisation
  transition in a 2D SC? \textcolor{gray}{Pessimistic -
  $n\sim10^{11}\text{cm}^{-2}$.}
  \item At what density do we have a crossover from isolated trion behaviour to
  a collective state in a 2D SC? \textcolor{gray}{Dunno. Efimkin/MacDonald
  suggest almost immediately.\footnote{~~But study huge densities\ldots} Spink \textit{et al.} HEG calculations disagree.}
  \item Can the Ewald-Keldysh interaction apply in other scenarios? Notably,
  may it realistically describe superfluid phases in gated bilayer VdW
  systems? \textcolor{gray}{Probably - if I had more time I'd tinker with this.}
\end{itemize}
\end{block}
\end{frame}

% }}*

% *{{ End

\begin{frame}
Acknowledgements:
\begin{itemize}
  \item Neil
\end{itemize}
\vfill
\begin{center}
  {\Large Thanks!}
\end{center}
\vfill
\end{frame}

\end{document}
% }}*
