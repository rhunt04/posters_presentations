\documentclass[hyperref={draft}, table, xcdraw]{beamer}
\usepackage[size = a0, orientation = portrait, scale = 1.8]{beamerposter}
\usepackage{tikz}
\usepackage{floatrow}
\usepackage{caption}

% Theme
\usetheme{Poster}
\usecolortheme{Ryan_blue}

% Number figures as per usual
\setbeamertemplate{caption}[numbered]

% Colours
\definecolor{EmphCol}{HTML}{0F51E3}
\setbeamercolor{caption name}{fg=black!70}

% Fonts
\usepackage[T1]{fontenc}
\usepackage[utf8]{inputenc}
\usepackage[libertine]{newtxmath}
\usepackage[scaled = 0.9]{libertine}
\usepackage[scaled = 0.75]{inconsolata}

% Commands
\newcommand{\red}[1]{{\bf\color{EmphCol}{#1}}}
\newcommand{\ket}[1]{\lvert {#1} \rangle}
\newcommand{\bra}[1]{\langle {#1} \rvert}
\newcommand{\expt}[1]{\langle {#1} \rangle}

\title{Quantum Monte Carlo Studies of Electronic Bandgaps \\
 \large{\underline{\smash{Ryan Hunt}}$^1$,
 Neil Drummond$^1$ \& Vladimir Fal'ko$^2$}
  \\ \vspace{0.3cm}
  \normalsize{$^1$Department of Physics, Lancaster University \\
              $^2$National Graphene Institute, University of Manchester}}

\author[]{\vspace{-0.3cm}\hspace{6.5cm}
\large{Presenter e-mail : \framebox{\strut\ \texttt{r.hunt4@lancaster.ac.uk}\ \ $\lvert$
                                          \ \texttt{ryan.hunt@postgrad.manchester.ac.uk}\ }}}

\begin{document}

% Images right and left of title - Logos
\addtobeamertemplate{headline}{}
{
  \begin{tikzpicture}[remember picture,overlay]
  \node [shift={(-9.5 cm, -7.2 cm)}] at (current page.north east) {\includegraphics[width=14cm]{figs/nownano-logo.png}};
  \node [shift={( 9.5 cm, -7.2 cm)}] at (current page.north west) {\includegraphics[width=14cm]{figs/lan_logo.png}};
  \end{tikzpicture}
}

\vfill

\begin{frame}[fragile]
\centering
\vspace{-1.0cm}
\begin{columns}[T]

% Main Left Column
\begin{column}{.001\textwidth}
\end{column}
\begin{column}{.47\textwidth}

\begin{block}{{\bf\large Introduction}}
\begin{itemize}
  \item Quantum Monte Carlo (QMC) methods offer an accurate means of probing
  both the \red{ground} and \red{excited state} properties of semiconductors
  from first-principles.
  \item In this series of projects, we have benchmarked the predictive power of
  QMC - and proven that the method can be successfully applied in
  \red{numerous} cases of interest.

\end{itemize}
\end{block}

\begin{block}{{\bf\large QMC Methods}}
\begin{itemize}
  \item In Variational MC, trial many-e$^-$ wavefunctions $\Psi_T({\bf R})$ are
  formed from \red{Slater} determinants $\mathcal{D}({\bf R})$ and
  \red{Jastrow} factors $e\ ^{\text{J}({\bf R})}$
  \begin{equation}
  \Psi_T({\bf R}) = e\ ^{\text{J}({\bf R})}\mathcal{D}({\bf R}).
  \end{equation}
  The Jastrow exponent J depends on parameters, which are \red{optimised} by
  stochastic techniques.

  \item In Diffusion MC, $\ket{\Psi_T}$ is evolved in imaginary time - with
  excited states dying away exponentially
  \begin{equation}
    \ket{\Psi^{\text{DMC}}(\tau)} =
    \sum^{\infty}_{i=0}c_i
    \ \mathrm{e}^{-(\epsilon_i - E_T)\tau}\ket{\Phi_i(0)}.
  \end{equation}

  \item Excited state DMC is non-trivial, and energies are not always
  variational [1]. In our work, the \red{nodes} of $\Psi_T({\bf R})$ are fixed
  and the \red{topology} of the (fixed) nodal surface determines energies.
\end{itemize}
\end{block}

\begin{block}{{\bf\large Which Gap...?}}
\begin{itemize}
  \item The \red{quasiparticle} gap is defined as
  \begin{equation}
    \Delta_{QP} = E_{N+1} + E_{N-1} - 2 E_N,
  \end{equation}

  \item The \red{excitonic} gap is defined as
  \begin{equation}
    \Delta_{Ex} = E^{+}_{N} - E_N,
  \end{equation}

  \item Their difference is the \red{exciton binding}, $E_X =
  \Delta_{QP} - \Delta_{Ex}$.

\end{itemize}
\end{block}

\begin{block}{{\bf\large Molecular Systems$^{[3]}$}}
  \begin{itemize}
  \item We have calculated the QP gaps of several small molecules, incl.
  \red{Anthracene}, via QMC.

  \begin{figure}[H]
    \floatbox[{\capbeside\thisfloatsetup{capbesideposition={left, center},capbesidewidth=0.5\textwidth}}]{figure}[\FBwidth]
    {\caption{The (DFT-relaxed) structure of Anthracene, a molecule of interest
    in molecular electronics.}\label{fig:anth}}
    {\includegraphics[width=0.4205\textwidth]{figs/anth.png}}
  \end{figure}

  \begin{table}[H]
  \centering
  \begin{tabular}{|c||cccc
  >{\columncolor[HTML]{6BCEF6}}c |}
  \hline
  Method &\ Expt. &\ DFT-LDA &\ $G_0W_0^{\star}$ &\ $GW$ &\ DMC \\ \hline \hline
  $\Delta_{QP}$ / eV&\ 6.9 &\ 2.25 &\ 6.15-6.86 &\ 6.74 &\ 6.89(6) \\ \hline
  \end{tabular}
  \caption{Experimental and theoretical $\Delta_{QP}$ for Anthracene ($GW$ from
  [2]). \\$^{\star}$ These gaps are orbital dependant.}
  \label{anth-qpg}
  \end{table}

  \item \red{Potential caveats:} Jahn-Teller effect, phonon renormalisation -
  each of these could affect our (static-nucleus) results.

  \end{itemize}

\end{block}

\end{column}

\hfill

% Main Right Column
\begin{column}{.47\textwidth}

\begin{block}{{\bf\large Boron Nitride$^{[4]}$}}
\begin{itemize}
\item We have calculated both the \red{excitonic} and \red{quasiparticle} gaps
of bulk and monolayer hBN.


\begin{figure}[H]
  \floatbox[{\capbeside\thisfloatsetup{capbesideposition={left, center},capbesidewidth=0.4\textwidth}}]{figure}[\FBwidth]
  {\caption{An example (288 e$^-$) supercell used in our
  bulk simulations.}\label{fig:bhbn}}
  {\includegraphics[width=0.55\textwidth]{figs/bhbn.png}}
\end{figure}

\item \red{Main Point:} Gaps are \red{strongly} enhanced in the monolayer

\begin{table}[H]
\centering
\vspace{0.2cm}
\caption{A subset of our hBN results. The calculated monolayer exciton binding
is $2.0(3)$ eV, with the bulk value to be determined.}
\label{hbn_table}
\begin{tabular}{|c||cc|}
\hline
System $\rightarrow$ &\ \ \  Bulk \ \ &\ \ \ Monolayer \ \ \\ \hline \hline
\ $\Delta_{Ex}(K_v \rightarrow K_c)$ / eV\ \ \ &\ \ \ $5.8(1)$ & 8.7(3)  \\
\ $\Delta_{Ex}(K_v \rightarrow \Gamma_c)$ / eV\ \ \ &\ \ $5.69(8)$ & 7.5(3)  \\ \hline
\end{tabular}
\end{table}

\item First QMC gap calculation for a \red{layered}, and a \red{2D} material.
\item First analysis of \red{finite size effects} in QMC energy gaps.
\end{itemize}

\end{block}


\begin{block}{{\bf\large Bulk Silicon {\it\large et al.}$^{[3]}$}}
\begin{itemize}
\item Silicon in the diamond structure is perhaps {\it the} most widely
studied semiconductor.
\begin{figure}[H]

  \centering
  \includegraphics[width=0.604\textwidth, angle = 0]{figs/si.png}
  \caption{An example (216 e$^-$) supercell used in our simulations.}
  \label{bhbn_332}

\end{figure}

\item Our ongoing simulations hope to act as thorough benchmarks of the DMC
method as used in gap calculations.

\item We have also opted to study cubic BN and $\alpha$-SiO$_2$. Both have
sizeable gaps, and cubic BN hosts a well-bound exciton.

\end{itemize}
\end{block}


\begin{block}{{\bf\large Conclusions}}
\begin{itemize}
\item QMC methods are robust and accurate, and may be used to calculate energy
gaps for semiconductors of any dimensionality.
\item QMC is also capable of describing systems bonded by a variety of means -
including by van der Waals interactions.
\end{itemize}
\end{block}

\begin{block}{{\bf\large Acknowledgements}}
\begin{itemize}
\item QMC calculations were performed with the \red{\textsc{CASINO}} code, and
DFT trial wavefunctions were obtained with \red{\textsc{CASTEP}}.
\item All calculations were performed on the \red{HEC} facility at Lancaster
and the \red{N8 HPC}.
\end{itemize}
\end{block}
\end{column}
\end{columns}

\begin{center}
\hfill
\begin{minipage}{0.976\textwidth}
\begin{block}{{\bf\large References}}
\begin{columns}[T]


% Left Column
\begin{column}{.48\textwidth}
\begin{enumerate}
\item Foulkes, W. M. C., {\it et al.} PRB {\bf 60} (1999): 4558.
\item Blase, X., {\it et al.} PRB {\bf 83} (2011): 115103.
\end{enumerate}
\end{column}
\hfill
% Right Column
\begin{column}{.48\textwidth}
\begin{enumerate}
\setcounter{enumi}{2}
\item Hunt, R. J., Drummond, N. D. Manuscript in preparation, (2016).
\item Hunt, R. J., {\it et al.} Manuscript in preparation, (2016).
\end{enumerate}
\end{column}
\end{columns}
\end{block}
\end{minipage}
\end{center}

\end{frame}

\end{document}

