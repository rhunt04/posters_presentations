% *{{ preamble
\documentclass[12pt, pdf, hyperref={draft}, usenames, dvipsnames]{beamer}
\usepackage{libertine, textcomp, amssymb, wasysym, color, ulem, verbatim}
\usepackage{floatrow, subcaption}
\usepackage[makeroom]{cancel}
\usepackage{mathtools}
% hide controls
\usenavigationsymbolstemplate{}
\usepackage[T1]{fontenc}
\synctex=1
\setbeamertemplate{caption}[numbered]
% }}*

% *{{ item colouring and themes
\usecolortheme{beaver}
\setbeamertemplate{footline}[frame number]
\definecolor{LancsRed}{HTML}{B5121B}
\setbeamercolor{item}{fg=LancsRed}
\setbeamercolor{title}{fg=LancsRed}
\setbeamercolor{structure}{fg=LancsRed}
\setbeamercolor{frametitle}{fg=LancsRed}
\setbeamercolor{footnote mark}{fg=LancsRed}
\setbeamercolor{footnote}{fg=LancsRed}
\setbeamercolor{bibliography entry author}{fg=LancsRed}
\setbeamercolor{bibliography entry title}{fg=black}
\setbeamercolor{bibliography entry note}{fg=black}
\setbeamercolor{section in toc}{fg=LancsRed}
%\setbeamercolor{subsection in toc}{fg=gray}
\newcommand{\gitem}[1]{\setbeamercolor{item}{fg=ForestGreen}\item[$\checkmark$] #1}
\newcommand{\bitem}[1]{\setbeamercolor{item}{fg=LancsRed}\item[$\times$] #1}
\newcommand{\nitem}[1]{\setbeamercolor{item}{fg=NavyBlue}\item[$-$] #1}
% }}*

% *{{ niceties
\newcommand{\dd}{\mathrm{d}}
\newcommand{\e}{\mathrm{e}}
\newcommand{\ket}[1]{\lvert{#1}\rangle}
\newcommand{\bra}[1]{\langle{#1}\rvert}
\newcommand{\expt}[1]{\langle{#1}\rangle}
\newcommand{\red}[1]{{\bf\color{LancsRed}{#1}}}
\newcommand{\blue}[1]{{\bf\color{NavyBlue}{#1}}}
\newcommand{\green}[1]{{\bf\color{ForestGreen}{#1}}}

% }}*

% *{{ title, subtitle, inst., date
\title{Modelling Charge Complexes in \\
       2D Materials \& Their Heterostructures}
\subtitle{Graphene NOWNANO CDT Monthly Seminar}

% authors and institutions
%\author{\emph{Ryan J. Hunt}\inst{1},
%        Neil D. Drummond\inst{1} \\
%        and Vladimir I. Fal'ko\inst{2}}

\author{Ryan Hunt \\
27$^{\text{th}}$ October, 2017}

%\institute[]{
%
%  \inst{1}
%  Department of Physics,\\
%  Lancaster University
%  \and
%
%  \inst{2}
%  National Graphene Institute,\\
%  University of Manchester}
%
\date{}

% }}*

% *{{ bibliography setup
\usepackage[backend=bibtex]{biblatex}
\bibliography{refs}
\renewcommand{\footnotesize}{\scriptsize}
\AtEveryCitekey{\clearfield{title}
                \clearfield{pagetotal}
                \clearfield{pages}}

% }}*

% *{{ graphics on front page
\titlegraphic{\vspace*{-1.5cm}\includegraphics[width=3cm]{figs/lan_logo.png}
\hfill\includegraphics[width=3cm]{figs/nownano_logo.png}}

% }}*

% *{{ titlepage + toc

% Section and subsections will appear in the presentation overview
% and table of contents.

\begin{document}

\begin{frame}[plain]
  \titlepage\end{frame}

%\begin{frame}{Outline}
%  \tableofcontents
%\end{frame}

% contents at ssubsection starts - comment to get rid
\AtBeginSection[]
{\begin{frame}<beamer>{Outline}
  \tableofcontents[currentsection]
  \end{frame}}

% }}*

% Setup: what's the question? EM in 2D (logarithms!)
% Monolayers: Keldysh, results, big things, word on metallics
% Bilayers (and beyond...?): potentials, approximations, exactitudes,
% experiments, decay rates?

% *{{ What is my question?

% define charge complexes, motivate their study, talk about why 2D materials
% are interesting.

\begin{frame}{What are charge complexes, why study them?}

\begin{itemize}
  \item Charge-carrier complexes are \blue{bound states} of two or more charges in a host
  material.
  \item Common example is an exciton.

  \begin{figure}[H]
    \floatbox[{\capbeside\thisfloatsetup{capbesideposition={left, center},
    capbesidewidth=0.5\textwidth}}]{figure}[\FBwidth]
    {\caption{Schematic view of an exciton - a bound state comprised of one
    electron and one hole.}\label{fig:exc_schematic}}
    {\includegraphics[width=0.5\textwidth]{figs/exciton_setup.pdf}}
  \end{figure}

  \item At low enough temperatures, charge complexes {\bf dominate} the optical
  response of semiconductors.

\end{itemize}

\end{frame}

\begin{frame}{Effective mass approximation}

  \begin{itemize}

    \item In conventional semiconductors,\footnote{Take III-Vs as an explicit
    example.} the binding of excitons may be described in the \blue{effective mass
    approximation}.

    \item The electronic band structures of these materials serve
    to supply effective masses for electrons and holes.

    \item We then solve Schr\"{o}dinger equations analogous to those for the
    hydrogen atom:

    \begin{equation}
      \left[-\sum_i\dfrac{\hbar^2}{2m_i}\nabla^2_i + \sum_{i<j}V({\bf
      r}_{ij})\right]\Psi(\{{\bf r}_i\}) = E\Psi(\{{\bf r}_i\})
    \end{equation}

  \end{itemize}
\end{frame}

% }}* end my question

% *{{ 2D EM, our monolayer model, QMC

\begin{frame}{Electromagnetism in Flatland}
\begin{itemize}
  \item If we lived in flatland (bona fide 2D space), electromagnetism would
  be a \blue{very different beast}.
  \item Poisson's equation would look the same, however, it's (two-dimensional)
  solution would predict a ``logarithmic Coulomb potential'' between pairs of
  point charges.\footnote{\
  Explanation for technical people: look at the inverse Fourier transform of
  the Green's function $G({k})=1/k^2$ in 2D and 3D. By power counting you can
  see that this is the case. EM in 1D is ever stranger.}
\end{itemize}
\begin{equation}
  \nabla^2 \phi({\bf r}) = -\dfrac{\rho({\bf r})}{\epsilon_0}.
\end{equation}
\begin{equation}
  V_q({\bf r}) = -q\ln{\lvert {\bf r} \rvert}.
\end{equation}
\begin{itemize}
  \item But we don't live in flatland, so we \red{don't get this}
  \frownie\
\end{itemize}
\end{frame}

\begin{frame}{Reality: 3D space, and electrostatic screening}
  \ldots or do we? In any realistic model, and in every experiment ever
  done on 2D materials, one other factor is very important: \green{screening}.

\begin{itemize}
  \item Electrostatic screening is the suppression in electric field strength
  due to the presence of mobile charges.
  \item Screening is lessened for 2D materials that are taken from their parent
  compound - interactions between charges are stronger.
\end{itemize}
\end{frame}

\begin{frame}{The Keldysh Interaction}
\begin{itemize}
  \item Keldysh\footfullcite{Keldysh1979} showed that the effective interaction between charges in
  a 2D material has a special form
  \begin{equation}
    V_{\text{K}}({\bf r}) = \dfrac{q_1q_2}{8\epsilon_r \epsilon_0 r_*}
    \left[ H_0\left(\frac{r}{r_*}\right) - Y_0\left(\frac{r}{r_*}\right)
    \right],
  \end{equation}
  with $r_*$ a material parameter (usually \red{inferred} from \textit{ab initio}
  DFT or $GW$ calculations of $\epsilon({\bf q},\omega)$).
  \begin{figure}[H]
    \floatbox[{\capbeside\thisfloatsetup{capbesideposition={right, center},
    capbesidewidth=0.5\textwidth}}]{figure}[\FBwidth]
    {\caption{The Keldysh interaction. Notice we have recovered a
    log-dependence at short-range\ldots}\label{fig:keldysh}}
    {\includegraphics[width=0.5\textwidth]{figs/full_int.eps}}
  \end{figure}

\end{itemize}
\end{frame}

\begin{frame}{Complexes in Monolayers}
\begin{itemize}
  \item It turns out that the Keldysh interaction permits numerous bound states
  to exist, for typical $r_*$ values ($\mathcal{O}$(50\ \AA), vacuum).
\end{itemize}
\begin{figure}[H]
  \centering
  \includegraphics[width=0.8\linewidth]{figs/various_complexes.pdf}
  \caption{A few of the charge carrier complexes which can exist in 2D
  semiconductors.}
\label{fig:various_complexes}
\end{figure}
\end{frame}


\begin{frame}{Our Model}{\& some of its limitations}
\begin{itemize}
  \item Taking monolayers as a goal, for now, we seek to solve
  \begin{equation}
      \left[-\sum_i\dfrac{\hbar^2}{2m_i}\nabla^2_i + \sum_{i<j}V_{\text{K}}({\bf
      r}_{ij})\right]\Psi(\{{\bf r}_i\}) = E\Psi(\{{\bf r}_i\}),
  \end{equation}
  for various complexes. We then compare total energies $E$ in order to
  evaluate binding energies (experimentally relevant).

  \item We ignore exchange effects, but can add
  them back in as a \blue{perturbative correction} (requires pair distribution
  functions, and unknown parameters\ldots).

  \item We are assuming that both the E.M.A. holds, and that the Keldysh
  interaction is a good approximation for real materials.
\end{itemize}
\end{frame}


\begin{frame}{Solving the few-body problem:\\ Quantum Monte Carlo}
\begin{itemize}
  \item We solve our few-body effective mass Schr\"{o}dinger equations by using
  the variational and diffusion quantum Monte Carlo methods (VMC, DMC).
  \item In VMC, estimates of high-dimensional integrals are formed from
  (cleverly weighted) random sampling. \red{The results are as good as the trial
  wave function}.
\end{itemize}
\begin{align}
  E\left[\ket{\Psi}\right] &= \dfrac{\bra{\Psi} \mathcal{\hat H}
  \ket{\Psi}}{\bra{\Psi}\Psi \rangle} = \displaystyle\int \dd{\bf R} \Pi({\bf
  R}) E_{\text{L}}({\bf R}) \nonumber \\
  \Pi({\bf R}) &= \dfrac{\lvert\Psi({\bf R}) \rvert^2}{\int\dd{\bf R}\lvert
  \Psi({\bf R})\rvert^2},\ E_{\text{L}} = \dfrac{\mathcal{\hat H} \Psi({\bf R})}{\Psi({\bf R})}
\end{align}
\end{frame}


\begin{frame}{Quantum Monte Carlo II - VMC}
\begin{itemize}
  \item VMC is usually only ever done as a prelude to DMC\@.
  \item In these models, we start with \blue{educated guesses} at the trial wave
  function which are of \blue{Jastrow} form
  \begin{equation}
    \Psi_{\text{T}}({\bf R}) = \exp{\left[ \mathcal{J}_{\{\alpha\}}({\bf R})\right]}
  \end{equation}
  where $\{\alpha\}$ are a set of optimisable parameters.
  \item In practice, we vary the $\{\alpha\}$ such that some property of the
  wave function is \blue{optimal} (minimise E, variance of E, or MAD of E).
\end{itemize}
\end{frame}

\begin{frame}{Quantum Monte Carlo III - DMC}
\begin{itemize}
  \item In DMC, a trial function is propagated in imaginary
  time,\footnote{Details not important - but we do this by interpreting the
  problem as a \textit{statistical} one.} such that
  any excited state components it may contain are removed. The results are
  \green{independent of the trial function (in our cases), and offer improvement over
  VMC results.}
  \item We exploit (with $t=i\tau$)
  \begin{equation}
    \lim_{\tau \rightarrow \infty} \exp{\left[ -\tau \mathcal{\hat H} \right]}
    \ket{\Psi_{\text{T}}} \sim \ket{\Psi_{\text{GS}}}
  \end{equation}

  \item DMC is \green{exact for nodeless wave functions}.

\end{itemize}
\end{frame}

\begin{frame}{Fairness: How else might one do this?}
Other approaches for solving these kinds of problems exist. I won't
talk about these, but for completeness I will mention them.
\begin{itemize}
  \item $GW$--$BSE$: Solve the Bethe-Salpeter equation for the spectral
  function of a material including two-body effects.
  \item First-principles QMC:\@ Solve the many-electron Schr\"{o}dinger
  equation for the interacting system. Few people have ever actually done this
  (I'm one!).
\end{itemize}
There's a good book covering both of these by Martin, Reining, and
Ceperley.\footfullcite{Martin2016}
\end{frame}


\begin{frame}{Results: Monolayers}

\begin{table}[H]
  \centering
\begin{tabular}{lcccccc}
\hline \hline
& \multicolumn{2}{c}{$E_{\rm X^-}^{\rm b}$
  (meV)} & \multicolumn{2}{c}{$E_{\rm
    X^+}^{\rm b}$ (meV)} \\
\raisebox{1.5ex}[0pt]{TMDC} & DMC&
Exp. &
DMC & Exp. \\ \hline

MoS$_2$ & $35.0$ &
$40$, $18.0(15)$, $43$
& $34.9$ & \\

MoSe$_2$ & $34.5$ &
$30$ & $34.4$ &
$30$ \\

WS$_2$ & $33.5$ &
$34$,$36$,
$10$--$15$, $30$,
$45$ & $33.6$ & & \\

WSe$_2$ & $29.6$ &
$30$ & $29.6$ &
$30$, $24$ \\

\hline \hline
\end{tabular}
\caption{Some trion binding energy results from our paper. Refs.\ and comparative results may be found
therein.\footfullcite{biexpaper}}
\label{tab:trion_results}
\end{table}

\begin{itemize}
  \item We also enjoy good agreement with other theoretical studies, based on
  DMC, PIMC, SVM, heavy-hole approx., and variational calculations.
\end{itemize}

\end{frame}

\begin{frame}{Results: Monolayers II}

\begin{itemize}
  \item Another aspect of this paper was the study of higher
  complexes. Surprisingly, we find that such systems are bound\ldots

\begin{table}[H] \caption{Binding energies of BIG complexes in different
TMDCs. Note: these aren't necessarily PL peak positions.
\label{table:big_complexes}} \begin{center} \begin{tabular}{lccc} \hline
\hline
% peak position == de-excitonisation energies.

 & \multicolumn{3}{c}{Binding energy (meV)} \\

\raisebox{1.5ex}[0pt]{TMDC} & XX$^-\ (\rm eeehh)$ & D$^-$X ($\rm Deeeh$) & D$^0$XX
($\rm Deeehh$) \\

\hline

MoS$_2$  & $58.6(6)$ & $84.4(4)$ & $61.6(6)$ \\

MoSe$_2$ & $57.0(4)$ & $57.9(2)$ & $56.9(9)$ \\

WS$_2$   & $57.4(3)$ & $59.2(4)$ & $58.2(6)$ \\

WSe$_2$  & $52.5(7)$ & $51.3(4)$ & $51(1)$ \\

\hline \hline
\end{tabular}
\end{center}
\end{table}

\item Biex.\ problem: Big things end up being called ``biexcitons''?

\end{itemize}

\end{frame}

% }}*

% *{{ Modelling heterostructures: multilayer interactions
\begin{frame}{Heterostructures: multilayer interactions}
\begin{itemize}
  \begin{figure}[H]
    \floatbox[{\capbeside\thisfloatsetup{capbesideposition={left, center},
    capbesidewidth=0.5\textwidth}}]{figure}[\FBwidth]
    {\caption{A tri-layer heterostructure of 2D semiconductors. There are $N$
    layer potentials, and $N(N-1)/2$ inter-layer potentials.}\label{fig:multilayers}}
    {\includegraphics[width=0.5\textwidth]{figs/multilayers.pdf}}
  \end{figure}

  \item Fourier components can be
  determined,\footnote{Analytically, upto $N=4$. Possibly higher $N$ in
  special cases, or, as always, with symmetry.} but
  real-space potentials obtained numerically (Hankel transform).
\end{itemize}
\end{frame}

\begin{frame}{Bilayers of 2D semiconductors}
How do charges interact in bilayers? More special functions?
\begin{figure}[H]
  \floatbox[{\capbeside\thisfloatsetup{capbesideposition={left, center},
  capbesidewidth=0.5\textwidth}}]{figure}[\FBwidth]
  {\caption{Exact inter-layer and intra-layer interaction potentials versus
  distance. Parameters taken from an experimentally relevant MoSe$_2$/WSe$_2$
  geometry.}\label{fig:interactions}}
  {\includegraphics[width=0.45\textwidth]{figs/both_wse2mose2.eps}}
\end{figure}

\begin{itemize}
  \item The screened interactions start to get a little cumbersome here:
\end{itemize}
\begin{equation}
  v_{\text{intra}}({\bf q}) =
  \frac{(1+r_{*j}q)\exp(qD)
  -r_{*j}q\exp(-qD)}{2\bar{\epsilon}
  q\left[(1+r_{*j}q)(1+r_{*i}q)\exp(qD)-
  r_{*i}r_{*j}q^2\exp(-qD)\right]} \label{eq:intralayer_pot}
\end{equation}
\end{frame}

\begin{frame}{A Quick Tangent}

\begin{itemize}
  \item The interactions are qualitatively similar to the (displaced) Coulomb
  interactions. We expect similar physics to that in CQWs of III-Vs.


\begin{figure}[H]
  \floatbox[{\capbeside\thisfloatsetup{capbesideposition={left, center},
  capbesidewidth=0.5\textwidth}}]{figure}[\FBwidth]
  {\caption{Schematic of an indirect trion in a coupled quantum well.}\label{fig:trion_cqw}}
  {\includegraphics[width=0.35\textwidth]{figs/trion_cqw.pdf}}
\end{figure}

  \item These were originally studied as the ``ideal testbed'' for finding BECs
  of excitons. We've studied them before.~\footfullcite{trion_paper}

\end{itemize}

\end{frame}

\begin{frame}{Bilayers: cont.}

\begin{itemize}
  \item Our work with bilayers is ongoing, but we find that we are capable of
  explaining at least one set of experimental results. [In fairness, many
  relevant experimental results now exist].
  \item We currently use an approximation to the bilayer interactions - but now
  have access to the full solution to Poisson's equation (i.e.\ we can do a
  little better?).
  \item We've also managed to generalise this to the case of $N$-layers. We may
  study higher heterostructures in future.
\end{itemize}
\end{frame}

\begin{frame}{What else can QMC do for us?}

\begin{itemize} \item Could try to solve full many-body Schr\"{o}dinger
equation? Is expensive. Failing that, {\bf is there anything useful QMC can do
beyond calculate binding energies?}

  \item In collaboration with Mark Danovich, David Ruiz-Tijerina, and Volodya
  Fal'ko, we've\footnote{Myself, Neil Drummond and Marcin Szyniszewski.} used
  QMC + pen-and-paper to form estimates of {\it lifetimes\/} in perturbation
  theory.
  \end{itemize}

  \begin{figure}[H]
    \centering
    \includegraphics[width=0.8\linewidth]{figs/lifetime.pdf}
\label{fig:lifetime}
  \end{figure}

\end{frame}

% }}*

\begin{frame}{Future (relevant) Research Avenues}

\begin{itemize}

  \item Excited states of few-body complexes: some of these could
  be bound, and (theoretically, I admit) show up in PL spectra. I'm interested in
  using the Faddeev equations and/or QMC to try to study these
  states.\footfullcite{Faddeev1961}

  \item e-h droplets in TMDCs? There's been recent interest in
  studying ``electron-hole droplets''.\footfullcite{Keldysh1986,Almand2014} These might show up in
  TMDC bilayerss. I'd like to find out.\footnote{This requires ``Ewaldising''
  the Keldysh interaction - i.e.\ making lattice sums of its $1/r$ component
  absolutely convergent. I'm doing this as I write the presentation, ask me how
  its going\ldots}

  \item Possible multi-component electron gases in TMDCs? Requires periodic
  Keldysh interaction.

\end{itemize}
\end{frame}

% *{{ thanks!
\begin{frame}[plain]
\begin{center}
  {\Huge Thank you all for listening!}
\end{center}
\end{frame}
% }}*

\end{document}

