\documentclass[12pt, a4paper]{article}

\usepackage[top=20mm, bottom=20mm, left=20mm, right=20mm, includefoot]{geometry}
\usepackage{amsmath, amssymb, enumitem, color, framed}
\usepackage[obeyspaces,spaces]{url}

\synctex=1

% fonts setup
\usepackage{cmbright}
\usepackage[T1]{fontenc}
\usepackage[mathscr]{euscript}

% general useful objects
%\pagenumbering{gobble}
\newcommand{\ket}[1]{\lvert{#1}\rangle}
\newcommand{\bra}[1]{\langle{#1}\rvert}
\newcommand{\expt}[1]{\langle{#1}\rangle}
\newcommand{\e}{\mathrm{e}}
\newcommand{\dd}{\mathrm{d}}

\title{Excited States by QMC}
\author{TTI 2017 Worksheet \\ Ryan Hunt \\ Lancaster University}

\begin{document}
\maketitle

\section{{The first ionization potential of O$_2$}}
In this exercise, we will calculate the first ionization potential of the
oxygen molecule. The ionization potential is defined as

\begin{equation}
  \text{IP} = E^{0}_{N-1} - E^{0}_{N},
\end{equation}

\noindent where $E^{0}_{N}$ denotes the ground state energy of the $N$-electron
molecule. We will perform this calculation with the ``G2'' geometry, a Gaussian
basis set, and we will not use pseudopotentials. We'll use VMC for speed, but
the extension to DMC is not difficult (if you are interested, time steps of
0.001 and 0.004 a.u.\ are reasonable). In this case, we should be able to
achieve reasonable agreement with experimental results
(\textcolor{red}{why?}).\footnote{See, for example, the data listed at
\url{http://webbook.nist.gov/cgi/cbook.cgi?ID=C7782447&Mask=20}.}

\subsection{Problem Setup}\label{sub:problem_setup}

\begin{enumerate}

  \item Copy the contents of \url{~/CASINO/examples/molecule/G2_SET/o2/} to a
  directory you wish to work from.\footnote{If \texttt{CASINO} is not installed
  in your home directory, find it, and copy the same files.}

  \item Look in the input file. It is relatively simple, and (if everything has
  worked!) specifies a VMC calculation. Something worth noting is that
  \texttt{neu}=9 and \texttt{ned}=7 - in its ground state, O$_2$ hosts a
  spin-triplet configuration ($^{3}\Sigma^{-}_{g}$).

  \item Make two directories: \texttt{gs} and \texttt{rem}, copy all of the
  files into each of them. The \texttt{gs} calculation is now ready, and we
  need to set up the excited state. Enter the \texttt{rem} directory and open
  the input file. We need to remove a single electron from this configuration.

  \item The correct electron to remove (which leads to the lowest IP) is one of
  the spin-up electrons. Edit the file so that \texttt{neu}=8. [EXTRA:\ If you
  feel like it, try to see what happens in the rest of the exercise when
  removing a spin-down electron. You may or may not be able to notice the
  difference, \textcolor{red}{why?}]

\end{enumerate}

\subsection{VMC Calculations}\label{sub:vmc_calculations}

\begin{enumerate}

  \item If everything has gone well, you will now have two VMC calculations
  ready to execute - one to calculate $E^{VMC}_{N}$ and one to calculate
  $E^{VMC}_{N-1}$.

  \item Run each calculation with \texttt{runqmc}. [EXTRA:\ If you want, before
  you do this, you can try to optimise the Jastrow factor (edit
  \texttt{correlation.data} terms, change \texttt{runtype} to \texttt{vmc\_opt}
  in the input file, and so on).]

  \item Is the number of VMC steps in the input file limiting you? Try
  achieving a better accuracy by adjusting the number of VMC steps. If you are
  struggling to think of a good number, think about statistics. The error bar
  should scale like $1/\sqrt{N_{\text{steps}}}$ - you have one result with a
  small number of steps.

\end{enumerate}

\subsection{Checking \& Thinking}\label{sub:checking_&_thinking}

\begin{enumerate}

  \item What is your final result for the IP?\ How does it compare to
  experiment?

  \item What major aspect of this example has been missed by our approach? Why
  might this make comparison to \textit{some} of the experiments difficult?

  \item If an experimentalist friend (they exist) approaches you, and asks if
  you can give them an accurate ionization potential accurately for their
  molecular system, what should you ask them about their experiment before you
  say ``yes'' or ``no''?

\end{enumerate}


\section{$\bf\Gamma \boldsymbol\rightarrow \Gamma$ energy gap in Silicon}

Perhaps a slightly more interesting case for QMC is that of a crystalline
solid. In this example, we'll consider a $2\times 2\times 2$ supercell of
Silicon, and try to calculate the energy gap from $\Gamma \rightarrow \Gamma$.
In particular, we will calculate the \textit{excitonic} energy gap

\begin{equation}
  \Delta_{\text{Exc}} = E^{\Gamma \rightarrow \Gamma}_{N} - E^{0}_{N}
\end{equation}

\noindent
where, again, $E^{0}_{N}$ is the ground state energy of the $N$ electron
system, but this time $E^{\Gamma \rightarrow \Gamma}_{N}$ is the energy of the
system which has had an electron promoted from the valence band at $\Gamma$ to
the conduction band at $\Gamma$.

\subsection{Problem Setup}\label{sub:problem_setup_2}

\begin{enumerate}
  \item Copy the contents of \textit{the silicon folder} to
  a directory you wish to work from.

  \item Look in the input file. Again, a relatively simple case, this
  calculation specifies a VMC calculation. Type \texttt{runqmc} and read the
  out file - try to figure out what is going on.

  \item Make two directories, one for the ground state calculation and one for
  the excited state ($\Gamma \rightarrow \Gamma$ promotion) calculation. Give
  them helpful names!

  \item Copy the \texttt{CASINO} files into the directory you created to hold
  the ground state calculation.

  \item Enter that directory and \texttt{runqmc}. Read the output file, where
  are the electrons? Where are they \textit{not}?

\end{enumerate}

\subsection{VMC calculations}\label{sub:vmc_calculations_2}

\begin{enumerate}
  \item Now you should know that the \texttt{bwfn.data} file contains orbitals
  for empty states. These are useful - we will want to promote electrons into
  these!

  \item Check the ground state calculation output file - we need to know which
  {\bf k}-point is the $\Gamma$ point - and which band number is the valence band
  maximum (\& conduction band minimum). Take note of these.

  \item Read section 7.4.5 of the \texttt{CASINO} manual, at
  \url{~/CASINO/manual/}. This will be important for the next stage!

  \item Copy the \texttt{bwfn.x} files into the directory holding the files for
  the promotion calculation, and move into that directory. We will need to take
  the advice of the manual - and use the MDET block to specify an excitation.

  \item Open the \texttt{correlation.data} file. We need to specify an
  excitation from the VBM at $\Gamma$ to the CBM at $\Gamma$. Try to do this
  \textbf{yourself} - using the band and {\bf k}-point numbers from
  earlier, and the guidance in the manual.\footnote{If you are stuck, ask for
  help!}

  \item Run the promotion calculation, and check the out file. Are the
  occupancies what you expect? They \textit{might not be}. \textcolor{red}{Why?}

  \item Try deleting the \texttt{bwfn.data.bin} file you copied in earlier.
  Re-run the calculation. Check the occupancies now. What has changed, and why?

  \item[{\bf FYI:}] \texttt{CASINO} won't always be like this, we want to make
  it easier to do these calculations. However, this is the easiest way to get
  an understanding of what happens (a) in the code and (b) to the wave
  function upon the creation of an electronic excitation.
\end{enumerate}

\subsection{Checking \& Thinking}\label{sub:checking_&_thinking_2}

\begin{enumerate}
  \item What is your final result for the $\Gamma \rightarrow \Gamma$ excitonic
  gap in silicon? How does this compare to literature values?

  \item Can you increase the number of VMC steps, to lower your VMC error bar?
  If so, try obtaining a more accurate VMC result.

  \item In the O$_2$ case, we missed something that could have been important -
  vibrational effects. We've missed those here, but we also have another
  \textbf{major} source of error. What is it? How could we try to address it?

  \item If you were to do the same calculation for another solid, or at
  another point in {\bf k}-space, what questions would you have to ask yourself
  before you start your project? [Hints: think about {\bf k}-point grids, the
  number of electrons in the system, how much computer time you expect to be
  able to sacrifice\ldots]
\end{enumerate}



\end{document}
