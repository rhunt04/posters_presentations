% *{{ preamble
\documentclass[12pt, pdf, hyperref={draft}, usenames, dvipsnames]{beamer}
\usepackage{libertine, textcomp, amssymb, wasysym, color, ulem, verbatim}
\usepackage{floatrow, subcaption}
% hide controls
\usenavigationsymbolstemplate{}
\usepackage[T1]{fontenc}
\synctex=1
\setbeamertemplate{caption}[numbered]
% }}*

% *{{ item colouring and themes
\usecolortheme{beaver}
\setbeamertemplate{footline}[frame number]
\definecolor{LancsRed}{HTML}{B5121B}
\setbeamercolor{item}{fg=LancsRed}
\setbeamercolor{title}{fg=LancsRed}
\setbeamercolor{structure}{fg=LancsRed}
\setbeamercolor{frametitle}{fg=LancsRed}
\setbeamercolor{footnote mark}{fg=LancsRed}
\setbeamercolor{footnote}{fg=LancsRed}
\setbeamercolor{bibliography entry author}{fg=LancsRed}
\setbeamercolor{bibliography entry title}{fg=black}
\setbeamercolor{bibliography entry note}{fg=black}
\setbeamercolor{section in toc}{fg=LancsRed}
\newcommand{\gitem}[1]{\setbeamercolor{item}{fg=ForestGreen}\item[$\checkmark$] #1}
\newcommand{\bitem}[1]{\setbeamercolor{item}{fg=LancsRed}\item[$\times$] #1}
\newcommand{\nitem}[1]{\setbeamercolor{item}{fg=NavyBlue}\item[$-$] #1}
% }}*

% *{{ niceties
\newcommand{\dd}{\mathrm{d}}
\newcommand{\e}{\mathrm{e}}
\newcommand{\ket}[1]{\lvert {#1} \rangle}
\newcommand{\bra}[1]{\langle {#1} \rvert}
\newcommand{\expt}[1]{\langle {#1} \rangle}
\newcommand{\red}[1]{{\bf\color{LancsRed}{#1}}}
\newcommand{\blue}[1]{{\bf\color{NavyBlue}{#1}}}
\newcommand{\green}[1]{{\bf\color{ForestGreen}{#1}}}

% }}*

% *{{ title, subtitle, inst., date
\title{Quantum Monte Carlo studies of \\
       Energy Gaps in Semiconductors}
\subtitle{CMT Group Meeting}

\author{Ryan Hunt}

\date{16$^{\text{th}}$ December, 2016}

% }}*

% *{{ bibliography setup
\usepackage[backend=bibtex]{biblatex}
\bibliography{dmc_bib}
\renewcommand{\footnotesize}{\scriptsize}
\AtEveryCitekey{\clearfield{title}
                \clearfield{pagetotal}
                \clearfield{pages}}

% }}*

% *{{ graphics on front page
\titlegraphic{\vspace*{-2.5cm}\includegraphics[width=3cm]{figs/lan_logo.png}
\hfill\includegraphics[width=3cm]{figs/nownano_logo.png}}

% }}*

% *{{ titlepage + toc

% Section and subsections will appear in the presentation overview
% and table of contents.

\begin{document}

\begin{frame}[plain]
  \titlepage
\end{frame}

\AtBeginSection[]
{ \begin{frame}<beamer>{Outline}
    \tableofcontents[currentsection]
  \end{frame} }

% }}*

% *{{ [S] Why QMC?
\section{Why QMC?}

\subsection{Firstly, what gaps and why?}
% *{{ What gaps and why?
\begin{frame}{What gaps and why?}{}
  \begin{itemize}
  \item There are actually \blue{two} bandgaps one could attempt to calculate

  \item[ \bf \# 1: ] \red{Excitonic Gap}
  \begin{equation}
  \Delta_{Ex} = E^{\text{Pr}}_N-E_N
  \end{equation}
  which is the price, in energy, a photon must pay to create a \blue{bound}
  electron-hole pair in the SC.
  \end{itemize}
  \vfill
  \begin{figure}[H]

  \centering
    \includegraphics[width=0.8\textwidth]{figs/excitonic_gap.pdf}

  \end{figure}

\end{frame}

% }}*

% *{{ Gaps cont.
\begin{frame}{Gaps cont.}{}
  \begin{itemize}

  \item[ \bf \# 2: ] \red{Quasiparticle Gap}
  \begin{equation}
  \Delta_{QP} = E_{N+1} + E_{N-1} - 2E_{N}
  \end{equation}
  the price a photon pays to create a \blue{free} electron, leaving behind a
  hole in the SC.

  \end{itemize}
  \vfill
  \begin{figure}[H]

  \centering
    \includegraphics[width=0.8\textwidth]{figs/quasiparticle_gap.pdf}

  \end{figure}

\end{frame}

% }}*

\subsection{Alternatives: Mean-field theories, $G_{x}W_{y}$, CC/CI}

% *{{ How might we calculate these?
\begin{frame}{How might we calculate these?}

  \begin{block}{Mean-field theories}
    \begin{itemize}
    \item Hartree-Fock, Density Functional Theory and hybrids thereof solve
    auxiliary problems
    \begin{equation}
    \left(-\dfrac{\nabla^2}{2} + v_{\text{eff}}({\bf r})\right) \psi_i({\bf r})
    = \epsilon_i \psi_i({\bf r})
    \end{equation}
    with $v_{\text{eff}}$ an effective potential {\it we do not
    \footnote{And indeed - may never know.} know in general}.
    \item "Determine gaps" from differences in single-particle \red{Kohn-Sham}
    or \red{Hartree-Fock} eigenvalues
    $\epsilon_i({\bf k})$. {\it \color{Gray}Shady.}

    \end{itemize}
  \end{block}

\end{frame}

\begin{frame}{Why is it {\it \color{Gray}Shady}?}
\begin{block}{Hartree-Fock Eigenvalues}
\begin{itemize}
  \item Hartree-Fock eigenvalues can be interpreted as electron addition
  and removal energies (charged excitation energies).
  \item However, HF theory includes \red{absolutely no} description of
  electronic correlation (gaps typically 2 times too big).
\end{itemize}
\end{block}
\begin{block}{Kohn-Sham DFT}
\begin{itemize}
  \item DFT includes {\it some} treatment of electronic correlation.
  \item \red{BUT} it's locked away in an uncontrolled approximation - the {\it
  exchange-correlation functional} - $E_{XC}[n({\bf r})]$.
\end{itemize}
\end{block}
\end{frame}

% }}*

% *{{ QC Methods

\begin{frame}{Quantum Chemistry Methods?}
\begin{block}{Coupled Cluster}
  \begin{itemize}
    \item Based on the exponential ansatz
    \begin{equation}
    \ket{\Psi_{CC}} = \e^{\hat T}\ket{\Psi_{Ref.}}
    \end{equation}
    \item Size-extensive, cost scales as $n_{e}^{6-7}$ depending on particulars
  \end{itemize}

\end{block}

\begin{block}{Full Configuration Interaction}
  \begin{itemize}
    \item Rather than circumventing exponential curse, embrace it
    \begin{equation}
    \ket{\Psi_{FCI}} = \sum_{symm.\ i}c_i\ket{\Psi_i}
    \end{equation}
  \end{itemize}

\end{block}

\end{frame}

% }}*

% *{{ GW Approximation

\begin{frame}{The $G_xW_y$ Approximation}{}
  \begin{itemize}
  \item The \blue{self-energy}, $\Sigma$ can be approximated by the product of
  the Green's Function $G$ and the screened Coulomb potential
  $W$~\footfullcite{hedin1965new, aryasetiawan1998gw}.
  \item If we can do this, we can solve the quasiparticle equation
  \begin{multline}
  \left( -\dfrac{\nabla_i^2}{2} + v_{\text{ion}} + v_{\text{H}} \right)
  \psi_i({\bf r}) + \int\dd {\bf r'}\ \Sigma({\bf r}, {\bf r'}, E^{QP}_i)
  \psi_i({\bf r'})\\ = E^{QP}_i \psi_i({\bf r})
  \end{multline}
  \item As before, but with non-local $\Sigma$ and hope of \blue{systematic
  extension} (via {\it vertex corrections}).
  \end{itemize}
\end{frame}

\begin{frame}{$GW$ vs Perturbation Theory Proper}
\begin{block}{Hedin's Equations\footfullcite{hedin1965new}}
  These are exact statements of many-body perturbation theory\footnote{\ $n
  \equiv ({\bf r}_n,t_n)$. All integrals over all of space-time.}. Don't they
  look tractable.
\footnotesize
  \begin{align}
    G(1,2) &= G_0(1,2) + \displaystyle\int\dd(34)\ G_0(1,3)\Sigma(3,4)G(4,2) \nonumber \\
    \Sigma(1,2) &= i\displaystyle\int\dd(34) W(1,4)G(1,3)\Lambda(3,2,4) \nonumber \\
    \Lambda(1,2,3) &= \delta(1,2)\delta(1,3) \nonumber + \displaystyle\int\dd(4567) \dfrac{\delta \Sigma(1,2)}{\delta G(4,5)}G(4,7)\Lambda(7,6,3)G(6,5)\nonumber \\
    W(1,2)\footnotemark &= v(1,2) + \displaystyle\int\dd(34)\ W(1,3)v(2,4)P(3,4) \nonumber \\
    P(1,2) &= -i\displaystyle\int\dd(34)\ G(2,3) \Lambda(3,4,1) G(4,2^+) \nonumber
  \end{align}
  \footnotetext{NB : some people split this into two, and define a dielectric
  matrix separately. This requires more space + makes business of functional
  derivatives stranger.}
\end{block}
\end{frame}

% }}*

% *{{ General Trends

\subsection{General Trends}
\begin{frame}{Short Summary}{Or, why is this presentation primarily about QMC?}
\begin{block}{Mean-field theories}
  \begin{itemize}
    \gitem \green{Antisymmetry} of (approx.) many-body wavefunction
    \bitem No real hope of systematic extension
    \bitem Even in principle, can't extract gaps -
    $E^{N}_{tot}, n^{N}({\bf r})$ only
  \end{itemize}
\end{block}

\begin{block}{$G_xW_y$}
  \begin{itemize}
  \gitem Systematic extension feasible
  \nitem Better than MFT on average
  \bitem \red{Dependance on $G_0$, etc.}, full SC not cheap (or desirable)
  \end{itemize}
\end{block}

\end{frame}

% }}* frame end

% }}* section end

% *{{ [S] QMC Methods
\section{QMC Methods}

\subsection{Variational MC}

% *{{ VMC
\begin{frame}{QMC Methods\footfullcite{Foulkes2001}}
  \begin{block}{Variational Monte Carlo}
    \begin{itemize}

      \item Endow a \blue{trial wavefunction} with
      variational freedom:
      \begin{equation}
      \Psi({\bf R})=\underbrace{\exp{\left[\ \mathcal{J}_{\{\alpha\}}({\bf R})\
      \right]}}_\text{\green{Our additions}}
      \ \times \underbrace{\mathcal{D}({\bf R})}_\text{\red{DFT, HF, ...}}
      \end{equation}
      and optimise some functional of $\Psi$ by varying $\{ \alpha\}$.

      \item The Jastrow factor, $\exp{\left[ \mathcal{J} \right]}$, allows
      $\Psi({\bf R})$ to satisfy properties of the \blue{exact} many-electron
      wavefunction\footfullcite{kato1957eigenfunctions}.

      %\item MC integration to extract expt. values of $\Psi({\bf R})$.

    \end{itemize}
  \end{block}

\end{frame}

% }}*

\subsection{Diffusion MC}

% *{{ Diffusion MC

\begin{frame}{QMC Methods II}
  \begin{block}{Diffusion Monte Carlo}
    \begin{itemize}
      \item DMC is a stochastic projector-based method for solving\footnote{$
      i\tau = t$ (this is a "Wick rotated" Schr\"odinger equation).}

      \begin{equation}
      \mathcal{\hat H}\ \Psi({\bf R}, \tau)
      = (E_T-\partial_{\tau})\Psi({\bf R}, \tau)
      \end{equation}
      or, if you like
      \begin{equation}
      \Psi({\bf R}, \tau+\Delta \tau)
      = \displaystyle\int G({\bf R} \leftarrow {\bf R'},\Delta \tau)
      \Psi({\bf R'}, \tau) \dd{\bf R'}
      \end{equation}

      \item \green{Separable} - time dependance is exponential
      \footnote{$\{\Phi_i({\bf R})\} \rightarrow$ complete basis of eigenstates of the
      interacting problem.}
      \begin{equation}
      \Psi(0) = \sum_n c_n \Phi_n \implies
      \Psi(\tau) = \sum_n c_n \Phi_n\exp{\left[-(\mathcal{E}_n -
      E_T)\tau\right]}
      \end{equation}
    \end{itemize}
  \end{block}
\end{frame}

\begin{frame}{DMC - cont.}{Key things I won't have enough time to
                           explain properly}
  \begin{itemize}
    \item \blue{Time steps}: don't know interacting system $G({\bf R}
    \leftarrow {\bf R'})$, but know approximation valid for small $\Delta
    \tau$.

    \item \blue{Population control}: number of walkers in a DMC simulation
    fluctuates. Control mechanism introduces (small, controllable) bias.

    \item \red{Finite Size effects}: extrapolation to the thermodynamic limit
    is a necessity, physics of FS critical.
    \item \green{Gaps}: we might expect these problems to matter even less!
  \end{itemize}
\end{frame}
% }}*

\subsection{Excited State DMC}

% *{{ Excited state DMC

\begin{frame}{Excited State DMC}{Wait a second...}
  \begin{itemize}
    \item To make DMC\footnote{Even GS DMC.} workable, we have to \red{fix}
    \blue{the} \green{nodes} of our trial wavefunctions.
    \begin{figure}[H]
      \floatbox[{\capbeside\thisfloatsetup{capbesideposition={left, center},capbesidewidth=0.5\textwidth}}]{figure}[\FBwidth]
      {\caption{2D slice through 321D nodal surface of a 161-e$^-$ system
      studied by Ceperley in 1991.}\label{fig:nodes}}
      {\includegraphics[width=0.3\textwidth]{figs/nodes.pdf}}
    \end{figure}
    \item This is the \blue{only way} we can calculate excitation energies.
    %\item FN error in a GS energy $\rightarrow$ second-order.
    %\item FN error in an XS energy $\rightarrow$ first-order.
  \end{itemize}
\end{frame}

% }}* frame end

% }}* qmc methods section end

% *{{ [S] Some Applications
\section{Some Applications}
\subsection{Small Molecules} % anthracene, benzothiazole, h2/o2??? [problem AE]

% *{{ SM - Anthracene
\begin{frame}{Small Molecules}

  \begin{minipage}{0.50\textwidth}
  \begin{block}{(A)nthracene - C$_{14}$H$_{10}$}

    \begin{figure}[H]
    \centering
    \includegraphics[width=\textwidth]{figs/anth.png}
    \end{figure}

  \end{block}
  \end{minipage}%
  \begin{minipage}{0.50\textwidth}
  \begin{block}{(B)enzothiazole - C$_{7}$H$_{5}$NS}

    \begin{figure}[H]
    \centering
    \includegraphics[width=\textwidth]{figs/bzth.png}
    \end{figure}

  \end{block}
  \end{minipage}

\begin{itemize}
  \gitem Both have a slew of associated $GW$/expt./QC data
  \gitem Both are small enough for geometry to matter (Jahn-Teller distortion)
  \gitem Both are desktop-sized DMC jobs ($N_e \sim$ 50 or so)
\end{itemize}

\end{frame}
% }}* small molecules section end

% *{{ SM - Benzothiazole
\begin{frame}{Small Molecules - cont.}
  \begin{block}{What do we find w/ QMC?}
    \begin{table}[H]
    \centering
    \begin{tabular}{|c||ccc|}
    \hline
    Method &\ Expt. &\ DMC &\ DMC-JT \\ \hline \hline
    $IP^A$ / eV&\ 7.439(6) &\ 7.34(4) &\ 7.28(5) \\ \hline
    $IP^B$ / eV&\ 8.72(5)  &\ 8.92(3) &\ 8.81(3) \\ \hline
    \end{tabular}
    %\caption{Experimental and theoretical ionization potential (IP$ = E_{N-1}
    %- E_N$) for (A) and (B).}
    \label{table:ips}
    \end{table}
  \end{block}
\begin{itemize}
  \item Jahn-Teller effect is important\footnote{Ok, it isn't clear here.
  In our tests of dimers / smaller molecules it is easier to make the
  difference statistically significant.}!
  \item (I would say) it's quite astounding that we can do this with a
  \blue{single determinant}. Quantum chemists would cringe.
\end{itemize}

\end{frame}
% }}*

\subsection{Boron Nitride} % cubic and hexagonal

% *{{ hbn
\begin{frame}{Hexagonal Boron Nitride}

\begin{minipage}{0.5\textwidth}
\begin{block}{Monolayer}
\begin{itemize}
  \item Gaps \red{hard to determine}.
\end{itemize}
\end{block}
\end{minipage}%
\begin{minipage}{0.5\textwidth}
\begin{block}{Bulk}
\begin{itemize}
  \item Gaps largely \green{known}.
\end{itemize}
\end{block}
\end{minipage}
\vfill
\hrule
\vfill
\begin{itemize}
  \item Dominant change after shedding dimensionality is the loss of
  \blue{screening}.
  \item DMC can access this physics, and treat both systems \green{fairly}.
  \item If our bulk results stand up to scrutiny, can reason that we
  have an \green{equally good} description of the physics in 2D (desirable).

\end{itemize}
\end{frame}

\begin{frame}{Hexagonal Boron Nitride}{Continued...}
\begin{itemize}
  \item Some of our monolayer and bulk results
\end{itemize}
\centering

\begin{table}[H]
\centering
%\vspace{0.2cm}
%\caption{Also : The calculated monolayer exciton binding
%is $2.0(3)$ eV, with the bulk value to be determined.}
\label{hbn_table}
\begin{tabular}{|c||cc|}
\hline
System $\rightarrow$ &\ \ \  Bulk/\green{Expt.} \ \ &\ \ \ Monolayer \ \ \\ \hline \hline
\ $\Delta_{Ex}(K_v \rightarrow K_c)$ / eV\ \ \ &\ \ \ $5.8(1)$/\green{5.971}$^\text{9}$ & 8.7(3)  \\
\ $\Delta_{Ex}(K_v \rightarrow \Gamma_c)$ / eV\ \ \ &\ \ $5.69(8)$ & 7.5(3)  \\
\ $\Delta_{Ex}(\Gamma_v \rightarrow \Gamma_c)$ / eV\ \ \ &\ \ \ $7.9(1)$ & --  \\ \hline
\end{tabular}
\end{table}

\begin{itemize}
 \item We see that there is a \blue{significant} enhancement of
  energy gaps on thinning to a monolayer.
  \item \green{In agreement} with the best available
  value \footfullcite{watanabe2004direct}(but again; phonons).
\end{itemize}
\end{frame}



\begin{frame}{Hexagonal Boron Nitride}{An interesting test}
\begin{itemize}
  \item Gaps are \blue{energy differences}, do we necessarily need to
  use computationally demanding sets of time steps?
  \begin{figure}[H]
    \centering
    \includegraphics[width=0.8\linewidth]{figs/dmc_gap_ts_test.pdf}
    \caption{Some gap time step tests, in a (3 3 1) supercell of BhBN.}
    \label{fig:ts_tests}
  \end{figure}
\end{itemize}
\end{frame}
% }}*

% *{{ cbn
\begin{frame}{Cubic Boron Nitride}
\begin{minipage}{0.4\textwidth}
\begin{itemize}
  \item CBN is also a \green{very good insulator}.
  \vspace{0.7cm}
  \item \red{Understudied} w.r.t. hBN.
\end{itemize}
\end{minipage}%
\begin{minipage}{0.6\textwidth}
  \begin{figure}[H]
    \centering
    \includegraphics[width=0.97\linewidth]{figs/cbn.png}
    %\caption{D}
    \label{fig:cbn}
  \end{figure}
\end{minipage}

\begin{itemize}
  \item Some ageing $GW$ studies to compare with\footfullcite{Satta2001,
  Satta2004}, and is of general interest
  given recent discoveries involving C/B/N materials\footfullcite{Pickard2016}.
\end{itemize}
\end{frame}
% }}*

\subsection{Silicon} % excitonic gaps in bulk - should have results then

% *{{ silicon
\begin{frame}{Silicon}{...in the diamond structure}
  \begin{block}{Excitonic Gaps}
    \begin{figure}[H]

    \centering
      \includegraphics[width=\textwidth]{figs/si_bs.pdf}

    \end{figure}
    \begin{itemize} % find reference for this
      \item Exciton binding tiny - $\mathcal{O}(10^{-2})$ eV $ \implies
      \Delta_{Ex} \sim \Delta_{QP}$.
    \end{itemize}
  \end{block}
\end{frame}

\begin{frame}{Silicon cont.}{Preliminary results}
\begin{itemize}
  \item So far, I have some preliminary results which are \red{without a
  treatment of FS errors}\footnote{Actually, we are discovering that FS is
  far less important for excitonic gaps. Don't expect significant change from
  full FS treatment.};
    \begin{table}[H]
    \centering
    \begin{tabular}{|c||ccccc|}
    \hline
    Gap / eV $\searrow$ &\ $L \rightarrow L$ &\ $\Gamma \rightarrow L$  &\
    $\Gamma \rightarrow \Gamma$ &\ $\Gamma \rightarrow X$ &\ $X \rightarrow X$  \\ \hline \hline
    DMC &\ 3.77(4) &\ 2.39(4) &\ 3.57(4)  &\ 1.24(4) &\ 4.55(4) \\ \hline
    \texttt{ioffe}\footnotemark &\ --  &\ 2.0  &\ 3.4  &\ 1.2 &\ -- \\ \hline
    \end{tabular}
    %\caption{Experimental and theoretical ionization potential (IP$ = E_{N-1}
    %- E_N$) for (A) and (B).}
    \label{table:si_gaps}
    \end{table}
\footnotetext{At 300K. These are from numerous sources, which I will omit for
length reasons. Sorry.}
  \item Interestingly, $\Delta_{Ex}$ \green{agree nicely with experiment} (w/
  deviation well within the realm of phononic effects).

  \item \red{BUT} so do $\Delta_{QP}$...
\end{itemize}

\end{frame}

\begin{frame}{Silicon cont.}{A topic of current investigation}
\begin{itemize}
  \item Our proposed FS treatment scheme for \blue{QP Gaps} is;
\begin{align}
  \Delta(N) &= \Delta(\infty) + bv_{M}(N) \nonumber \\
  &{\color{gray}+\ } \color{gray}{c\left( \Delta_{DFT}(N) - \Delta_{DFT}(\infty)  \right)}
\end{align}
where $v_M$ is the \blue{Madelung} constant evaluted for a given supercell
(system size).
  \item $\therefore$ for Si, where we know $\Delta_{Ex} \sim \Delta_{QP}$,
  we don't expect to have the situation we do!
\end{itemize}

\end{frame}

\begin{frame}{Silicon cont.}
  \begin{figure}[H]
    \centering
    \includegraphics[width=0.9\textwidth]{figs/qp_ex_gaps_extrap.pdf}
    \caption{The possible FS behaviours of quasiparticle and excitonic gaps.}
    \label{fig:}
  \end{figure}

\begin{itemize}
  \item \blue{Take-home point} : We know FS effects are more difficult for
  charged excitations. We are \green{open} to the idea that DMC could be
  {\bf over}/{\color{gray}under} estimating QP gaps.\footnote{I am currently
  determining the DMC excitation energies of various small molecules, in order
  to try to get a handle on this.}

\end{itemize}
\end{frame}

% }}* silicon section end

% }}* some applications section end

% *{{ [S] Summary
\section*{Summary}

\begin{frame}{Summary}
  \begin{itemize}
  \item FN-DMC is capable of \blue{predictive} determination of
  energy gaps\footnote{For the pessimists : at least for one important kind of
  gap (excitonic). Investigations ongoing. Monsters may emerge.} for \green{diversely bonded} systems of all
  dimensionalities.
  \vspace{1cm}
  \item The full extent to which computational savings can be made for DMC
  energy gap calculations is \blue{unclear} and we will continue to probe!
  \end{itemize}

%  \begin{itemize}
%  \item
%    Outlook
%    \begin{itemize}
%    \bitem
%    A \red{bad} item
%    \gitem
%      A \green{good} item
%    \end{itemize}
%  \end{itemize}
\end{frame}

% }}*

% *{{ thanks!
\begin{frame}[plain]
\begin{center}
  {\Huge Thank you all for listening!} \\
  %\vspace{2cm}
  %{\Large} Thanks to the NOWNano CDT / EPSRC for my funding.
\end{center}
\end{frame}
% }}*

\end{document}

