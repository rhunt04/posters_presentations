% *{{ Actual preamble
% *{{ preamble
\documentclass[10pt, pdf, hyperref={draft}, usenames, dvipsnames]{beamer}
\usepackage{libertine,textcomp,amssymb,wasysym,color,ulem,verbatim,calc,tikz}
\usepackage{floatrow,subcaption,appendixnumberbeamer,dcolumn,booktabs,multirow}
% hide controls
\usenavigationsymbolstemplate{}
\usepackage[T1]{fontenc}
\synctex=1
\setbeamertemplate{caption}[numbered]

\newcolumntype{d}[1]{D{.}{.}{#1}}
\newcommand\mc[1]{\multicolumn{1}{c}{#1}}

% }}*

% *{{ item colouring and themes
\usecolortheme{beaver}
%\setbeamertemplate{footline}[frame number]
\definecolor{LancsRed}{HTML}{B5121B}
\setbeamercolor{item}{fg=LancsRed}
\setbeamercolor{structure}{fg=black}
\setbeamercolor{bibliography entry author}{fg=LancsRed}
\setbeamercolor{bibliography entry title}{fg=black}
\setbeamercolor{bibliography entry note}{fg=black}
%\setbeamercolor{section in toc}{fg=LancsRed}
%\setbeamercolor{title}{fg=LancsRed}
%\setbeamercolor{frametitle}{fg=LancsRed}
%\setbeamercolor{footnote mark}{fg=LancsRed}
%\setbeamercolor{footnote}{fg=LancsRed}

\setbeamercolor{subsection in toc}{fg=gray}

% Page numbering:
%\setbeamertemplate{footline}[frame number]{}
%\setbeamercolor{footline}{fg=LancsRed}
%\setbeamerfont{footline}{size=\small}

\setbeamertemplate{frametitle}{\nointerlineskip
    \begin{beamercolorbox}[wd=\paperwidth,ht=2.75ex,dp=1.375ex]{frametitle}
        \hspace*{2ex}\insertframetitle \hfill
        {\raisebox{1mm}{\textcolor{gray}{\small$\langle$\ \insertframenumber\
        of
        \inserttotalframenumber\ $\rangle$}}} \hspace*{1ex}%
    \end{beamercolorbox}}

\newcommand{\gitem}[1]{\setbeamercolor{item}{fg=ForestGreen}\item[$\checkmark$] #1}
\newcommand{\bitem}[1]{\setbeamercolor{item}{fg=LancsRed}\item[$\times$] #1}
\newcommand{\nitem}[1]{\setbeamercolor{item}{fg=NavyBlue}\item[$-$] #1}
% }}*

% *{{ margins
\setbeamersize{text margin left=5mm, text margin right=5mm}
% }}*

% *{{ niceties
\newcommand{\dd}{\mathrm{d}}
\newcommand{\e}{\mathrm{e}}
\newcommand{\ket}[1]{\lvert{#1}\rangle}
\newcommand{\bra}[1]{\langle{#1}\rvert}
\newcommand{\expt}[1]{\langle{#1}\rangle}
\newcommand{\red}[1]{{\bf\color{LancsRed}{#1}}}
\newcommand{\blue}[1]{{\bf\color{NavyBlue}{#1}}}
\newcommand{\green}[1]{{\bf\color{ForestGreen}{#1}}}
\newcommand{\orange}[1]{{\bf\color{BurntOrange}{#1}}}
\newcommand{\mred}[1]{{\color{LancsRed}{#1}}}
\newcommand{\mblue}[1]{{\color{NavyBlue}{#1}}}
\newcommand{\mgreen}[1]{{\color{ForestGreen}{#1}}}
\newcommand{\morange}[1]{{\color{BurntOrange}{#1}}}
\newcommand{\kfr}{{\bf k}_{\rm f}}
\newcommand{\kto}{{\bf k}_{\rm t}}
\let\OLDitemize\itemize
\renewcommand\itemize{\OLDitemize\addtolength{\itemsep}{5pt}}
% }}*

% *{{ title, subtitle, inst., date
\title{{\LARGE Excitons in two-dimensional materials from first-principles (by
Quantum Monte Carlo!)}}
%\subtitle{Ryan Hunt \\ CMT Seminar \\ 20$^{\text{th}}$ June}

% authors and institutions
\author{\large Ryan J. Hunt }% \\ 2D Conference 2019 \\ 2$^{\text{nd}}$ July }

\institute[]{\normalsize
Department of Physics \\
Lancaster University}

%\date{today}
\date{}

% }}*

% *{{ bibliography setup
\usepackage[backend=biber, style=phys]{biblatex}
\bibliography{refs}
\renewcommand{\footnotesize}{\scriptsize}
\AtEveryCitekey{\clearfield{title}
                \clearfield{pagetotal}
                \clearfield{pages}
                \clearfield{doi}}
\renewcommand*{\bibfont}{\footnotesize}

% }}*

% *{{ graphics on front page
\titlegraphic{\includegraphics[width=4cm]{figs/lan_logo.png}
\hfill\includegraphics[width=4cm]{figs/nownano_logo.png}}

% }}*

% *{{ titlepage + toc

% Section and subsections will appear in the presentation overview
% and table of contents.

\begin{document}

\begin{frame}[plain]
  \titlepage
\end{frame}

%\begin{frame}{Outline}
%  \tableofcontents
%\end{frame}

% contents at ssubsection starts - comment to get rid
% \AtBeginSection[]
% {\begin{frame}<beamer>{Outline}
%  \tableofcontents[currentsection]
% \end{frame}}

% }}*

% }}* End Preamble

\begin{frame}{Excitons}

\begin{itemize}

  \item Objects are translucent when some of the light incident on them is
  \textit{absorbed}.

  \item Absorption is mediated by individual absorbing units (\blue{$e^{-}$}).

\end{itemize}

\begin{minipage}[t]{0.3\textwidth}

\begin{figure}[H]
  \centering
  \includegraphics[width=\linewidth]{figs/cv_gaps.pdf}
  %\caption{}
%\label{fig:}
\end{figure}

\end{minipage}%
\hfill
\begin{minipage}[t]{0.65\textwidth}

\begin{itemize}
\vspace{0.5cm}
  \item In crystals, \blue{$e^{-}$} live in discrete \textit{energy bands}.  In
  semiconductors, all of the \blue{$e^{-}$} live under a \textit{band
  gap}.\footnotemark
\vspace{0.5cm}
  \item The onset of optical absorption is defined by the position of
  lowest-lying \textit{exciton levels}.
\vspace{0.5cm}
  \item Band theory + independent \blue{$e^{-}$} approx. $\implies$ cannot describe
  excitonic effects.

\end{itemize}


\end{minipage}%
\footnotetext{~~Or $\epsilon_{\text{F}}$ lies in a \textit{band gap}.}

\end{frame}

\begin{frame}{Energy Gaps \& Exciton Binding}

\begin{itemize}
  \item How do we define the exciton binding energy in a way that allows for
  first-principles calculation?
  \item For excitations between ${\kfr}$ and ${\kto}$,
\end{itemize}
\begin{align}
  \Delta_{\text{QP}}(\kfr,\kto) &= \varepsilon_{\text{CBM}}(\kto) -
  \varepsilon_{\text{VBM}}(\kfr), \nonumber \\
   &=E_{{\rm N}+1}(\kto)+E_{{\rm N}-1}(\kfr)-
  E_{\rm N}(\kto)-E_{\rm N}(\kfr), \nonumber \\
  \Delta_{\text{Ex}}(\kfr,\kto) &= E^{+}_{\rm N}(\kfr, \kto) - E_{\rm N},
\end{align}

Most challenging aspects?

\begin{enumerate}
  \item Can only calculate these in finite \textit{supercells}.
  \item $E^{+}_{\rm N}$ is a genuine many-body excited state energy.
\end{enumerate}

\end{frame}

\begin{frame}{Step back: Mott-Wannier Models}

\textbf{Q:} What could we do instead of first-principles modelling?

\begin{itemize}
  \item Invoke effective mass approximation, try to cook up realistic
  electron-hole interaction potentials
  $V$.\footfullcite{Rytova1965,Rytova1967,Keldysh1979}
\end{itemize}

Then, solve
\begin{equation}
  \left[ -\sum^{n}_{\alpha=1}\dfrac{\hbar^2}{2m_{\alpha}}\nabla^{2}_{\alpha}  +
  \sum_{\alpha<\beta} \dfrac{q_i q_j}{4\pi\epsilon_0}V({\bf r}_{\alpha},{\bf
  r}_{\beta})
  \right]\Psi({\bf R}) = E\Psi({\bf R}),\ {\bf R} \equiv ({\bf r}_1,\ldots,{\bf
  r}_n)\label{eq:fpse}
\end{equation}

\begin{itemize}
  \item Neglecting many-particle correlation effects, relying on parameters,
  and \textit{failing} in physically reasonable cases.\footnote{~~Frenkel
  excitons, for example, are perfectly reasonable objects.}
  \item Have done this numerous times for ``reasonable''
  cases.\footfullcite{biex_paper,hbl_paper,tri_paper,ring_paper}
\end{itemize}

\end{frame}

\begin{frame}{Quantum Monte Carlo methods (abridged)\footfullcite{Foulkes2001}}

\begin{itemize}

  \item  In Variational Monte Carlo (VMC), we \textit{guess} a many particle
  wave function, and improve our guess by monitoring observables (whilst
  tweaking parameters $\{\alpha\}$!).
  \begin{equation}
    \Psi({\bf R}) = \exp{[J({\bf R}; \{\alpha\})]} \times D({\bf R}) =
    \Psi({\bf R},0).
  \end{equation}

  \item In Diffusion Monte Carlo (DMC), we take our trial wave function and
  propagate it through imaginary time. Removing excited components, and
  allowing us to sample an estimate of the true energy.\footnote{~~With various
  caveats.}
  \begin{align}
    \Psi_{\text{DMC}}({\bf R},\tau) &= \exp{[-\tau(E_{\rm
    T}-\mathcal{H})]}\Psi({\bf R},0) \\ &~~~\vdots \nonumber \\
    \Psi_{\text{Exact}} &\sim \lim_{\tau \rightarrow \infty}
    \Psi_{\text{DMC}}({\bf R},\tau)
  \end{align}

  \item \red{People don't generally do this because it requires the use of
  supercomputers.}

\end{itemize}

\end{frame}

\begin{frame}{Phosphorene}
A direct gap 2D semiconductor, with large exciton binding
energy.

\begin{figure}[H]
  \centering
  \includegraphics[width=0.6\linewidth]{figs/phos_spec_paper.jpg}
  \caption{PL measurements of Phosphorene $n$-layers.\footnotemark}
\label{fig:phos_spec_paper}
\end{figure}
\footnotetext{\fullcite{yang2015}}

\begin{itemize}
  \item Can we extract the $n=1$ exciton binding energy w/ QMC?
  \item What are the issues that affect such QMC calculations?
\end{itemize}
\end{frame}

\begin{frame}{Phosphorene (cont.)}

\begin{minipage}[t]{0.45\textwidth}

Yes, expt. on suspended monolayers give 0.9(1) eV.\footnotemark\\

We must pay attention to:
\vspace{0.3cm}
\begin{itemize}
  \item Finite-size errors.
  \item Vibrational effects.
  \item Computational efficiency.
\end{itemize}

\end{minipage}%
\hfill
\begin{minipage}[t]{0.5\textwidth}

\begin{figure}[H]
  \centering
  \includegraphics[width=\linewidth]{figs/phos.pdf}
  %\caption{Name}
%\label{fig:name}
\end{figure}

\end{minipage}%
\footnotetext{\fullcite{wang2015,carvalho2016}}


\begin{itemize}
  \item Nuclear vibrations renormalise the QP gap by
  $\mathcal{O}(-0.09\ \text{eV})$ @ 0K.\footnote{~~B.
  Monserrat, private communication (2018). Correction roughly doubles at room temperature.}

  \item We have tested the relaxation of a serious technical assumption in DMC,
  finding that it \textit{does not} affect our results.\footfullcite{Hunt2018}

\end{itemize}

\end{frame}

\begin{frame}{Conclusions: Outlook for QMC}
  \begin{itemize}
    \item It is possible to model excitons (and band gaps!) in 2D materials
    using QMC. Results are in \green{agreement with experiment}, \textit{where
    reasonable}.

    \item QMC calculations are still expensive, but perhaps to a \blue{similar degree
    to alternatives} (e.g.~fully converged $GW$-BSE).

    \item Extraction of more relevant (?) quantities (oscillator strengths,
    lifetimes, full absorption spectra) in solids still a \red{pipe dream}
    for QMC.

    \item Extraction of the same quantities for more \textit{realistic} cases
    (defects, finite carrier concentration, under external fields) is
    \green{perfectly possible}.\footnote{~~Though maybe in a couple of
    centuries.}

  \end{itemize}
\end{frame}

% *{{ End

\begin{frame}{Acknowledgements}

\begin{minipage}[t]{0.45\textwidth}

Lancaster:
\begin{itemize}
  \item N. D. Drummond
  \item M. Szyniszewski
  \item D. M. Thomas
\end{itemize}
\vspace{0.5cm}
\textit{JAIST} (Japan):
\begin{itemize}
  \item G. I. Prayogo
  \item R. Maezono
\end{itemize}

\end{minipage}%
\hfill
\begin{minipage}[t]{0.45\textwidth}

Manchester:
\begin{itemize}
  \item D. A. Ruiz-Tijerina
  \item M. Danovich
  \item V. I. Fal'ko
\end{itemize}

\vspace{0.5cm}
Cambridge:
\begin{itemize}
  \item B. Monserrat
\end{itemize}
\vspace{0.5cm}

\ldots and all the people in my CDT year group.

\end{minipage}%
\vfill
\centering\par\noindent\rule{0.7\textwidth}{0.8pt}

\vfill
\begin{center}
  {\Large Thanks for listening!}
\end{center}
\vfill
\end{frame}

\end{document}
% }}*
