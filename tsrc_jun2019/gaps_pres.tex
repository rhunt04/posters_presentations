% *{{ Actual preamble
% *{{ preamble
\documentclass[10pt, pdf, hyperref={draft}, usenames, dvipsnames]{beamer}
\usepackage{libertine,textcomp,amssymb,wasysym,color,ulem,verbatim,calc,tikz}
\usepackage{floatrow,subcaption,appendixnumberbeamer,dcolumn,booktabs,multirow}
% hide controls
\usenavigationsymbolstemplate{}
\usepackage[T1]{fontenc}
\synctex=1
\setbeamertemplate{caption}[numbered]

\newcolumntype{d}[1]{D{.}{.}{#1}}
\newcommand\mc[1]{\multicolumn{1}{c}{#1}}

% }}*

% *{{ item colouring and themes
\usecolortheme{beaver}
%\setbeamertemplate{footline}[frame number]
\definecolor{LancsRed}{HTML}{B5121B}
\setbeamercolor{item}{fg=LancsRed}
\setbeamercolor{structure}{fg=black}
\setbeamercolor{bibliography entry author}{fg=LancsRed}
\setbeamercolor{bibliography entry title}{fg=black}
\setbeamercolor{bibliography entry note}{fg=black}
%\setbeamercolor{section in toc}{fg=LancsRed}
%\setbeamercolor{title}{fg=LancsRed}
%\setbeamercolor{frametitle}{fg=LancsRed}
%\setbeamercolor{footnote mark}{fg=LancsRed}
%\setbeamercolor{footnote}{fg=LancsRed}

\setbeamercolor{subsection in toc}{fg=gray}

% Page numbering:
%\setbeamertemplate{footline}[frame number]{}
%\setbeamercolor{footline}{fg=LancsRed}
%\setbeamerfont{footline}{size=\small}

\setbeamertemplate{frametitle}{\nointerlineskip
    \begin{beamercolorbox}[wd=\paperwidth,ht=2.75ex,dp=1.375ex]{frametitle}
        \hspace*{2ex}\insertframetitle \hfill
        {\raisebox{1mm}{\textcolor{gray}{\small$\langle$\ \insertframenumber\
        of
        \inserttotalframenumber\ $\rangle$}}} \hspace*{1ex}%
    \end{beamercolorbox}}

\newcommand{\gitem}[1]{\setbeamercolor{item}{fg=ForestGreen}\item[$\checkmark$] #1}
\newcommand{\bitem}[1]{\setbeamercolor{item}{fg=LancsRed}\item[$\times$] #1}
\newcommand{\nitem}[1]{\setbeamercolor{item}{fg=NavyBlue}\item[$-$] #1}
% }}*

% *{{ margins
\setbeamersize{text margin left=5mm, text margin right=5mm}
% }}*

% *{{ niceties
\newcommand{\dd}{\mathrm{d}}
\newcommand{\e}{\mathrm{e}}
\newcommand{\ket}[1]{\lvert{#1}\rangle}
\newcommand{\bra}[1]{\langle{#1}\rvert}
\newcommand{\expt}[1]{\langle{#1}\rangle}
\newcommand{\red}[1]{{\bf\color{LancsRed}{#1}}}
\newcommand{\blue}[1]{{\bf\color{NavyBlue}{#1}}}
\newcommand{\green}[1]{{\bf\color{ForestGreen}{#1}}}
\newcommand{\mred}[1]{{\color{LancsRed}{#1}}}
\newcommand{\mblue}[1]{{\color{NavyBlue}{#1}}}
\newcommand{\mgreen}[1]{{\color{ForestGreen}{#1}}}
\newcommand{\morange}[1]{{\color{BurntOrange}{#1}}}
\newcommand{\kfr}{{\bf k}_{\rm f}}
\newcommand{\kto}{{\bf k}_{\rm t}}
\let\OLDitemize\itemize
\renewcommand\itemize{\OLDitemize\addtolength{\itemsep}{5pt}}
% }}*

% *{{ title, subtitle, inst., date
\title{{\LARGE Quantum Monte Carlo calculations of energy gaps from
first-principles}}
% \subtitle{CMT Seminar \\ X$^{\text{th}}$ Whenever}

% authors and institutions
\author{\large Ryan J. Hunt \\ (sup. Neil Drummond)}

\institute[]{\normalsize
  Department of Physics,\\
  Lancaster University
}

% date
% \date{today}
\date{}

% }}*

% *{{ bibliography setup
\usepackage[backend=biber, style=phys]{biblatex}
\bibliography{refs}
\renewcommand{\footnotesize}{\scriptsize}
\AtEveryCitekey{\clearfield{title}
                \clearfield{pagetotal}
                \clearfield{pages}
                \clearfield{doi}}
\renewcommand*{\bibfont}{\footnotesize}

% }}*

% *{{ graphics on front page
\titlegraphic{\includegraphics[width=4cm]{figs/lan_logo.png}
\hfill\includegraphics[width=4cm]{figs/nownano_logo.png}}

% }}*

% *{{ titlepage + toc

% Section and subsections will appear in the presentation overview
% and table of contents.

\begin{document}

\begin{frame}[plain]
  \titlepage
\end{frame}

%\begin{frame}{Outline}
%  \tableofcontents
%\end{frame}

% contents at ssubsection starts - comment to get rid
% \AtBeginSection[]
% {\begin{frame}<beamer>{Outline}
%  \tableofcontents[currentsection]
% \end{frame}}

% }}*

% }}* End Preamble

% *{{ Problem
\begin{frame}{What's the problem?}
We'd like to be able to predictively model the (opto)electronic behaviour of
materials. Because this could be {\it useful}.
\vfill
\begin{minipage}[t]{0.55\textwidth}
\vspace{1.2cm}
\begin{itemize}
  \item Specifically, $\Delta_{\text{Ex.}}$, $\Delta_{\text{QP}}$, and $E^{\rm
  X}_{\rm B}$ in semiconductors.\footnotemark
  \item Don't define ``material heaven'', but are a start.
  \item (The blue LED is blue for a reason.)
\end{itemize}
\end{minipage}%
\hfill
\begin{minipage}[t]{0.4\textwidth}
\vspace{0pt}
\begin{figure}[H]
  \centering
  \includegraphics[width=0.75\linewidth]{figs/cv_gaps.pdf}
  %\caption{Name}
\label{fig:name}
\end{figure}
\end{minipage}%
\footnotetext{\fullcite{Hunt2018}}
\end{frame}

\begin{frame}{Energy gaps: definitions}

The \blue{quasiparticle gap}, $\Delta_{\text{QP}}$, is defined as the
difference between the \mgreen{CBM} and the \mred{VBM}:
\begin{align}
  &\Delta_{\text{QP}}(\kfr,\kto)=\mgreen{{\cal E}_{\text{CBM}}(\kto)}-
  \mred{{\cal E}_{\text{VBM}}(\kfr)}
  \nonumber \\
  &=\mgreen{\left[E_{{\rm N}+1}(\kto)-E_{\rm N}(\kto)\right]}-
    \mred{\left[E_{\rm N}(\kfr)-E_{{\rm N}-1}(\kfr)\right]} \nonumber \\
  &=E_{{\rm N}+1}(\kto)+E_{{\rm N}-1}(\kfr)-E_{\rm N}(\kto)-E_{\rm N}(\kfr),
\end{align}
\vfill
The \blue{excitonic gap}, $\Delta_{\text{Ex.}}$, is defined as the energy
difference between an \morange{excited} $\rm N$-electron state and the ground
$\rm N$-electron state:
\begin{equation}
  \Delta_{\text{Ex.}}(\kfr,\kto) = \morange{E^{+}_{\rm N}(\kfr, \kto)} - E_{\rm N},
\end{equation}

Their difference is the \blue{exciton binding}
($\Delta_{\text{QP}}\geq\Delta_{\text{Ex.}}$, in TD limit).\footnote{~~The interaction
energy of a quasielectron at $\kto$ and a quasihole at $\kfr$. Many-body
Bloch conditions restrict us!}
\end{frame}

% }}*

\begin{frame}{Excited-state QMC}
\begin{block}{Variational Principles}
Excitonic gaps:
\begin{itemize}
  \item In VMC, if $\psi_{\rm T}({\bf R};{\{ \alpha\}})$ has some symmetry
  $\forall \{\alpha\}$, have VP w.r.t.\ lowest energy state of given symmetry.
  \item In FN-DMC, such symmetry may be \red{broken} (nodes are not
  enough!).\footfullcite{Foulkes1999}
  % Specific transformation properties of $\psi_{\rm T}$ needed to
  % maintain.
\end{itemize}
\end{block}
\begin{block}{Implications: the practical upshot}
\begin{itemize}
  \item $\Delta_{\text{QP}}$: all ground states. \green{Effective VP} on QP
  gap.
  \item $\Delta_{\text{Ex.}}$: vulnerable to variational collapse unless
  symmetry constraint met. \textbf{Beware}.\footnote{~~Safe if one
  \textit{constructs} determinantal expansion with target symmetry class.}
\end{itemize}
\end{block}
\end{frame}


\begin{frame}{Finite-size effects}
\begin{block}{Semiconductors}
\begin{itemize}
  \item Single-particle FS negligible, Coulomb or many-body FS
  dominant.
\end{itemize}
\end{block}
\begin{block}{Gaps}
  \begin{itemize}
    \item $\Delta_{\text{QP}}$: a difference in QP energies\ldots
    \begin{itemize}
      \item Leading-order FS error in each calc. $\sim v_M$ assoc. with
      the \textit{quasiparticles} in their host
      \textit{environment}.\footnote{~~Not the ``bare'' $v_M$ which goes into the
      Ewald interaction\ldots}
      \item Remainder negligible in solids, but appears systematic
      (charge-quadrupole) in 2D materials (expect from
      Makov-Payne\ldots\footfullcite{Makov1995}).
    \end{itemize}
  \end{itemize}
\end{block}

\begin{minipage}[t]{0.65\textwidth}
\begin{itemize}
  \item $\Delta_{\text{Ex.}}$: energy needed to create exciton
\end{itemize}

\textit{If} cell big enough \\
\ \ $\implies$ FSE $\sim$ lattice of image-excitons.\\
\textit{Else} \\
\ \ $\implies$ FSE $\sim$ uncorrelated QPs (exciton \textit{squeezed}).


\end{minipage}%
\hfill
\begin{minipage}[t]{0.35\textwidth}

\begin{figure}[H]
  \centering
  \includegraphics[width=0.65\linewidth]{figs/per_full.pdf}
  %\caption{}
\end{figure}

\end{minipage}%

\end{frame}


\begin{frame}{Miscellany}
\begin{itemize}

  \item ``Excited state QMC'' isn't just gaps. Lifetimes, dip. mom.,
  etc.\footfullcite{Danovich2018,Barnett1992}

  \begin{minipage}[t]{0.20\textwidth}

  \begin{figure}[H]
    \centering
    \includegraphics[width=0.8\linewidth]{figs/hbl_teaser.pdf}
    %\caption{df}
  \end{figure}

  \end{minipage}%
  \hfill
  \begin{minipage}[t]{0.65\textwidth}

  \begin{figure}[H]
    \centering
    \includegraphics[width=0.8\linewidth]{figs/doh_diag.png}
    %\caption{df}
  \end{figure}

  \end{minipage}%

  \item Won't say anything about ``intraband'' excitations. Excitations in
  metals (HEG, parameterising FLT) studied
  elsewhere.\footfullcite{Kwon1994,Drummond2013,Drummond2013F}


\end{itemize}
\end{frame}

\begin{frame}{Some Results: Atoms and Molecules}

\begin{block}{Neon}
  Is single-determinant only inherently bad? $1^{\text{st}}\ldots8^{\text{th}}$
  IP of Ne (MAE):
  \begin{equation}
    \underbrace{0.83\%}_{\text{SJ-VMC}} \rightarrow
    \underbrace{0.67\%}_{\text{SJB-VMC}} \rightarrow
    \underbrace{0.38\%}_{\text{SJ-DMC}} \rightarrow
    \underbrace{0.34\%}_{\text{SJB-DMC}}
  \end{equation}
  no real trend amongst the individual IPs, save for larger-than-typical error
  on 7th IP (3-e$^{-}$!).
\end{block}

\begin{block}{O$_2$ dimer}

Electronic ground state triplet: ($^{3}\Sigma^{-}_g$).  Singlet-triplet
splitting around 0.9773 eV.

\begin{itemize}
  \item SD SJ-DMC: 1.62(2) eV, MD-$\Delta$ SJ-DMC: 0.20(3) eV.
  \item Other little tests;
  \begin{itemize}
    \item PP ($\mathcal{O}(0.3\ \text{eV})$ in IP).
    \item Geometry ($\mathcal{O}(0.5\ \text{eV})$ in IP).
  \end{itemize}
\end{itemize}

\end{block}

\begin{block}{H$_2$ dimer}

  \begin{itemize}
    \item Full-quantum electron/proton DMC calculation. See paper. More to
    come\ldots?
  \end{itemize}
\end{block}

\end{frame}


\begin{frame}{Some Results: Bulk Solids}

\begin{minipage}[t]{0.45\textwidth}

\begin{block}{Silicon}
  Previous studies looked at Si. SD SJ calculations, claimed good
  agreement with expt.\footnotemark \\
  \begin{itemize}
    \item Revisit suggests FS effects hinder prior study ($\mathcal{O}(0.2\
    \text{eV})$).
    \item Inclusion of backflow ($\mathcal{O}(0.1\ \text{eV})$) (\& re-opt. --
    further $\mathcal{O}(0.1 \text{eV})$) crucial for good QP energies.
  \end{itemize}
\end{block}

\end{minipage}%
\hfill
\begin{minipage}[t]{0.50\textwidth}

\begin{figure}[H]
  \centering
  \includegraphics[width=\linewidth]{figs/si_gaps.pdf}
  \caption{Uncorrected SJ-DMC gaps in Si.}
\label{fig:si_gaps}
\end{figure}

\end{minipage}%
\footnotetext{\fullcite{Williamson1998}}

\begin{itemize}
  \item $\Delta_{\text{Ex.}}$ studied before, \textbf{but} cannot directly
  study $\Gamma \rightarrow 0.85{\rm X}$ excitation.
\end{itemize}


\end{frame}


\begin{frame}{Some Results: Phosphorene}

\begin{minipage}[t]{0.45\textwidth}

\begin{figure}[H]
  \centering
  \includegraphics[width=0.9\linewidth]{figs/phos.pdf}
  %\caption{}
%\label{fig:}
\end{figure}

\end{minipage}%
\hfill
\begin{minipage}[t]{0.45\textwidth}

\begin{itemize}

  \item FS treatment: Keldysh-screened $v_{\text{M}}$ necessary for QP.
  Keldysh-exponent charge-quadrupole necessary for Ex.\footnotemark

  \item Backflow doesn't lower gaps (QP or Ex.) by statistically significant
  amt.

  \item Prelim. vibrational corrections: ZP ($\geq$)-0.07 eV, 300K -0.17
  eV.
\end{itemize}

\end{minipage}%
\footnotetext{\fullcite{Keldysh1979}}
\vfill


NB another QMC study on phosphorene exists. Authors there also considered HW
BCs (make dominant FS error kinetic,
$\mathcal{O}(L^{-2})$).\footfullcite{Frank2019} Generally, need to be careful
doing this.\footfullcite{Drummond2005}
\end{frame}


% \begin{frame}{Some Results: Bulk hBN}
% \begin{itemize}
%   \item FS errors: highly anisotropic dielectric environment\ldots Can show (by
%   principal axis transformation) that screened Madelung constant is
%   \begin{equation}
%     {\tilde v}_{\text{M}}(\textbf{S})=\dfrac{1}{\sqrt{\text{det}(\epsilon)}}
%     v_{\text{M}}(\epsilon^{-\frac{1}{2}}\textbf{S})
%   \end{equation}
%   with $\epsilon$ the (high-frequency) dielectric tensor.\footnote{~~We take this
%   from DFPT, it depends weakly on the functional, and the ultimate difference
%   in ${\tilde v}_{\text{M}}$ is much less than the statistical uncertainty.}
% \end{itemize}
%
% \begin{minipage}[t]{0.45\textwidth}
%
% \begin{itemize}
%   \item ${\tilde v}_{\text{M}}$ subtraction eliminates dominant FS effect,
%   remnants exist.
%   \item Present case issue is worse: can't use same supercells for each gap
%   (${\rm K},\Gamma$ co-existence in grid of ${\bf k}$-vectors)
% \end{itemize}
%
% \end{minipage}%
% \hfill
% \begin{minipage}[t]{0.45\textwidth}
%
% \begin{figure}[H]
%   \centering
%   \includegraphics[width=0.9\linewidth]{figs/bulk_gaps.eps}
%   %\caption{}
% \label{fig:hbn_bulk}
% \end{figure}
%
% \end{minipage}%
%
% \end{frame}

\begin{frame}{Some Results: Monolayer hBN}
  A few of you know of these calculations, Neil (Drummond) performed years
  ago.
  \begin{itemize}
    \item FS treatment: as in phosphorene.

    \item Vibrations: ZP correction -0.54 eV (-0.73 eV @ 300K)
    (similar for ${\rm K} \rightarrow {\rm K}\ \text{or}\ \Gamma$).
  \end{itemize}

\begin{minipage}[t]{0.45\textwidth}

\begin{figure}[H]
  \centering
  \includegraphics[width=0.9\linewidth]{figs/bn_gap_v_NP_DMC_CPP.eps}
\end{figure}
\end{minipage}%
\hfill
\begin{minipage}[t]{0.45\textwidth}

\begin{itemize}
  \item Systematic $1/N$ FS effect recovered after subtraction of ${\tilde
  v}_{\text{M}}$ from $\Delta_{\text{QP}}$. $\Delta_{\text{Ex.}}$, and
  corrected $\Delta_{\text{QP}}$ same systematic FS. % (charge-quadrupole under
  %the screened interaction!).

  %\item Same ${\bf k}$-vector constraint for $\Delta_{\text{Ex.}}({\rm K}
  %\rightarrow \Gamma)$.

  \item $E^{\rm X}_{\rm B}({\rm K} \rightarrow \Gamma)$=1.9(4) eV

  \item $E^{\rm X}_{\rm B}({\rm K} \rightarrow {\rm K})$=1.8(4) eV

  \item (Wirtz \textit{et al.} $GW$-BSE: 2.1 eV)

\end{itemize}

\end{minipage}%

\end{frame}

\begin{frame}{Conclusions}

\begin{block}{QMC Specifics}
\begin{itemize}
  \item Non-negligible comp. savings possible: time step bias in a gap is
  almost imperceptible (definitely is at $\mathcal{O}(0.1\
  \text{eV})$).\footnote{~~Potential further savings are in localization of
  low-lying states.}
  \item FS effects accuracy limiting.
  \item Best possible QP energies (safely) rely on best possible individual wfns.
  \item FN error from excitonic gaps cannot be too high: all else accounted
  for, we've never done too badly. We have been
    \textbf{super-conservative}.
\end{itemize}
\end{block}

\begin{block}{Gap Generalities}
\begin{itemize}

  \item Vibrations: for gaps, an electronic theory alone is not enough.

\end{itemize}
\end{block}

\end{frame}


% *{{ OLD CONTENT
%%     % *{{ Excited state DMC
%%
%%     \begin{frame}{Excited States}
%%     Briefly:
%%     \begin{itemize}
%%       \item  QP gap $\rightarrow E_{{\rm N}, {\rm N}\pm1}$ (VP on each
%%       \textit{ground} state).
%%       \item Ex. gap $\rightarrow$ \textit{may} have VP on $E^{+}$. \textit{May} only
%%       have at VMC level. FN-DMC VP obtained in special
%%       circumstances.~\footfullcite{Foulkes1999}
%%       \item \textcolor{gray}{(FN
%%       constraint means effective VP)}
%%     \end{itemize}
%%     \end{frame}
%%
%%     % \begin{frame}{Excited State DMC}{Wait a second... what I said applies to ground
%%     % states.}
%%     % \vspace{0cm}
%%     % In special circumstances, what I have just said
%%     % also applies to the ground state {\it of a given
%%     % symmetry}.\footnotemark~Key idea concerns
%%     % excited-states and generalised variational principles
%%     % \only<2->{\footnotetext{~~Actually, the variational principle on ground states
%%     % in fixed-node DMC is an excited state one, fermion GS transforms as 1D irrep.
%%     % of permutation group.}}
%%     %
%%     %   \begin{itemize}
%%     %   \pause
%%     %     \item \red{Jargon}: If trial wfn. transforms as a 1D irrep. of the symmetry group
%%     %     of the Hamiltonian, have a variational principle on states of that
%%     %     symmetry (could be excited state).~\only<2->{\footfullcite{Foulkes1999}}
%%     %     \pause
%%     %     \item \green{Simple practical upshot}: the many-body Bloch conditions for periodic
%%     %     calculations mean that states with definite crystal momentum belong to a 1D
%%     %     irrep. (of the translation group).
%%     %   \end{itemize}
%%     % \vfill
%%     % There are {\it many} other means of obtaining variational
%%     % bounds.\only<3->{~\footfullcite{Zhao2016,MacDonald1933,Hipes2011}}
%%     % \end{frame}
%%
%%     % }}* frame end
%%
%%     \section{Case studies}
%%
%%     % \subsection{Atoms and molecules}
%%
%%     % % *{{ Atoms and molecules
%%     %
%%     % \begin{frame}{Intrinsic accuracy: Ne atom}
%%     %
%%     % The neon atom ground state is (famously) closed shell. None of the cations
%%     % are, and there's a potential for build-up of nodal\footnote{~~FN-error
%%     % $\equiv$ error arising from (local) B.C. on $\Psi$ ($\Psi=0$ ``surface''
%%     % fixed to please Pauli).}
%%     % error in IP.~\footfullcite{Rasch2014,Kulahlioglu2014}
%%     % \begin{equation}
%%     % \text{IP}(n) = E(n) - E(n-1),
%%     % \end{equation}
%%     % Q: Does this happen?
%%     % \begin{figure}[H]
%%     %   \floatbox[{\capbeside\thisfloatsetup{capbesideposition={left, center},
%%     %   capbesidewidth=0.5\textwidth}}]{figure}[\FBwidth]
%%     %   {\caption{DMC and FCI correlation energies of atoms versus atomic
%%     %   number (from Rasch {\it et al.}). NB 30mHa $\sim$ 0.8
%%     %   eV.}\label{fig:nodal_error}}
%%     %   {\includegraphics[width=0.5\textwidth]{figs/nodal_error.png}}
%%     % \end{figure}
%%     % \end{frame}
%%     %
%%     % \begin{frame}{Ne atom - cont.}
%%     %   ``Exact'' $\implies$ expt. minus best est. relativistic correction.
%%     %   \footfullcite{Chakravorty1993}
%%     %   \begin{table}[H]
%%     %   \centering
%%     %   \begin{tabular}{c *{5}{d{3.5}}}
%%     %   \multirow{2}{*}{$n$} & \multicolumn{5}{c}{IP($n$) (eV)} \\
%%     %   & \mc{Exact}&\mc{SJ-VMC}&\mc{SJB-VMC}&\mc{SJ-DMC}&\mc{SJB-DMC}\\ \hline
%%     %   1& 21.61333&22.08(2)&21.96(2) &21.72(1) &21.72(1) \\
%%     %   2& 40.99110&41.48(2)&41.39(2) &41.10(1) &41.06(1) \\
%%     %   3& 63.39913&63.44(2)&63.23(1) &63.35(2) &63.39(1) \\
%%     %   4& 97.29312&97.91(2)&97.78(1) &97.75(2) &97.72(1) \\
%%     %   5&126.28846&126.85(2)&126.72(2)&126.85(1)&126.79(1)\\
%%     %   6&157.80001&158.43(2)&158.30(1)&158.25(2)&158.34(1)\\
%%     %   7&207.04137&204.48(2)&204.56(1)&205.04(2)&205.26(1)\\
%%     %   8&238.78949&238.10(1)&238.49(1)&238.70(2)&238.79(1)\\
%%     %   \hline
%%     %   \mc{\text{MAE}} & \mc{0\%} & 0.83\%    & 0.67\%    & 0.38\%    & 0.34\%
%%     %   \end{tabular}
%%     %   \end{table}
%%     % \vspace{-0.2cm}
%%     % So? IPs related to QP energies. More later!
%%     % \end{frame}
%%     %
%%     % \begin{frame}{Molecules - dimers}
%%     % \begin{itemize}
%%     %   \item O$_2$: singlet/triplet, (near)degeneracy, multideterminants.
%%     % \end{itemize}
%%     % \begin{figure}[H]
%%     %   \centering
%%     %   \includegraphics[width=0.6\linewidth]{figs/o2_dimer/o2_dimer.pdf}
%%     % \end{figure}
%%     % \vfill
%%     % \begin{itemize}
%%     %   \item H$_2$: vibrations, quantum protons.
%%     % \end{itemize}
%%     % \begin{figure}[H]
%%     %   \centering
%%     %   \includegraphics[width=0.5\linewidth]{figs/h2_dimer/h2_dimer.pdf}
%%     % \end{figure}
%%     % \end{frame}
%%     %
%%     % \begin{frame}{Dimers (cont.)\footfullcite{Hunt2018}}
%%     % \begin{figure}[H]
%%     %   \floatbox[{\capbeside\thisfloatsetup{capbesideposition={left, center},
%%     %   capbesidewidth=0.3\textwidth}}]{figure}[\FBwidth]
%%     %   {\caption{O$_2$ and H$_2$ dimer results.}\label{fig:dimers}}
%%     %   {\includegraphics[width=0.6\textwidth]{figs/dimers.png}}
%%     % \end{figure}
%%     %
%%     % \begin{itemize}
%%     %   \item For O$_2$, also determined singlet-triplet splitting of 1.62(2)
%%     %   [0.20(3)] eV, c.f.
%%     %   0.9773 eV.
%%     %   \item $\mathcal{D}$/$\Sigma_i\mathcal{D}_i$ and ability of QMC
%%     %   to treat on equal footing is key!
%%     % \end{itemize}
%%     % \end{frame}
%%     %
%%     % \begin{frame}{Molecules - nondimers}
%%     % In terms of computational cost, QMC for small molecules (${\rm N}_e \sim 100$) is roughly on par with
%%     % $GW$(-BSE). Recently, a slew of such studies have focussed on studying small
%%     % molecules.
%%     % \begin{figure}[H]
%%     %   \centering
%%     %   \includegraphics[width=0.7\linewidth]{figs/combo.png}
%%     %   \caption{C$_{14}$H$_{10}$ $\rightarrow$ C$_6$N$_4$ $\rightarrow$ C$_7$H$_5$NS
%%     %   $\rightarrow$ BF$_3$.}
%%     % \label{fig:combo}
%%     % \end{figure}
%%     % \begin{itemize}
%%     %   \item Results vary significantly, and are $G_0$, $W_0$, $v_{\text{XC}}$
%%     %   \ldots dependent.
%%     %   \item Is QMC a worthy competitor in the realm of small molecules?
%%     % \end{itemize}
%%     % Example use: $T(E)$ in molecular electronics, for example. Determines $I$-$V$
%%     % characteristics of a molecular junction.
%%     % \end{frame}
%%     %
%%     % \begin{frame}{Nondimers (cont.)}
%%     % \begin{minipage}[t]{0.45\textwidth}
%%     % Anthracene (C$_{14}$H$_{10}$):
%%     % \begin{itemize}
%%     %   \item $\Delta_{\text{Ex.}}^{S=0,1}=3.07(3),2.36(3)$ eV. (PAH \&
%%     %   \green{Hund}!).
%%     %   \item Expt. says 3.38-3.433,
%%     %   1.84-1.85$^{\star}$ eV.
%%     %   \item No comparable $GW$.\footnotemark \\
%%     %   \hspace{2cm}$\downarrow$
%%     % \end{itemize}
%%     % \end{minipage}%
%%     % \hfill
%%     % \begin{minipage}[t]{0.45\textwidth}
%%     % Benzothiazole (C$_7$H$_5$NS):
%%     % \begin{itemize}
%%     %   \item IP = $\underbrace{8.92(2)}_{\rm V}\rightarrow \underbrace{8.80(2)}_{\rm A}$ eV
%%     %   \item Expt. says 8.72(5) eV [$\rm A$].
%%     %   \item $GW$ says 8.2-8.5 eV [$\rm V$].
%%     % \end{itemize}
%%     % \end{minipage}%
%%     % \footnotetext{~~Hummer \textit{et al.} studied mol.
%%     % cry. anthracene: {\bf PRL 92(14)}, (2004). Unreferenced numbers found (w/ refs)
%%     % in our article, excluded for space.}
%%     % \vfill
%%     % \begin{itemize}
%%     %   \gitem QMC as tool for probing \blue{molecular excitons}. Relevant in
%%     %   materials screening for biological imaging, for example. (FRET).
%%     %   \bitem Relevant quantities are (really) resonance timescales.
%%     % \end{itemize}
%%     % \end{frame}
%%     % % }}*
%%
%%     \subsection{Bulk solids}
%%
%%     % *{{ Bulk solids
%%
%%     \begin{frame}{Bulk solids}
%%
%%     We've studied Si, $\alpha$-SiO$_2$, and cubic BN in the current work. Previous
%%     QMC studies had claimed success in evaluation of ``QMC band
%%     structures'',~\footfullcite{Kent1998,Williamson1998}
%%     minus discussions of:
%%
%%     \begin{minipage}[t]{0.35\textwidth}
%%     \vspace{0.3cm}
%%     \begin{itemize}
%%       \item Finite-size errors.
%%       \item Fixed-nodal errors.
%%       \item $\Delta_{\text{QP}}$ vs. $\Delta_{\text{Ex.}}$.
%%     \end{itemize}
%%     \vspace{0.3cm}
%%     Will concentrate on Si here, exploring the above.
%%     \end{minipage}%
%%     \hfill
%%     \begin{minipage}[t]{0.55\textwidth}
%%     \begin{figure}[H]
%%       \centering
%%       \includegraphics[width=0.9\linewidth]{figs/si.png}
%%       %\caption{Si}
%%     \label{fig:silicon}
%%     \end{figure}
%%     \end{minipage}%
%%     \end{frame}
%%
%%     \begin{frame}{Bulk solids: FS errors}
%%     \begin{itemize}
%%       \item Able only to simluate a finite \textit{chunk} of material
%%       (supercell), under PBCs. Excitations ``1/${\rm N}$'' effects. Need statistical
%%       accuracy + careful FS treatment.
%%     \end{itemize}
%%     \begin{figure}[H]
%%       \floatbox[{\capbeside\thisfloatsetup{capbesideposition={right, center},
%%       capbesidewidth=0.5\textwidth}}]{figure}[\FBwidth]
%%       {\caption{Uncorrected SJ-DMC gaps of Si. FS effect characteristic and
%%       quantifiable, largely from image-interactions.}\label{fig:si_gaps}}
%%       {\includegraphics[width=0.45\textwidth]{figs/si_gaps.pdf}}
%%     \end{figure}
%%     \begin{itemize}
%%       \item Then why do $\Delta_{\text{QP}}$ \& $\Delta_{\text{Ex.}}$ behave same?
%%     \end{itemize}
%%     \end{frame}
%%
%%     \begin{frame}{Bulk solids: Fixed-nodal error}
%%     \begin{itemize}
%%       \item Probe with Backflow transformation:
%%       \begin{equation}
%%         {\bf r}_i \rightarrow {\bf x}_i = {\bf r}_i + {\boldsymbol \xi}_i({\bf R})
%%       \end{equation}
%%       which can change nodal surface.~\footfullcite{lopezrios2006}
%%       \item Tested $\Delta_{\text{QP/Ex}}(\Gamma_{\rm v} \rightarrow \Gamma_{\rm
%%       c})$ and $\Delta_{\text{QP}}(\Gamma_{\rm v} \overset{\sim}{\rightarrow}
%%       \text{CBM})$, in $2\times2\times2$ supercell.
%%       \item We find that backflow leads to a reduction in gaps, of at least 0.2
%%       eV, but upto 0.3--0.4 eV when one re-optimises ${\boldsymbol
%%       \xi}_i$.\footnote{~~C.f. controllable uncertainty: $\mathcal{O}(0.1\
%%       \text{eV})$ for each of pseudopotentials, statistics, NLO FS effects (?).}
%%     \end{itemize}
%%     \end{frame}
%%
%%     % \begin{frame}{Bulk solids: Fixed-nodal error (cont.)}
%%     % \begin{block}{We find:}
%%     %  \begin{itemize}
%%     %    \item Inclusion of backflow correlations:
%%     %    \begin{itemize}
%%     %      \item Lowers $\Delta_{\text{QP}}(\Gamma_{\rm v} \rightarrow \Gamma_{\rm
%%     %      c})$ by 0.13 eV.
%%     %      \item Lowers $\Delta_{\text{Ex.}}(\Gamma_{\rm v} \rightarrow \Gamma_{\rm
%%     %      c})$ by 0.12 eV.
%%     %    \end{itemize}
%%     %    \item Re-optimisation of excited-state backflow functions (${\boldsymbol
%%     %    \xi}_i({\bf R})$) further lowers:
%%     %    \begin{itemize}
%%     %      \item $\Delta_{\text{QP}}(\Gamma_{\rm v} \rightarrow \Gamma_{\rm
%%     %      c})$ by 0.18 eV.
%%     %      \item $\Delta_{\text{QP}}(\Gamma_{\rm v} \overset{\sim}{\rightarrow}
%%     %   \text{CBM})$ by 0.20 eV.
%%     %    \end{itemize}
%%     %  \end{itemize}
%%     % \end{block}
%%     %  \vfill
%%     %  \red{Point:} FN errors \red{not} insignificant ($\gtrsim 0.2$ eV), but can be
%%     %  remedied\footnote{~~Always safe for QP gaps. Sometimes safe for Ex. gaps.} w/ backflow.
%%     % \end{frame}
%%
%%     %%  \begin{frame}{Bulk solids: $\Delta_{\text{QP}}$ and $\Delta_{\text{Ex.}}$}
%%     %%  In bulk Si $E^{\rm X}_B \sim 15$ meV.
%%     %%  \begin{itemize}
%%     %%    \item Expect $\Delta_{\text{QP}} \sim \Delta_{\text{Ex.}}$ (in TD limit).
%%     %%    \item (potentially) qualitatively different FS effects, however.
%%     %%    \item \ldots but already seen that this doesn't happen in Si.
%%     %%  \end{itemize}
%%     %%  \vfill
%%     %%  \begin{block}{Explanation:}
%%     %%    Confine an exciton too tightly, $E_{\text{K}}$ dominates, no binding, behave
%%     %%    effectively uncorrelated. Balance of supercell length scale and exciton
%%     %%    length scale important.
%%     %%  \vfill
%%     %%    $\implies$ $a^{\star}_{\rm B}$ in Si is \blue{huge}. Cell never big enough to
%%     %%    accomodate ${\rm X}$.
%%     %%  \end{block}
%%     %%  \end{frame}
%%
%%     % \begin{frame}{Bulk solids: the old calculations (briefly)}
%%     % No $q_{\bf k} \implies \Delta \uparrow$ \hfill
%%     % Bad wfn. $\implies \Delta \uparrow$ \hfill
%%     % FS effects $\implies \Delta \downarrow$
%%     % \begin{figure}[H]
%%     %   \centering
%%     %   \includegraphics[width=0.8\linewidth]{figs/perspective.jpg}
%%     %   \caption{Perspective.}
%%     % \label{fig:perspective}
%%     % \end{figure}
%%     % {\tiny \ldots and if you're trying to justify the need for many-body theory,
%%     % you'd best not connect $\Delta_{\text{QP}}$ and
%%     % $\Delta_{\text{Ex.}}$ by means of a M.-W. model!}
%%     % \end{frame}
%%     % }}*
%%
%%     \subsection{Phosphorene}
%%
%%     % *{{ Phosphorene
%%
%%     \begin{frame}{Phosphorene}
%%     A direct gap 2D semiconductor, with large exciton binding
%%     energy.\footfullcite{yang2015}
%%
%%     \begin{figure}[H]
%%       \centering
%%       \includegraphics[width=0.6\linewidth]{figs/phos_spec_paper.jpg}
%%       \caption{PL measurements of Phosphorene $n$-layers.}
%%     \label{fig:phos_spec_paper}
%%     \end{figure}
%%
%%     \begin{itemize}
%%       \item Do \red{not} expect $\Delta_{\text{QP}} \sim \Delta_{\text{Ex.}}$
%%       \item FS effects in $\Delta_{\text{QP/Ex.}}$ much more important.
%%     \end{itemize}
%%     \end{frame}
%%
%%     % \begin{frame}{Physics of FS effects in 2D}
%%     % \begin{minipage}[t]{0.5\textwidth}
%%     % \vfill
%%     % \begin{center}
%%     %   \begin{tikzpicture}
%%     %     \node<1> (img1) {\includegraphics[width=.9\linewidth]{figs/per_exc_fig/free.pdf}};
%%     %     \node<2> (img2) {\includegraphics[width=.9\linewidth]{figs/per_exc_fig/free_box.pdf}};
%%     %     \node<3> (img3) {\includegraphics[width=.9\linewidth]{figs/per_exc_fig/per_emp.pdf}};
%%     %     \node<4-> (img4) {\includegraphics[width=.9\linewidth]{figs/per_exc_fig/per_full.pdf}};
%%     %   \end{tikzpicture}
%%     % \vfill
%%     % \end{center}
%%     % \end{minipage}%
%%     % \hfill
%%     % \begin{minipage}[t]{0.5\textwidth}
%%     % \begin{itemize}
%%     % \item<1-> Want to model a free excitonic complex
%%     % \item<2-> Supercell calculation (characteristic size $L$)
%%     % \item<3-> Under periodic BCs
%%     % \item<4-> Hence incurr an unphysical image-interaction
%%     % \end{itemize}
%%     % \end{minipage}%
%%     % \vfill
%%     % \begin{itemize}
%%     % \item<5-> Need to remove $E_{\text{int}}$.
%%     % \item<5-> Scaling arguments, or model calculations?
%%     % \end{itemize}
%%     % \end{frame}
%%
%%     \begin{frame}{Physics of FS effects in 2D}
%%     \begin{minipage}[t]{0.5\textwidth}
%%     \vfill
%%     \begin{center}
%%       \begin{figure}[H]
%%         \centering
%%         \includegraphics[width=0.9\linewidth]{figs/per_exc_fig/per_full.pdf}
%%       \end{figure}
%%     \vfill
%%     \end{center}
%%     \end{minipage}%
%%     \hfill
%%     \begin{minipage}[t]{0.5\textwidth}
%%     \vspace{1.4cm}
%%     \begin{itemize}
%%     \item We want to model a \textit{free} excitonic complex
%%     \item Perform supercell calculation (SC characteristic size $L$), subject to periodic BCs
%%     \item Hence incurr an unphysical image-interaction
%%     \end{itemize}
%%     \end{minipage}%
%%     \vfill
%%     \begin{itemize}
%%     \item Need to remove $E_{\text{int}}$. \red{How does it scale?}
%%     \end{itemize}
%%     \end{frame}
%%
%%     \begin{frame}{Physics of FS effects in 2D (cont.)}
%%     \begin{figure}[H]
%%     \centering
%%     \begin{subfigure}{.5\textwidth}
%%       \centering
%%       \includegraphics[width=\linewidth]{figs/coul_origin.png}
%%       % \caption{$r_*=0$ (Coulomb).}
%%       \label{fig:sub1}
%%     \end{subfigure}%
%%     \begin{subfigure}{.5\textwidth}
%%       \centering
%%       \includegraphics[width=\linewidth]{figs/keld_rs_50_origin.png}
%%       % \caption{$r_*=50$ a.u. (Keldysh).}
%%       \label{fig:sub2}
%%     \end{subfigure}
%%     \vspace{-1cm}
%%     \caption{Unscreened (left) and screened (right) field lines from point charges at $\rho=z=0$.}
%%     \label{fig:field_lines}
%%     \end{figure}
%%     \vspace{-.3cm}
%%     \begin{itemize}
%%       \item With 2D screening (Keldysh interaction), charge-quadrupole
%%       interaction\footnote{~~Note no dipole in inversion symmetric system!} leads to
%%       expected scaling which is $\mathcal{O}(L^{-2})$.% \footnote{~~(from taking sum
%%       % to be an integral - really only valid at large distances).}
%%     \end{itemize}
%%     \end{frame}
%%
%%     % \begin{frame}{Physics of FS effects in 2D (cont.)}
%%     %
%%     % \begin{itemize}
%%     %   \item With 2D screening (Keldysh interaction), charge-quadrupole
%%     %   interaction\footnote{~~Note no dipole in inversion symmetric system!} leads to
%%     %   expected scaling which is $\mathcal{O}(L^{-2})$.\footnote{~~(from taking sum
%%     %   to be an integral - really only valid at large distances).}
%%     %
%%     %   \item Model calculation: study periodic excitons directly (lattice sums over
%%     %   screened interaction). \textbf{In progress.}\footnote{~~At last check, the
%%     %   data I had were best fit by a model $\Delta(a)=\Delta(\infty)+C a^{-P}$, with
%%     %   exponent $p$ of 2.0073\ldots}
%%     % \end{itemize}
%%     % \end{frame}
%%
%%     \begin{frame}{Physics of FS effects in 2D (cont.)}
%%     \begin{block}{$\Delta_{\text{QP}}$}
%%       \begin{itemize}
%%         \item Similar image effects, easier to manage.
%%         \item Subtract single-particle $v_{\rm M}$ ($\mathcal{O}(L^{-1})$).
%%         \item From regularized lattice sum over screened interaction ($\sim$ Ewald
%%         sum).
%%       \end{itemize}
%%       \begin{equation}
%%         \sum_{\bf R} W({\bf r}-{\bf R}) \rightarrow \overbrace{\sum_{\bf R} V({\bf
%%         r}-{\bf R})}^{\green{Ewald}}
%%         + \underbrace{\sum_{\bf R} \delta V({\bf r}-{\bf R})}_{\blue{{\it safe}}}.
%%       \end{equation}
%%       \begin{itemize}
%%         \item ``\blue{{\it Safety}}'': $\displaystyle\lim_{r \rightarrow \infty} \delta
%%         V(r)=0$.% (screening has a range).
%%         % is realised if screening is irrelevant
%%         % at long-range in real-space.
%%       \end{itemize}
%%     \end{block}
%%     \end{frame}
%%
%%     \begin{frame}{Physics of FS effects in 2D (cont.)}
%%       \begin{itemize}
%%         \item FSE appear to scale as argued.
%%         \item Big gaps (\red{$\epsilon$!}),\footnote{~~Also, this is not due to FN
%%         error! Backflow has $\mathcal{O}(0.05\ \text{eV})$ effect here.} but good agreement
%%         w/ Gaufr\`{e}s \textit{et
%%         al.}\footnote{~~This result is unpublished, so far, but was presented at GW
%%         2018 by A. Loiseau. $\Delta_{\text{Ex}} = 1.95$ eV.}
%%         \item Phonon renormalisation $\sim 0.17$ eV @ 300K.\footnote{~~Via
%%         Tomeu Monserrat, also as yet unpublished.}
%%       \end{itemize}
%%     \begin{figure}[H]
%%       \floatbox[{\capbeside\thisfloatsetup{capbesideposition={left, center},
%%       capbesidewidth=0.4\textwidth}}]{figure}[\FBwidth]
%%       {\caption{QMC energy gaps in phosphorene vs. system size.}\label{fig:phos}}
%%       {\includegraphics[width=0.5\textwidth]{figs/phos.pdf}}
%%     \end{figure}
%%     \end{frame}
%%
%%     \begin{frame}{Physics of FS effects (cont.)}
%%     Another approach is to consider passivated (finite) clusters. Here FS effect
%%     is kinetic in origin (confinement).\footfullcite{Frank2018}
%%     \begin{itemize}
%%       \item FS converge faster (QP gap $\mathcal{O}(L^{-2})$ by default), {\bf but}\ldots
%%       \item State under study may not be relevant %not necessarily adiabatically connected to relevant
%%       \footfullcite{Drummond2005}\ldots
%%     \end{itemize}
%%     \begin{figure}[H]
%%       \centering
%%       \includegraphics[width=0.6\linewidth]{figs/diamondoid.pdf}
%%       \caption{Band charge densities in C$_{29}$H$_{36}$.}
%%     \label{fig:diamondoid}
%%     \end{figure}
%%     \end{frame}
%%
%%     \begin{frame}{Another QMC study of Phosphorene}
%%     Frank \textit{et al.} have also studied phosphorene. We're
%%     dissatisfied with their approach. Why?
%%     \begin{itemize}
%%       \item Used cluster calculations to argue scaling in bulk calculations.
%%       \item Calculated the \textit{wrong gap}:
%%     \begin{figure}[H]
%%       \floatbox[{\capbeside\thisfloatsetup{capbesideposition={left, center},
%%       capbesidewidth=0.5\textwidth}}]{figure}[\FBwidth]
%%       {\caption{Excerpt from preprint.}\label{fig:frank}}
%%       {\includegraphics[width=0.5\textwidth]{figs/frank.png}}
%%     \end{figure}
%%       \item Our excitonic gap (2.2(2) eV) agrees with their ``quasiparticle'' gap
%%       (2.4 eV) \smiley{}. \footnote{~~Guessed wrong scaling exponent (1/N) for QP gap, but
%%       this isn't a QP gap! Just so happen to have calculated and taken TD limit for
%%       an excitonic gap.  Assuming they've done the calculations correctly, a good
%%       test of our result!}
%%     \end{itemize}
%%     \end{frame}
%%
%%     %%  \begin{frame}{Nodal errors in Phosphorene}
%%     %%  We've tested the inclusion of backflow functions in a small Phosphorene test
%%     %%  system.
%%     %%  \begin{itemize}
%%     %%    \item $\Delta_{\text{QP}} $ lowered by $ 0.03(9)$ eV.
%%     %%    \item $\Delta_{\text{Ex.}} $ lowered by $ 0.04(5)$ eV.
%%     %%  \end{itemize}
%%     %%  \begin{figure}[H]
%%     %%    \floatbox[{\capbeside\thisfloatsetup{capbesideposition={left, center},
%%     %%    capbesidewidth=0.5\textwidth}}]{figure}[\FBwidth]
%%     %%    {\caption*{\red{Q} Why is backflow seemingly irrelevant here, when it
%%     %%    was critical in Si? When, generally, do nodal errors tend to matter?}\label{fig:puzzled}}
%%     %%    {\includegraphics[width=0.3\textwidth]{figs/thinking.png}}
%%     %%  \end{figure}
%%     %%  Backflow QMC has $\mathcal{O}({\rm N}^4_e)$ cost scaling. Further
%%     %%  ``computational exploration'' unlikely to provide answer.
%%     %%  \end{frame}
%%     % }}*
%%
%%     \section{Conclusion}
%%
%%     % *{{ Conclusion
%%     \begin{frame}{Conclusion}
%%     \end{frame}
%%
%%     % }}*
%%
%%     % }}* END OLD CONTENT

% *{{ Thanks!
\begin{frame}{Acknowledgements}

\begin{minipage}[t]{0.45\textwidth}
{\bf Lancaster:}
\begin{itemize}
  \item Neil Drummond
  \item Marcin Szyniszewski
\end{itemize}
\vspace{0.5cm}
{\bf Manchester:}
\begin{itemize}
  \item Vladimir I. Fal'ko
  \item Viktor Z\'{o}lyomi
\end{itemize}
\end{minipage}%
\hfill
\begin{minipage}[t]{0.45\textwidth}
{\bf Japan (JAIST):}
\begin{itemize}
  \item Ryo Maezono
  \item Genki Prayogo
\end{itemize}
\vspace{0.5cm}
{\bf Cambridge:}
\begin{itemize}
  \item Tomeu Monserrat
\end{itemize}
\end{minipage}%
\vfill
\begin{center}
{\Large Thanks for your attention!}
\end{center}
\end{frame}
% }}*

% *{{ References
%\begin{frame}[t,allowframebreaks]{References}

%\printbibliography[]

%\end{frame}
% }}*

\end{document}
% }}*
