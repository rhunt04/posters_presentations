% *{{ preamble
\documentclass[12pt, pdf, hyperref={draft}, usenames, dvipsnames]{beamer}
\usepackage{libertine, textcomp, amssymb, wasysym, color, ulem, verbatim}
\usepackage{floatrow, subcaption}
\usepackage[makeroom]{cancel}
\usepackage{mathtools}
% hide controls
\usenavigationsymbolstemplate{}
\usepackage[T1]{fontenc}
\synctex=1
\setbeamertemplate{caption}[numbered]
% }}*

% *{{ item colouring and themes
\usecolortheme{beaver}
\setbeamertemplate{footline}[frame number]
\definecolor{LancsRed}{HTML}{B5121B}
\setbeamercolor{item}{fg=LancsRed}
\setbeamercolor{title}{fg=LancsRed}
\setbeamercolor{structure}{fg=LancsRed}
\setbeamercolor{frametitle}{fg=LancsRed}
\setbeamercolor{footnote mark}{fg=LancsRed}
\setbeamercolor{footnote}{fg=LancsRed}
\setbeamercolor{bibliography entry author}{fg=LancsRed}
\setbeamercolor{bibliography entry title}{fg=black}
\setbeamercolor{bibliography entry note}{fg=black}
\setbeamercolor{section in toc}{fg=LancsRed}
%\setbeamercolor{subsection in toc}{fg=gray}
\newcommand{\gitem}[1]{\setbeamercolor{item}{fg=ForestGreen}\item[$\checkmark$] #1}
\newcommand{\bitem}[1]{\setbeamercolor{item}{fg=LancsRed}\item[$\times$] #1}
\newcommand{\nitem}[1]{\setbeamercolor{item}{fg=NavyBlue}\item[$-$] #1}
% }}*

% *{{ niceties
\newcommand{\dd}{\mathrm{d}}
\newcommand{\e}{\mathrm{e}}
\newcommand{\ket}[1]{\lvert{#1}\rangle}
\newcommand{\bra}[1]{\langle{#1}\rvert}
\newcommand{\expt}[1]{\langle{#1}\rangle}
\newcommand{\red}[1]{{\bf\color{LancsRed}{#1}}}
\newcommand{\blue}[1]{{\bf\color{NavyBlue}{#1}}}
\newcommand{\green}[1]{{\bf\color{ForestGreen}{#1}}}

% }}*

% *{{ title, subtitle, inst., date
\title{Quantum Monte Carlo Methods in \\
       Solid State Physics}
% \subtitle{}

% authors and institutions
%\author{\emph{Ryan J. Hunt}\inst{1},
%        Neil D. Drummond\inst{1} \\
%        and Vladimir I. Fal'ko\inst{2}}

\author{Ryan Hunt \\ Physically Speaking \\ 20$^{\text{th}}$ June, 2017 \\
\vspace*{0.5cm}\includegraphics[width=2cm]{figs/dice.png}}

%\institute[]{
%
%  \inst{1}
%  Department of Physics,\\
%  Lancaster University
%  \and
%
%  \inst{2}
%  National Graphene Institute,\\
%  University of Manchester}
%
\date{}

% }}*

% *{{ bibliography setup
\usepackage[backend=bibtex]{biblatex}
\bibliography{dmc_bib}
\renewcommand{\footnotesize}{\scriptsize}
\AtEveryCitekey{\clearfield{title}
                \clearfield{pagetotal}
                \clearfield{pages}}

% }}*

% *{{ graphics on front page
\titlegraphic{\vspace*{-4.5cm}\includegraphics[width=3cm]{figs/lan_logo.png}
\hfill\includegraphics[width=3cm]{figs/nownano_logo.png}}

% }}*

% *{{ titlepage + toc

% Section and subsections will appear in the presentation overview
% and table of contents.

\begin{document}

\begin{frame}[plain]
  \titlepage\end{frame}

%\begin{frame}{Outline}
%  \tableofcontents
%\end{frame}

% contents at ssubsection starts - comment to get rid
\AtBeginSection[]
{\begin{frame}<beamer>{Outline}
  \tableofcontents[currentsection]
  \end{frame}}

% }}*

% *{{ the quantum many-body problem
\section{The (Quantum) Many-Body Problem}
\subsection{Why is this relevant for solid state physics?}

\begin{frame}{Solid State Physics}

Solid State Physics is intrinsically many-body in nature. Solids
comprise of {\itshape\/many bodies}.

\begin{figure}[H]
  \floatbox[{\capbeside\thisfloatsetup{capbesideposition={left, center},
  capbesidewidth=0.5\textwidth}}]{figure}[\FBwidth]
  {\caption*{``The underlying physical laws necessary for the
  mathematical theory of a large part of physics and the whole of chemistry are
  thus completely known, and the difficulty is only that the exact application
  of these laws leads to equations much too complicated to be
  soluble.'' \\ \hfill -- Paul Dirac (1929).}\label{fig:dirac}}
  {\includegraphics[width=0.35\textwidth]{figs/dirac_1929.jpg}}
\end{figure}
\end{frame}

\begin{frame}{The Quantum Many-Body Problem}
\begin{block}{Problem Statement}
  \begin{itemize}
    \item Solve:
    \begin{equation}
      \mathcal{\hat H}\ket{\Psi_\lambda} = E_\lambda\ket{\Psi_\lambda},
    \end{equation}
    with
    \begin{align}
      \mathcal{\hat H} = &-\dfrac{\hbar^2}{2m_e}\sum_i \nabla^2_i
      -\sum_{i,I} \dfrac{Z_I e^2}{\lvert {\bf r}_i - {\bf R}_I\rvert}
      +\dfrac{1}{2}
      \sum_{i \neq j}\dfrac{e^2}{\lvert {\bf r}_i -
      {\bf r}_j\rvert} \nonumber \\
      &- \cancelto{\text{0}}{\sum_I \dfrac{\hbar^2}{2M_I}\nabla^2_I} +
      \dfrac{1}{2}
      \sum_{I \neq J} \dfrac{Z_I Z_J e^2}{\lvert {\bf R}_I - {\bf R}_J \rvert}.
    \end{align}
    for $E_\lambda$ and $\ket{\Psi_\lambda}$. Hint: don't really try this.
  \end{itemize}
\end{block}
\end{frame}

\subsection{Why is this a hard problem?}
\begin{frame}{Why is this hard?}
\begin{itemize}
  \item Let ${\bf R} = \{ {\bf r}_1, \ldots, {\bf r}_{N_e} \}$ be a set of
  e$^-$ positions in $3N_e$-dimensions.

  \item \red{Curse of dimensionality:} $\Psi({\bf R})$ doesn't
  factorise, the \# of antisymmetric combinations of basis
  functions\footnote{Also known as \textit{Slater
  determinants}.} needed to express $\Psi$ goes like $\e^{N_e}$.
\end{itemize}

\begin{figure}[H]
  \floatbox[{\capbeside\thisfloatsetup{capbesideposition={right, center},
  capbesidewidth=0.6\textwidth}}]{figure}[\FBwidth]
  {\caption*{``I cannot foresee an advance in computer science which can
  minimize a quantity in a space of $10^{150}$ dimensions.'' \\ \hfill
-- Walter Kohn, Nobel Lecture (1998).}\label{fig:kohn}}
  {\includegraphics[width=0.25\textwidth]{figs/kohn.jpg}}
\end{figure}

\end{frame}

% }}*

% *{{ qmc methods

\section{Quantum Monte Carlo Methods}

\begin{frame}{Quantum Monte Carlo (QMC) Methods}
\begin{itemize}
\item Solid State theory has taken a lot from ``single-particle''
theories.

\item We've learned that these \red{fail} to describe certain important bits
of physics (many-body effects, electronic correlation).\footnote{Major
consequences include: misclassification of metals / semiconductors, systematic
underestimation of energy gaps, systematic failure to describe dispersion
interactions, and the inability to describe many-particle bound states.}

\item Quantum Monte Carlo methods are \blue{strikingly different}.
\end{itemize}
\end{frame}

\subsection{Variational Monte Carlo}

\begin{frame}{Variational MC}
\begin{itemize}
  \item In Variational MC, we evaluate various quantities, e.g.
  \begin{equation}
    E_T = \dfrac{\int\dd{\bf R} \lvert \Psi_T({\bf R})\rvert^2
    \overbracket{\left[{\Psi_T({\bf R})}^{-1} \mathcal{\hat H} \Psi_T({\bf R})
    \right]}^{\text{``Local Energy''}}}
    {\int\dd{\bf R} \lvert \Psi_T({\bf R})\rvert^2}
  \end{equation}
  with stochastic techniques (Metropolis algorithm to sample $\lvert \Psi_T
  \rvert^2$, Monte Carlo integration with subsequent samples).

  \item This allows us to evaluate the energies, variances, \textit{etc} of
  \green{trial wavefunctions}, $\Psi_T$.
\end{itemize}
\end{frame}

\begin{frame}{VMC Trial Wavefunctions}
\begin{itemize}
  \item We obviously don't want to add combinations of determinants.
  Walter Kohn is still correct.

  \item What we can do, however, is build on a single Slater determinant,
  $\mathcal{D}$, and form
  \begin{equation}
    \Psi_{\text{SJ}} = \exp{\left[ \mathcal{J}_{\{\alpha\}}({\bf R})\right]}
    \cdot \mathcal{D}({\bf R}),
  \end{equation}
  with $\mathcal{J}$ the \green{Jastrow exponent} (describes
  correlation, amongst other things).

  \item Aside: one of these ``other things'' is called the \green{Kato cusp
  condition} -- an important property of exact many-body wavefunctions.
\end{itemize}
\end{frame}

\subsection{Diffusion Monte Carlo}

\begin{frame}{Diffusion Monte Carlo}
\begin{itemize}
  \item I spent some time telling you about how single-particle theories fail,
  then I used a Slater determinant in VMC\@. \red{Why is this reasonable?}

  \item VMC is not the whole story. Whilst useful, it is usually done as a
  prelude to DMC\@.

  \item DMC is a \blue{projector-based} method, relying on the general
  fact:\footnote{This is imaginary time evolution, $t = i\tau$.}

  \begin{equation}
    \ket{\Psi_{\text{GS}}} = \lim_{\tau \rightarrow \infty} \exp{\left(-\tau
    \mathcal{\hat H} \right)}\ket{\Psi_T}.
  \end{equation}
\end{itemize}
\end{frame}

\begin{frame}{Diffusion Monte Carlo (II)}

\begin{itemize}
  \item [Q] How do we do enact this projection? How do we \textit{evolve} in
  imaginary time?

  \item [A] An approximate (interacting) Green's function,
  \begin{equation}
  \Psi({\bf R}, \tau+\Delta \tau)
  = \displaystyle\int G({\bf R} \leftarrow {\bf R'},\Delta \tau)
  \Psi({\bf R'}, \tau) \dd{\bf R'}.
  \end{equation}

  \item We \green{don't represent / store $\Psi$}, we store configurations
  \textit{distributed} as $\lvert\Psi\rvert^2$.\footnote{This is how we get
  away with what Dr.\ Kohn rightly said was impossible!}

  \item \red{Caveats:} Relies on interpretation of $\Psi$ as a probability
  density - have to treat +/- regions separately!\footnote{This is the
  \textit{fixed-node approximation} - and is our way of avoiding the so-called
  ``sign-problem'' for fermions.}
  %(but $\Psi$ is \textit{signed}\ldots) - we fix the \textit{nodes} of
  %$\Psi$, and disallow moves in the DMC algorithm which would have crossed a
  %node.
\end{itemize}

\end{frame}

\begin{frame}{Diffusion Monte Carlo (III)}{Credit where credit is due\ldots}

\begin{figure}[H]
  \floatbox[{\capbeside\thisfloatsetup{capbesideposition={left, center},
  capbesidewidth=0.5\textwidth}}]{figure}[\FBwidth]
  {\caption*{``\ldots as suggested by \green{Fermi}, the time-independent
  Schr\"{o}dinger equation \ldots can be interpreted as describing the
  behaviour of a system of particles each of which performs a random walk,
  i.e., diffuses isotropically and at the same time is subject to a
  multiplication\ldots'' \\ \hfill -- {{\bf Nicholas Metropolis} \\
  \hfill \& Stanislaw Ulam (1949)}.}\label{fig:metropolis}}
  {\includegraphics[width=0.45\textwidth]{figs/metropolis.jpg}}
\end{figure}

\end{frame}

% }}*

% *{{ examples
\section{Examples}
\subsection{Charge complexes in semiconductor heterostructures}

\begin{frame}{Trion Formation in Coupled Quantum Wells}{w/ Oliver Witham, and
Neil Drummond}

% \begin{block}{Coupled Quantum Wells}
% Pairs of 2DEGs, separated by a distance, $d$, and electrically biased such
% that $e^-$ are bound in a layer, $h^+$ in another.
% \end{block}

\begin{minipage}[t]{0.55\textwidth}

\vspace{1cm}
\begin{itemize}
  \item Experimentalists are interested in shining lasers on ``real'' CQWs, and
  looking for Bose-Einstein condensates of excitons.\footnotemark%
  \vspace{0.5cm}

  \item They've yet to find anything conclusive: Why?
\end{itemize}
\vspace{1cm}

\end{minipage}%
\hfill
\begin{minipage}[t]{0.4\textwidth}

\begin{figure}[H]
  \centering
  \includegraphics[width=\linewidth]{figs/cqw_biexc.pdf}
  \caption{A schematic of the system in question, and the problem setup for the
  ground-state biexciton.}
\label{fig:biex}
\end{figure}

\end{minipage}%
 \footnotetext{A particularly good group who have studied this extensively is
 that of Ronen Rapaport. They shy away from calling what they have found a BEC,
 instead naming it a ``correlated quantum liquid''.}
\end{frame}

\begin{frame}{What have we done?}
  \begin{itemize}
    \item Previously, the biexciton was thought to hinder the formation of a
    condensate. It is stable for very large $d$ values.

    \item [Q] Is it \blue{safe} to assume that the charged trions are less
    stable than
    the neutral biexcitons?

    \item [A] \red{Nope.}
  \end{itemize}
\end{frame}

\begin{frame}{What have we done? (II)}
  \begin{itemize}
    \item By solving the three-body Schr\"{o}dinger equation for this
    system \textit{exactly} in DMC, we have found,

    \begin{equation}
      d_{\text{crit}}(X^{\pm}) \sim 10 \cdot d_{\text{crit}}(XX)
    \end{equation}
  \end{itemize}

\begin{figure}[H]
  \floatbox[{\capbeside\thisfloatsetup{capbesideposition={left, center},
  capbesidewidth=0.32\textwidth}}]{figure}[\FBwidth]
  {\caption{Trion binding energies against $d$ for two experimentally
  relevant $e^-$/$h^+$ mass ratios, $\sigma$.}\label{fig:trion}}
  {\includegraphics[width=0.7\textwidth]{figs/trion_sigma2_4.pdf}}
\end{figure}

\end{frame}


\subsection{{\itshape\/Grown-up QMC}: 500+ quantum particles}

\begin{frame}{{\itshape\/ Grown-up QMC}: 500+ quantum particles}{w/ Neil
Drummond, Marcin Szyniszewski, and Ryo Maezono}
\begin{itemize}
  \item Modern Solid State Physics is now an arena in which we routinely
  simulate the properties of \red{large}, \green{realistic} systems.

  \item Traditionally, excited state properties have been very interesting:
  excitonic effects, band gaps, \ldots

  \item QMC is the most accurate means we have of studying these properties
  for remotely realistic systems: competitor methods scale as high powers of
    system size,\footnote{Often $\mathcal{O}(N_e^{6-7})$, vs. QMC - which is
    usually $\mathcal{O}(N_e^3)$.} or get the \red{wrong answer}.\footnote{For
    reasons that we now largely understand: correlation effects are critical!}
\end{itemize}
\end{frame}

\begin{frame}{What have we done?}

  \begin{itemize}
    \item By using DMC for large supercells of solid systems, we have
    calculated:\footnote{or, are still calculating!}
    \begin{itemize}
      \item Quasiparticle and Excitonic energy gaps of \green{hexagonal BN},
      cubic BN, diamond Si, SiO$_2$ ($\alpha$-quartz), phosphorene.

      \item Charge-complex binding energies (excitons, and biexcitons) -
      \textit{without} any over-simplifying approximations.
    \end{itemize}
    \item We have found that such large-scale excited state calculation are
    possible, however, \textit{they can bite}.
  \end{itemize}
\end{frame}

\begin{frame}{Finite Size Effects}
  \begin{itemize}
    \item In large-scale QMC simulations, we are limited by $N_e$. We therefore
    study systems of $\mathcal{O}(10)$ unit cells.
  \end{itemize}

\begin{figure}[H]
  \centering
  \includegraphics[width=0.68\linewidth]{figs/bulk_qpgaps.pdf}
  \caption{Quasiparticle energy gaps in bulk hBN vs.\ system size.}
\label{fig:bhbn_gaps}
\end{figure}

\end{frame}

\begin{frame}{Is all hope lost?}
  \begin{itemize}
    \item Probably not - hBN is, realistically, a \red{worst-case} scenario.
    Why?
    \begin{itemize}
      \item Localisation vs.\ delocalisation of quasiparticles.
      \item Anisotropic dielectric screening.
      \item Flat bands, with \red{unusual features}.
    \end{itemize}
  \end{itemize}

\begin{figure}[H]
  \centering
  \includegraphics[trim={2cm 6cm 2cm 6cm},clip,width=0.9\linewidth]
  {figs/bhbn_gamma_cbm.png}
  \caption{Band charge density isosurface of the (delocalised) $\Gamma_c$
  band in bulk hBN\@. This quasi-electron is almost \textit{free}.}
\label{fig:bhbn_gamma}
\end{figure}

\end{frame}

% }}*

% *{{ thanks!
\begin{frame}[plain]
\begin{center}
  {\Huge Thank you all for listening} \\
  \vspace{2cm}
  {\small Incidentally, the QMC community is quite small. You all now know
  more this topic than most condensed matter physicists!}
\end{center}
\end{frame}
% }}*

\end{document}

