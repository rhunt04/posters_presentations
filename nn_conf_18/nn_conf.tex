% *{{ Actual preamble
% *{{ preamble
\documentclass[12pt, pdf, hyperref={draft}, usenames, dvipsnames]{beamer}
\usepackage{libertine, textcomp, amssymb, wasysym, color, ulem, verbatim}
\usepackage{floatrow, subcaption}
% hide controls
\usenavigationsymbolstemplate{}
\usepackage[T1]{fontenc}
\synctex=1
\setbeamertemplate{caption}[numbered]
% }}*

% *{{ item colouring and themes
\usecolortheme{beaver}
\setbeamertemplate{footline}[frame number]
\definecolor{LancsRed}{HTML}{B5121B}
\setbeamercolor{item}{fg=LancsRed}
\setbeamercolor{title}{fg=LancsRed}
\setbeamercolor{structure}{fg=LancsRed}
\setbeamercolor{frametitle}{fg=LancsRed}
\setbeamercolor{footnote mark}{fg=LancsRed}
\setbeamercolor{footnote}{fg=LancsRed}
\setbeamercolor{bibliography entry author}{fg=LancsRed}
\setbeamercolor{bibliography entry title}{fg=black}
\setbeamercolor{bibliography entry note}{fg=black}
\setbeamercolor{section in toc}{fg=LancsRed}
%\setbeamercolor{subsection in toc}{fg=gray}
\newcommand{\gitem}[1]{\setbeamercolor{item}{fg=ForestGreen}\item[$\checkmark$] #1}
\newcommand{\bitem}[1]{\setbeamercolor{item}{fg=LancsRed}\item[$\times$] #1}
\newcommand{\nitem}[1]{\setbeamercolor{item}{fg=NavyBlue}\item[$-$] #1}
% }}*

% *{{ niceties
\newcommand{\dd}{\mathrm{d}}
\newcommand{\e}{\mathrm{e}}
\newcommand{\ket}[1]{\lvert{#1}\rangle}
\newcommand{\bra}[1]{\langle{#1}\rvert}
\newcommand{\expt}[1]{\langle{#1}\rangle}
\newcommand{\red}[1]{{\bf\color{LancsRed}{#1}}}
\newcommand{\blue}[1]{{\bf\color{NavyBlue}{#1}}}
\newcommand{\green}[1]{{\bf\color{ForestGreen}{#1}}}
\newcommand{\mred}[1]{{\color{LancsRed}{#1}}}
\newcommand{\mblue}[1]{{\color{NavyBlue}{#1}}}
\newcommand{\mgreen}[1]{{\color{ForestGreen}{#1}}}

% }}*

% *{{ title, subtitle, inst., date
\title{\large{Modelling charge complexes in 2D
semiconductors, and their heterostructures}}
\subtitle{CDT Summer Conference 2018}

% authors and institutions
\author{\textbf{Ryan J. Hunt}\inst{1},
        Neil D. Drummond\inst{1} \\
        and Vladimir I. Fal'ko\inst{2}}

%\author{Ryan Hunt}

\institute[]{\
  \inst{1}
  Department of Physics,\\
  Lancaster University
  \and
  \inst{2}
  National Graphene Institute,\\
  University of Manchester
}

% date
\date{2${}^{\text{nd}}$ July, 2018}

% }}*

% *{{ bibliography setup
\usepackage[backend=bibtex, style=phys]{biblatex}
\bibliography{refs}
\renewcommand{\footnotesize}{\scriptsize}
\AtEveryCitekey{\clearfield{title}
                \clearfield{pagetotal}
                \clearfield{pages}
                \clearfield{doi}}
\renewcommand*{\bibfont}{\footnotesize}

% }}*

% *{{ graphics on front page
\titlegraphic{\vspace*{-3.1cm}\includegraphics[width=3cm]{figs/lan_logo.png}
\hfill\includegraphics[width=3cm]{figs/nownano_logo.png}}

% }}*

% *{{ titlepage + toc

% Section and subsections will appear in the presentation overview
% and table of contents.

\begin{document}

\begin{frame}[plain]
  \titlepage\end{frame}

%\begin{frame}{Outline}
%  \tableofcontents
%\end{frame}

% contents at ssubsection starts - comment to get rid
\AtBeginSection[]
{\begin{frame}<beamer>{Outline}
  \tableofcontents[currentsection]
  \end{frame}}

% }}*

% }}* End Preamble

% *{{ ``Modelling''
\begin{frame}{``Modelling''}

For me, this means using \textit{quantum Monte Carlo} (QMC) methods.
% An alternative might be density functional theory.
QMC relies on random sampling techniques, which can be used (with
\textsc{casino}\footfullcite{Needs2009}) to solve the many-body
Schr\"{o}dinger equation:

\begin{equation}
  \left[-\dfrac{1}{2}\nabla^2_{\bf R} + \mred{V({\bf R})}\right]\Psi({\bf R}) =
  E\Psi({\bf R}),\ {\bf R}=\{{\bf r}_1,\ldots,{\bf r}_{\rm N}\} \label{eq:sch}
\end{equation}

{\bf Main point:} This is a \red{hard problem}. ${\rm N}$ might be
6,\footnote{~~ $\sim$ charge complex in the effective mass approximation} but
it might be 60, or even 600\footnote{~~ $\sim$ charge complex in ``heroic''
QMC}\ldots

\begin{itemize}
  \item QMC deals with it, without \textit{excessively muddying} $V({\bf R})$.
\end{itemize}

\end{frame}
% }}*

% *{{ Quintons in TMDCs
\begin{frame}{Quintons in monolayer TMDs\footfullcite{Mostaani2017}}{A.k.a. ``charged biexciton'', but
that's no fun.}

\begin{figure}[H]
  \floatbox[{\capbeside\thisfloatsetup{capbesideposition={left, center},
  capbesidewidth=0.7\textwidth}}]{figure}[\FBwidth]
  {\caption{A \textit{quinton} (XX$^{-}$, pictured) is a
  \textit{hypothetical} bound state of five particles in a
  semiconductor.}\label{fig:quinton}}
  {\includegraphics[width=0.3\textwidth]{figs/quinton.png}}
\end{figure}
% In the 2D semiconductors, (quasi-)electrons and holes can be thought of as free
% particles with an effective mass. Their interaction is {\bf screened} by the
% presence of the core electronic states, and the ions.

We predicted that quintons would be stable in monolayers of
TMDCs. Recent experiments corroborate this
claim.\footfullcite{Hao2017}$^{,}$\footnote{~~ 1806.03775, 1805.04950,
1802.10247}

{\bf Disclaimer:} Our model neglects valley effects, these would act to
alter interactions at short-range.

\end{frame}
% }}*

% *{{ Interlayer complexes in heterobilayers
\begin{frame}{Interlayer complexes in heterobilayers of
TMDs\footfullcite{hbl_paper}}

\begin{minipage}[t]{0.24\textwidth}
\begin{figure}[H]
  \centering
  \includegraphics[width=\linewidth]{figs/hbl_teaser.pdf}
\end{figure}
\end{minipage}%
\hfill
\begin{minipage}[t]{0.74\textwidth}
\begin{itemize}
  \item Optical properties of type-II TMD heterobilayers are dominated by
  interlayer complexes bound to impurity centers.
\end{itemize}

\begin{figure}[H]
  \centering
  \includegraphics[width=\linewidth]{figs/doh_diag.png}
  \caption{(one) Diagram (of four) representing D$^{0}$ recombination with
  h$^{+}$.}
\end{figure}

\end{minipage}%

\end{frame}
% }}*

% *{{ ``Heroic'' QMC
\begin{frame}{``Heroic'' electronic structure
calculations\footfullcite{gaps_paper}}
\begin{minipage}[t]{0.75\textwidth}

\begin{itemize}

  \item[Q] Can we access the optoelectronic properties of a material by knowing
  only the atomic numbers of its constituents?

\vspace{.5cm}

  \item[A] {\bf Yeah}, but beware:

  \begin{itemize}
    \item It's not cheap (CPU time).
    \item Technical difficulties: finite size effects, variational collapse.
    \item Sometimes, the electronic problem is not the whole story\ldots
  \end{itemize}

\end{itemize}


\end{minipage}%
\hfill
\begin{minipage}[t]{0.25\textwidth}

\begin{figure}[H]
  \centering
  \includegraphics[width=0.8\linewidth]{figs/heman.png}
\end{figure}
\end{minipage}%
\begin{itemize}
  \item Paper is mostly technical, we do consider excitons in phosphorene.
\end{itemize}
\end{frame}
% }}*

% *{{ Next steps: intro to the Mahan exciton
\begin{frame}{The Future}

My next 2D materials project will concern charge complexes at finite
density.

\begin{itemize}
  \item Superfluidity of excitons in vdW crystals? {\tiny (needs thought)}
  \item The \textit{\green{Mahan exciton}} (ME) in 2D ``metals''?\footnote{~~
  2D semiconductors, at high density.} {\tiny (thinking done)}
\end{itemize}

% The Mahan exciton is, effectively, the state described by a hole immersed
% in an electron gas (at sufficiently high density).\footnote{Beyond what you
% might typically achieve in a PL experiment.}

\begin{figure}[H]
  \centering
  \includegraphics[width=\linewidth]{figs/mahan_ipe.png}
\end{figure}

% i.e.\ \textit{Ewaldise} the Keldysh interaction and use it in model
% calculations.

\end{frame}
% }}*

% *{{ The Mahan exciton
\begin{frame}{The Mahan Exciton}{``Excitons in Metals''}

Formation coincides with onset of absorption in metals,\footnote{~~ Most
clearly at low temperature\ldots} in a feature known as the ``Fermi Edge
Singularity'' (FES).\footfullcite{Mahan_book,Mahan_PRL} \red{Anderson} and
\blue{Mahan} studied this, FES boils down to spectral function:

\begin{equation}
  A(\omega) =
  A^{(0)}(\omega)\underbrace{\mred{\Theta(\omega-\omega_T)}}_{\text{Orth.
  Catastrophe}}
  {\overbrace{\mblue{\left(\dfrac{\xi_{0}}{\omega-
  \omega_{T}}\right)}^{\mgreen{\alpha}}}^{\text{e-h int.}}}.
\end{equation}

(In practice, you'd expect to see an increasingly asymmetric peak appear
in PL measurements at lower and lower temperatures.)

%$\alpha$ is affected by both physical effects.

% Explain line shape, experimental signatures, FS ``shake-up'' and Anderson's
% orthogonality catastrophe?

% Sensitivity of the orthogonality issue to interaction strength (and form).

\end{frame}

\begin{frame}{ME II}

\begin{itemize}
  \item ME a distinctly many-body effect -- ``exciton'' is a \textbf{misnomer}.

  \item To properly describe ME in doped 2D SCs, would need to account for all
  electrons, ions.\ldots Such ``heroic''-scale realistic computations are
  hard.\footnote{~~ At least for mere mortals with finite computer power.}

  \item Can we do \textit{something}\ldots? \green{I think so}.

  \item ME has been studied in (Coulomb) 2DEG,~\footfullcite{Spink_ME} but key
  ingredient in (\textit{bona fide}) 2D is screening. Options:

\begin{enumerate}

  \item ``Heroic'' calculations $\rightarrow$ screening \blue{emergent} from
  $V({\bf R})$.

  \item Devise a clever model $\rightarrow$ screening \blue{modelled}. (Other
  clever models exist for the
  crossover\ldots\footfullcite{Efimkin2017,Efimkin2018})

\end{enumerate}
\end{itemize}

\end{frame}
% }}*

% *{{ thanks and end document
\begin{frame}[plain]
\begin{center}
  {\Large Thank you all for listening, and thanks to my
collaborators:}
\end{center}

\vfill

\begin{minipage}[t]{0.45\textwidth}

{\bf Manchester (\textit{NGI}):}
\begin{itemize}
  \item M. Danovich
  \item D. A. Riuz-Tijerina
  \item V. I. Fal'ko
\end{itemize}

\vspace{0.5cm}

{\bf Cambridge (\textit{CGC}):}
\begin{itemize}
  \item E. Mostaani
\end{itemize}

\end{minipage}%
\hfill
\begin{minipage}[t]{0.45\textwidth}

{\bf Japan (\textit{JAIST}):}
\begin{itemize}
  \item R. Maezono
  \item G. Prayogo
\end{itemize}

\vspace{0.5cm}

{\bf Lancaster:}
\begin{itemize}
  \item D. M. Thomas
  \item M. Szyniszewski
  \item N. D. Drummond
  \item \textcolor{gray}{O. Witham (Frankfurt)}
\end{itemize}

\end{minipage}%
\end{frame}

\begin{frame}[t,allowframebreaks]{References}

\printbibliography[]

\end{frame}

\end{document}
% }}*
