\documentclass[hyperref={draft}]{beamer}
\usepackage[size = a0, orientation = portrait, scale = 1.8]{beamerposter}

% use my themes/colours
\usetheme{Poster}
\usecolortheme{Ryan_lancaster}

% zigzagging line in tikz
\usetikzlibrary{decorations.pathmorphing, patterns,shapes}

% number figures as per usual
\setbeamertemplate{caption}[numbered]

% fonts and scaling
\usepackage[utf8]{inputenc}
\usepackage[T1]{fontenc}
\usepackage{tikz, libertine, floatrow, caption, graphicx}
\usepackage[scaled = 0.82]{inconsolata}
\usepackage[libertine]{newtxmath}

% command for emphasis + bra/kets
\newcommand{\red}[1]{{\bf\color{red}{#1}}}
\newcommand{\ket}[1]{\lvert {#1} \rangle}
\newcommand{\bra}[1]{\langle {#1} \rvert}
\newcommand{\expt}[1]{\langle {#1} \rangle}

\title{\textit{Ab Initio} Modelling of Two- \\
	   Dimensional Semiconductors \\
	   \Large{\underline{\smash{Ryan Hunt}}$^1$, Neil Drummond$^1$ and Vladimir Fal'ko$^2$} 
	   \\ \vspace{0.2cm} \normalsize{$^1$Department of Physics, Lancaster University\\ $^2$National Graphene Institute, University of Manchester}}
\author[\large{$\langle$ Alternate : \texttt{ryan.hunt@postgrad.manchester.ac.uk} $\rangle$}]{\Large{Presenter e-mail : \framebox{\strut\ \texttt{r.hunt4@lancaster.ac.uk}\ }}}
%\institute{\quad}

% Footer logos (as per requirement...)
\footimage{\centering\includegraphics[height=6.5cm]{figs/all_logos.png} \hspace{6cm} }
\begin{document}

% Images right and left of title
\addtobeamertemplate{headline}{} 
{ \begin{tikzpicture}[remember picture,overlay] 
\node [shift={(-9 cm, -6.5 cm)}] at (current page.north east) {\includegraphics[width=15cm]{figs/nownano-logo.png}};
\node [shift={( 9 cm, -6.5 cm)}] at (current page.north west) {\includegraphics[width=15cm]{figs/lan_logo.png}};
\end{tikzpicture} }

\begin{frame}[fragile]\centering

\begin{columns}[T]
\hfill
% Main Left Column
\begin{column}{.46\textwidth}

\begin{block}{Introduction}
\begin{itemize}
\item Well-known, popular modelling approaches like Density Functional Theory (DFT) are often \red{inadequate} for the study of condensed matter systems.
\item Quantum Monte Carlo (QMC) techniques are a family of alternative methods for the study of a wide variety of condensed matter systems [$1$], offering \red{systematic improvement} over other techniques.
\end{itemize}
\end{block}

\begin{block}{Variational MC}
\begin{itemize}
\item Simplest of all QMC techniques, reliant on variational principle for the \red{trial many-e$^-$ wavefunction} $\ket{\Psi_{\alpha}}$:
\begin{equation}
	E(\alpha) = \dfrac{\bra{\Psi_{\alpha}} \mathcal{\hat H} \ket{\Psi_{\alpha}}}{\bra{\Psi_{\alpha}}\Psi_{\alpha} \rangle}
	\geq E_0
\end{equation}
  \begin{description}
  \item[$\alpha$] $\rightarrow$ variational parameters, chosen \red{specially}
  \item[$\mathcal{\hat H}$] $\rightarrow$ many-e$^-$ Hamiltonian

  \end{description}
\item VMC used as prelude to Diffusion MC (DMC) studies. Cheaper to add \red{variational freedom} and exploit it \textit{before} using more computationally expensive DMC.
\end{itemize}
\end{block}

\begin{block}{DMC - Biexcitons in Coupled QWs}
\begin{itemize}
\item Two 2DEGs, separated by $d$. Already realised in III-V bilayers, \textit{similar}$^{\dagger}$ physics seen in \red{truly} 2D heterostructures.
\end{itemize}
\begin{figure}[H]
%	\centering
	\includegraphics[width=0.65\textwidth, trim={0 0 0 0.9cm}, clip]{figs/cqw_biexc.pdf}
	\caption{Schematic of biexciton formation in a CQW system. Inset: form of the biexciton trial wavefunction.}
	\label{biexc_dmc}
\end{figure}

\vspace{1cm}
\begin{itemize}
\item DMC is a \red{projection} method, $\ket{\Psi}$ evolved in \emph{imaginary} time ($\tau=it$)
\begin{equation}
	\ket{\Psi(\tau)} = \sum^{\infty}_{i=0}c_i\ \mathrm{e}^{-(\epsilon_i - E_T)\tau}\ket{\Phi_i(0)}
\end{equation}
\item $e^-h^+$ \& $e^-h^+e^-h^+$ $\rightarrow$ \red{nodeless} model systems
\end{itemize}
\vspace{1cm}
$^{\dagger}$ \red{Problem}: screening (like [$2$]) in realistic systems...

\end{block}

\end{column}
\begin{column}{.04\textwidth}

\begin{tikzpicture}
\node[draw] at (0,   0) (A) {\scriptsize{$E_{DFT}$}};
\node[draw] at (0, -49) (B) {\scriptsize{$E_{VMC}$}};
\node[draw] at (0, -77) (C) {\scriptsize{$E_{DMC}$}};
\node[draw] at (0, -81) (D) {\scriptsize{$E_{0}$}};
\draw[black] (A.south) -- (A.south|-B.north);
\draw[black] (B.south) -- (B.south|-C.north);
%\draw[decoration = {zigzag,segment length = 6mm, amplitude = 3mm},decorate] (C.south) -- (D.south|-D.north);
\draw[decoration = {snake, segment length = 6mm, amplitude = 3mm}, decorate] (C.south) -- (D.south|-D.north);
\end{tikzpicture}

\end{column}
\hfill
% Main Right Column


\begin{column}{.46\textwidth}

%\begin{block}{Example CQW Results}
%\begin{itemize}
%\item For given model parameters, it is possible to \red{highly} optimise $\ket{\Psi}$ in VMC, and run DMC simulations giving results like...
%\begin{figure}[H]
%	\floatbox[{\capbeside\thisfloatsetup{capbesideposition={left, center},capbesidewidth=0.4\textwidth}}]{figure}[\FBwidth]
%	{\caption{Results taken during a DMC simulation for a biexciton with $\sigma=1$ and $d = 0.2$ au. Notice the (short!) equilibration period.}\label{fig:dmc_side}}
%	{\includegraphics[width=0.6\textwidth]{figs/0_01au_data.pdf}}
%\end{figure}
%\item Highly accurate \red{binding energies}, pair-interaction potentials, etc... [$3$]
%\end{itemize}
%\end{block}

\begin{block}{DMC - hBN energy gaps}
\begin{itemize}
\item QMC methods aren't just for toy problems
\begin{figure}[H]

	\centering
	\includegraphics[width=0.75\textwidth]{figs/bhbn2.pdf}
	\caption{An example (3 3 2) supercell used in our simulations. Blue atoms are B, Green are N. This sim. cell contains 144 e$^-$ (PPs!).}
	\label{bhbn_332}

\end{figure}

\item \red{Goal:} Use QMC to calculate, from first-principles, the {\it\underline{excitonic}} and {\it quasiparticle} gaps of bulk hBN,

\begin{equation}
	\Delta^P_{Ex} = E^P_1(N) - E^P_0(N)
\end{equation}

\begin{figure}[H]
	\floatbox[{\capbeside\thisfloatsetup{capbesideposition={left, center},capbesidewidth=0.3\textwidth}}]{figure}[\FBwidth]
	{\caption{Example {\it monolayer} hBN DFT (PBE) bandstructure, labelled for explanation.}\label{fig:mhbn_pbe_labelled}}
	{\includegraphics[width=0.7\textwidth]{figs/mhbn_pbe_labelled.pdf}}
\end{figure}
\item \red{Why?} $P \in \{ {K,M, \cdots} \} \implies$ can {\it accurately} determine gaps between arbitrary points in $\vec k$-space..

\item \red{Progress?} Ask me for a real-time update!
\begin{figure}[H]

	\centering
	\includegraphics[width=\textwidth]{figs/331_energies_first.pdf}
	\caption{My first results, for g.s.e of the (3 3 1) supercell.}
	\label{331_dmc_fist}

\end{figure}
\end{itemize}

\end{block}

\begin{block}{Acknowledgements}
\begin{itemize}
\item QMC calculations performed with the \texttt{CASINO} code [$4$].
\item All calculations were performed on the \red{HEC} facility at Lancaster and the \red{N8 HPC}.
\end{itemize}
\end{block}
\end{column}
\end{columns}

\begin{center}
\begin{minipage}{0.993\textwidth}
\begin{block}{References}
\begin{columns}[T]

% Left Column
\begin{column}{.38\textwidth}
\begin{enumerate}
\item Foulkes, W.M.C. et al., \textit{RMP}, {\bf 73}, 33-83, (2001).
\item Keldysh, L. V., \textit{JETP}, {\bf 29}, 658, (1979).
\end{enumerate}
\end{column}

% Right Column
\begin{column}{.55\textwidth}
\begin{enumerate}
\setcounter{enumi}{2}
\item Lee, R. M. et al., \textit{PRB}, {\bf 79}, 125308, (2009).
\item Needs, R. J. et al., \text{J. Phys. Cond. Mat.}, {\bf 22}, 0232011, (2009).
\end{enumerate}
\end{column}

\end{columns}

\end{block}
\end{minipage}
\end{center}
\end{frame}
\end{document}
