% *{{ preamble, non-content

 % *{{ preamble
\documentclass[12pt, pdf, hyperref={draft}, usenames, dvipsnames]{beamer}
\usepackage{libertine, textcomp, amssymb, wasysym, color, ulem, verbatim}
\usepackage{floatrow, subcaption}
% hide controls
\usenavigationsymbolstemplate{}
\usepackage[T1]{fontenc}
\synctex=1
\setbeamertemplate{caption}[numbered]
% }}*

% *{{ item colouring and themes
\usecolortheme{beaver}
\setbeamertemplate{footline}[frame number]
\definecolor{LancsRed}{HTML}{B5121B}
\setbeamercolor{item}{fg=LancsRed}
\setbeamercolor{title}{fg=LancsRed}
\setbeamercolor{structure}{fg=LancsRed}
\setbeamercolor{frametitle}{fg=LancsRed}
\setbeamercolor{footnote mark}{fg=LancsRed}
\setbeamercolor{footnote}{fg=LancsRed}
\setbeamercolor{bibliography entry author}{fg=LancsRed}
\setbeamercolor{bibliography entry title}{fg=black}
\setbeamercolor{bibliography entry note}{fg=black}
\setbeamercolor{section in toc}{fg=LancsRed}
%\setbeamercolor{subsection in toc}{fg=gray}
\newcommand{\gitem}[1]{\setbeamercolor{item}{fg=ForestGreen}\item[$\checkmark$] #1}
\newcommand{\bitem}[1]{\setbeamercolor{item}{fg=LancsRed}\item[$\times$] #1}
\newcommand{\nitem}[1]{\setbeamercolor{item}{fg=NavyBlue}\item[$-$] #1}
% }}*

% *{{ niceties
\newcommand{\dd}{\mathrm{d}}
\newcommand{\e}{\mathrm{e}}
\newcommand{\ket}[1]{\lvert {#1} \rangle}
\newcommand{\bra}[1]{\langle {#1} \rvert}
\newcommand{\expt}[1]{\langle {#1} \rangle}
\newcommand{\red}[1]{{\bf\color{LancsRed}{#1}}}
\newcommand{\blue}[1]{{\bf\color{NavyBlue}{#1}}}
\newcommand{\green}[1]{{\bf\color{ForestGreen}{#1}}}

% }}*

% *{{ title, subtitle, inst., date
\title{First-Principles Modelling \\
       of 2D Semiconductors}
\subtitle{NOWNANO Day Visit}

% authors and institutions
%\author{\emph{Ryan J. Hunt}\inst{1},
%        Neil D. Drummond\inst{1} \\
%        and Vladimir I. Fal'ko\inst{2}}

\author{Ryan Hunt}

%\institute[]{
%
%  \inst{1}
%  Department of Physics,\\
%  Lancaster University
%  \and
%
%  \inst{2}
%  National Graphene Institute,\\
%  University of Manchester}
%
% date
\date{Friday 11$^{\text{th}}$ November}

% }}*

% *{{ bibliography setup
\usepackage[backend=bibtex]{biblatex}
\bibliography{dmc_bib}
\renewcommand{\footnotesize}{\scriptsize}
\AtEveryCitekey{\clearfield{title}
                \clearfield{pagetotal}
                \clearfield{pages}}

% }}*

% *{{ graphics on front page
\titlegraphic{\vspace*{-2.5cm}\includegraphics[width=3cm]{figs/lan_logo.png}
\hfill\includegraphics[width=3cm]{figs/nownano_logo.png}}

% }}*

% *{{ titlepage + toc

% Section and subsections will appear in the presentation overview
% and table of contents.

\begin{document}

\begin{frame}[plain]
  \titlepage
\end{frame}

%\begin{frame}{Outline}
%  \tableofcontents
%\end{frame}

% contents at ssubsection starts - comment to get rid
%\AtBeginSection[]
%{ \begin{frame}<beamer>{Outline}
%    \tableofcontents[currentsection]
%  \end{frame} }

% }}*

% }}*

% *{{ What do I do? - Properties of Semiconductors
\section{What do I do?}
\begin{frame}{What do I do?}
  \begin{itemize}
    \item Broadly speaking, I use \green{advanced} \blue{electronic structure
    methods} to gain insight into the electronic\footnote{or optoelectronic, or
    structural, or all.} properties of materials.
  \end{itemize}
  \begin{figure}[H]
    \centering
    \includegraphics[width=0.8\linewidth]{figs/bhbn.png}
    \caption{Hexagonal Boron Nitride. If we didn't know about this material,
    how would we go about \blue{determining} its properties?}
    \label{fig:bhbn}
  \end{figure}
\end{frame}
% }}*

% *{{ Why do I do it? - Basic research / technological importance
%                     - Semiconductors are challenging for first-principles...
%                     - 2D is challenging for first-principles (vdw, screening)
\section{Why?}
\begin{frame}{Why do I do it?}
\begin{minipage}{0.6\textwidth}
  \begin{itemize}
    \item We \textit{know} existing alternative approaches are \red{deficient}.
    % DFT -> TM Oxides, qualitative failures.
    % GW  -> would be good if didn't have obv DFT dependance.
    % figure here - bandgap problem?
    \vspace{0.8cm}
    \item We \red{don't} always have the experimental capability to study a
    certain material in a certain environment.
    %\item Hopefully I don't have to tell you semiconductors are key to:
    %  \begin{itemize}
    %    \item Transistor technologies
    %    \item Light harvesting technologies
    %    \item Light emission technologies
    %    \item Funding PhD students
    %  \end{itemize}
  \end{itemize}
\end{minipage}%
\hfill
\begin{minipage}{0.4\textwidth}
\centering
  \begin{figure}[H]
    \centering
    \includegraphics[width=0.9\linewidth]{figs/dft_gaps.pdf}
    \caption{Bandgap problem of DFT \blue{(and $GW$..?)}\footnotemark.}
    \label{fig:dft_gaps}
  \end{figure}
\end{minipage}
\vspace{0.1cm}
  \begin{itemize}
    \item I believe that we should be capable of answering:  "Why does material
    X have property Y". There is merit in \green{knowing}.
  \end{itemize}
  \footnotetext{van Schilfgaarde, {\it et al.}, PRL {\bf 96}, 2006.}
\end{frame}

% }}*

% *{{ What are our methods? QMC!
\section{How do I do what I do?}
\begin{frame}{How do I do it?}
  \begin{itemize}

    \item \green{Quantum Monte Carlo} : we use random sampling
    techniques to approximately solve the \red{full} many-electron
    Schr\"{o}dinger equation.

    \item \blue{Density Functional Theory} : despite its faults, DFT is still
    an immensely successful theory. If you know \textit{nothing} about a
    system, DFT is worth using.
  \end{itemize}
    \begin{figure}[H]
      \centering
      \includegraphics[width=0.8\linewidth]{figs/hpcs.jpg}
      \caption{At any one time, I am running calculations on around 700+ HPC
      cores, usually $\sim$ 350 here on \green{HEC} and more on the
      \green{N8 HPC}.}
      \label{fig:hpcs}
    \end{figure}
\end{frame}

% }}*

% *{{ What have I studied? binding cc's, gaps in solids
\section{What kinds of things have I studied?}
\begin{frame}{Examples of my research}
  \begin{itemize}
    \item Lots of time recently on calculating \blue{band gaps} for solids.
  \end{itemize}
\begin{minipage}{0.60\textwidth}
  \begin{figure}[H]
    \centering
    \includegraphics[width=\linewidth]{figs/si_bs.pdf}
    \caption{Schematic bandstructure of Si.}
    \label{fig:quartz}
  \end{figure}
\end{minipage}%
\hfill
\begin{minipage}{0.4\textwidth}
  \begin{itemize}
    \item Some results :
  \vspace{0.2cm}
  \end{itemize}
\centering
\begin{tabular}{ccc}  \hline
Gap & DMC & Expt.\footnotemark \\ \hline
$\Delta_{\Gamma \rightarrow \Gamma}$ & 3.57(4) eV & 3.4 eV \\
$\Delta_{\Gamma \rightarrow X}$      & 1.24(4) eV & 1.2 eV \\
$\Delta_{\Gamma \rightarrow L}$      & 2.39(4) eV & 2.0 eV
\end{tabular}

\end{minipage}
\vspace{0.1cm}
  \begin{itemize}
    \item Also studied hexagonal and cubic BN, bulk $\alpha$-quartz, several
    small molecules and dimers.
  \end{itemize}

\footnotetext{Taken from \texttt{ioffe.ru} database - all at 300K.}

\end{frame}

\begin{frame}{Examples cont.}
  \begin{itemize}
    \item Not-so-recently, charge carrier complex binding in coupled quantum
    wells
  \end{itemize}
  \begin{minipage}{0.5\textwidth}
  \centering
  \begin{figure}[H]
    \centering
    \includegraphics[width=0.8\linewidth]{figs/cqw_biexc.pdf}
    \caption{Schematic of a biexciton in a CQW system.}
    \label{fig:biexc}
  \end{figure}
  \end{minipage}%
  \hfill
  \begin{minipage}{0.5\textwidth}
  \centering
  \begin{figure}[H]
    \centering
    \includegraphics[width=0.8\linewidth]{figs/pretty.pdf}
    \caption{QMC Binding energy of this biexciton.}
    \label{fig:be_binding}
  \end{figure}
  \end{minipage}

\end{frame}
% }}*

% *{{ Another significantly important aspect of my experience
\section{Summary}
\begin{frame}{Summary}
  \begin{itemize}
    \item I use highly accurate\footnote{Systematically
    improvable, unbiased - but also very costly...} electronic structure
    methods to (\green{hopefully}) obtain the right answers to \blue{important}
    questions.
    \item I am a member of the \textasciitilde 1 in 9 "moved to Lancaster club"
    - and hence you can ask me general things if you are considering a move!
  \end{itemize}
\end{frame}
% }}*

\end{document}

