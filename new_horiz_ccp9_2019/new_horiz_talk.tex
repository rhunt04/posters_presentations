% *{{ Actual preamble
% *{{ preamble
\documentclass[10pt, pdf, hyperref={draft}, usenames, dvipsnames]{beamer}
\usepackage{libertine,textcomp,amssymb,wasysym,color,ulem,verbatim,calc,tikz}
\usepackage{floatrow,subcaption,appendixnumberbeamer,dcolumn,booktabs,multirow}
% hide controls
\usenavigationsymbolstemplate{}
\usepackage[T1]{fontenc}
\synctex=1
\setbeamertemplate{caption}[numbered]

\newcolumntype{d}[1]{D{.}{.}{#1}}
\newcommand\mc[1]{\multicolumn{1}{c}{#1}}

% }}*

% *{{ item colouring and themes
%\usecolortheme{beaver}
%\setbeamertemplate{footline}[frame number]
\definecolor{LancsRed}{HTML}{B5121B}
\setbeamercolor{item}{fg=LancsRed}
\setbeamercolor{structure}{fg=black}
\setbeamercolor{bibliography entry author}{fg=LancsRed}
\setbeamercolor{bibliography entry title}{fg=black}
\setbeamercolor{bibliography entry note}{fg=black}
%\setbeamercolor{section in toc}{fg=LancsRed}
%\setbeamercolor{title}{fg=LancsRed}
%\setbeamercolor{frametitle}{fg=LancsRed}
%\setbeamercolor{footnote mark}{fg=LancsRed}
%\setbeamercolor{footnote}{fg=LancsRed}

\setbeamercolor{subsection in toc}{fg=gray}

% Page numbering:
%\setbeamertemplate{footline}[frame number]{}
%\setbeamercolor{footline}{fg=LancsRed}
%\setbeamerfont{footline}{size=\small}

\setbeamertemplate{frametitle}{\nointerlineskip
    \begin{beamercolorbox}[wd=\paperwidth,ht=2.75ex,dp=1.375ex]{frametitle}
        \hspace*{2ex}\insertframetitle \hfill
        {\raisebox{1mm}{\textcolor{gray}{\small$\langle$\ \insertframenumber\ |
        \inserttotalframenumber\ $\rangle$}}} \hspace*{1ex}%
    \end{beamercolorbox}}

\newcommand{\gitem}[1]{\setbeamercolor{item}{fg=ForestGreen}\item[$\checkmark$] #1}
\newcommand{\bitem}[1]{\setbeamercolor{item}{fg=LancsRed}\item[$\times$] #1}
\newcommand{\nitem}[1]{\setbeamercolor{item}{fg=NavyBlue}\item[$-$] #1}
% }}*

% *{{ margins
\setbeamersize{text margin left=5mm, text margin right=5mm}
% }}*

% *{{ niceties
\newcommand{\dd}{\mathrm{d}}
\newcommand{\e}{\mathrm{e}}
\newcommand{\ket}[1]{\lvert{#1}\rangle}
\newcommand{\bra}[1]{\langle{#1}\rvert}
\newcommand{\expt}[1]{\langle{#1}\rangle}
\newcommand{\red}[1]{{\bf\color{LancsRed}{#1}}}
\newcommand{\blue}[1]{{\bf\color{NavyBlue}{#1}}}
\newcommand{\green}[1]{{\bf\color{ForestGreen}{#1}}}
\newcommand{\mred}[1]{{\color{LancsRed}{#1}}}
\newcommand{\mblue}[1]{{\color{NavyBlue}{#1}}}
\newcommand{\mgreen}[1]{{\color{ForestGreen}{#1}}}
\newcommand{\kfr}{{\bf k}_{\rm f}}
\newcommand{\kto}{{\bf k}_{\rm t}}
\let\OLDitemize\itemize
\renewcommand\itemize{\OLDitemize\addtolength{\itemsep}{5pt}}
% }}*

% *{{ title, subtitle, inst., date
\title{{\LARGE Quantum Monte Carlo calculations of energy gaps from
first-principles}}
% \subtitle{CMT Seminar \\ X$^{\text{th}}$ Whenever}

% authors and institutions
\author{\large Ryan J. Hunt}

\institute[]{\normalsize
  Department of Physics,\\
  Lancaster University
}

% date
% \date{today}
\date{}

% }}*

% *{{ bibliography setup
\usepackage[backend=biber, style=phys]{biblatex}
\bibliography{refs}
\renewcommand{\footnotesize}{\scriptsize}
\AtEveryCitekey{\clearfield{title}
                \clearfield{pagetotal}
                \clearfield{pages}
                \clearfield{doi}}
\renewcommand*{\bibfont}{\footnotesize}

% }}*

% *{{ graphics on front page
\titlegraphic{\includegraphics[width=4cm]{figs/lan_logo.png}
\hfill\includegraphics[width=4cm]{figs/nownano_logo.png}}

% }}*

% *{{ titlepage + toc

% Section and subsections will appear in the presentation overview
% and table of contents.

\begin{document}

\begin{frame}[plain]
  \titlepage
\end{frame}

%\begin{frame}{Outline}
%  \tableofcontents
%\end{frame}

% contents at ssubsection starts - comment to get rid
% \AtBeginSection[]
% {\begin{frame}<beamer>{Outline}
%  \tableofcontents[currentsection]
% \end{frame}}

% }}*

% }}* End Preamble

\section{Motivation}

\subsection{What's the problem?}

% *{{ Problem
\begin{frame}{What's the problem?}
We'd like to be able to predictively model the (opto)electronic behaviour of
materials. Because this could be {\it useful}.
\vfill
\begin{minipage}[t]{0.55\textwidth}
\vspace{1.2cm}
\begin{itemize}
  \item Specifically, $\Delta_{\text{Ex.}}$, $\Delta_{\text{QP}}$, and $E^{\rm
  X}_{\rm B}$ in semiconductors.\footnotemark
  \item Don't define ``material heaven'', but are a start.
  \item (The blue LED is blue for a reason.)
\end{itemize}
\end{minipage}%
\hfill
\begin{minipage}[t]{0.4\textwidth}
\vspace{0pt}
\begin{figure}[H]
  \centering
  \includegraphics[width=0.75\linewidth]{figs/cv_gaps.pdf}
  %\caption{Name}
\label{fig:name}
\end{figure}
\end{minipage}%
\footnotetext{
{FYI work discussed here is in: {\red{Hunt \textit{et
al.}}, Phys. Rev. B {\bf 98(7)} (2018).}}}
\end{frame}

\begin{frame}{Energy gaps: definitions}

The \blue{quasiparticle gap}, $\Delta_{\text{QP}}$, is defined as the
difference between the CBM and the VBM:
\begin{align}
  &\Delta_{\text{QP}}(\kfr,\kto)={\cal E}_{\text{CBM}}(\kto)-
  {\cal E}_{\text{VBM}}(\kfr)
  \nonumber \\
  &=\left[E_{{\rm N}+1}(\kto)-E_{\rm N}(\kto)\right]-
    \left[E_{\rm N}(\kfr)-E_{{\rm N}-1}(\kfr)\right] \nonumber \\
  &=E_{{\rm N}+1}(\kto)+E_{{\rm N}-1}(\kfr)-E_{\rm N}(\kto)-E_{\rm N}(\kfr),
\end{align}
\vfill
The \blue{excitonic gap}, $\Delta_{\text{Ex.}}$, is defined as the energy
difference between an excited $\rm N$-electron state and the ground
$\rm N$-electron state:
\begin{equation}
  \Delta_{\text{Ex.}}(\kfr,\kto) = E^{+}_{\rm N}(\kfr, \kto) - E_{\rm N},
\end{equation}

Their difference is the \blue{exciton binding}.\footnote{~~The interaction
energy of a quasielectron at $\kto$ and a quasihole at $\kfr$.}
\end{frame}

\begin{frame}{\ }
\vspace{-1cm}
\begin{figure}[H]
  \centering
  \includegraphics[width=0.6\linewidth]{figs/laughlin_slide.png}
  \caption{Introductory slide from Laughlin's Nobel lecture.}
\label{fig:laughlin_slide}
\end{figure}
\end{frame}

% }}*

\subsection{Why Quantum Monte Carlo (QMC)?}

% *{{ Why QMC?
\begin{frame}{Why Quantum Monte Carlo (QMC)?}
What else?

\begin{itemize}
  \item Density functional theory (or HF | hybrids)
  \begin{itemize}
    \item Take differences in Kohn-Sham (Hartree-Fock) SP eigenvalues.
  \end{itemize}
  \item Many-body perturbation theory ($GW$ | $GW$-BSE | MP$n$)
  \begin{itemize}
    \item QP energies from QP equation (feat. self-energy, $\Sigma({\bf
    k},\omega)$).
  \end{itemize}
  \item Quantum chemistry (post HF | CC | CI | FCI)
  \begin{itemize}
    \item Most similar to present: direct calculation of total energies.
  \end{itemize}
\end{itemize}
\vfill
Either too crude, too scattered, or too expensive.
\end{frame}

\begin{frame}{Why QMC - cont.}
QMC methods:
\begin{itemize}
  \gitem Are highly accurate, and systematically improvable.
  \gitem Are non-perturbative, and treat correlation effects exactly.
  \gitem Have $\mathcal{O}({{\rm N}_{e}}^3)$ cost, not much worse in
  ``abnormal'' cases.
\end{itemize}
\vfill
Proof? Lots available, see reviews,~\footfullcite{Foulkes2001,Needs2009}
or below.~\footfullcite{Ceperley1980}

\begin{figure}[H]
  \floatbox[{\capbeside\thisfloatsetup{capbesideposition={left, center},
  capbesidewidth=0.4\textwidth}}]{figure}[\FBwidth]
  {\caption{The basis of much modern (computational) electronic structure
  theory.}\label{fig:cep}}
  {\includegraphics[width=0.4\textwidth]{figs/ceperley.pdf}}
\end{figure}
\end{frame}
% }}*

\section{QMC Methods}

\subsection{VMC and DMC}

% *{{ VMC and DMC

% *{{ VMC
\begin{frame}{QMC Methods\footfullcite{Foulkes2001}}
\begin{block}{Variational Monte Carlo}
  \begin{itemize}
    \item Endow a {\it trial} wavefunction with variational freedom:
    \begin{equation}
    \Psi({\bf R})=\underbrace{\exp{\left[\ \mathcal{J}_{\{\alpha\}}({\bf R})\
    \right]}}_\text{\green{Our additions}}
    \ \times \underbrace{\mathcal{D}({\bf R})}_\text{\red{DFT, HF, ...}},
    \end{equation}
    and pick $\{\alpha\}$.
    \item MC integration used, for example, to evaluate
    \begin{equation}
    \bra{\Psi}\mathcal{\hat H}\ket{\Psi} = \displaystyle\int\dd {\bf R}\
    |\Psi({\bf R})|^2
    \left[\dfrac{\mathcal{\hat H}\Psi({\bf R})}{\Psi({\bf R})}\right]
    \approx \sum_{i}\dfrac{\mathcal{H}({\bf R}_i)\Psi({\bf
    R}_i)}{\Psi({\bf R}_i)},
    \end{equation}
    ($\{{\bf R}_i\}$ distributed as $|\Psi({\bf R})|^2$).
  \end{itemize}
\end{block}

\end{frame}

% \begin{frame}{VMC - cont.}
% \begin{block}{Picking $\{\alpha\}$: a super-simple example}
% For example, a ``guessed'' H atom trial wave function might look like:
%
% \begin{equation}
% \Psi_{H}({\bf r}_e) =
% \exp{\left[\underbrace{(r_e-L)^C\Theta(L-r_e)}_{\text{Smooth
% cutoff}}\sum_{i=0}\alpha_i r^i_e\right]}.
% \end{equation}
%
% We use robust optimization algorithms to vary the $\{\alpha\}$ such that our
% guess looks more like an eigenstate.
%
% \begin{itemize}
%   \item Minimise $E$, $\sigma^2_{E}$, or another measure of spread, \ldots
% \end{itemize}
%
% \end{block}
% \end{frame}

% \begin{frame}{VMC - cont.}
%
% \begin{itemize}
%   \item {\it Jastrow factor} ($\exp{\left[ \mathcal{J} \right]}$) describes
%   $n$-body correlations {\bf explicitly} in real-space, and allows
%   $\Psi$ to satisfy properties of the {\bf exact} many-electron
%   wavefunction.\footnote{~~ Also critical for efficiently describing {\it
%   dispersion} interactions.}
%
%   \item E.g. the Kato~\footfullcite{Kato1957} cusp conditions:
%
% \end{itemize}
%
% \begin{minipage}[t]{0.25\textwidth}
% \vspace{0.75cm}
% \centering
% \hfill\includegraphics[width=\linewidth]{figs/pair.pdf}
%
% \end{minipage}%
% \hfill
% \begin{minipage}[t]{0.65\textwidth}
%
% \begin{itemize}
%   \item As (1,2) coalesce, if $V(r_{12})$ diverges,
%   $T$ must also ($\mathcal{\hat H}={\hat T}+{\hat V}$).
%   \item Kinetic energy $\implies$ derivatives of $\Psi$ $\implies$ derivatives
%   of $\mathcal{J}$.
%   \item Fixes some of the $\{\alpha\}$.
% \end{itemize}
%
% \end{minipage}%
%
% \end{frame}

% }}*

% *{{ Diffusion MC

\begin{frame}{QMC Methods II}
  \begin{block}{Diffusion Monte Carlo}
    \begin{itemize}
      \item DMC is a stochastic projector-based method for solving
      \begin{equation}
      \mathcal{\hat H}\ \Psi({\bf R}, \tau)
      = (E_T-\partial_{\tau})\Psi({\bf R}, \tau),
      \end{equation}
      or, if you like
      \begin{equation}
      \Psi({\bf R}, \tau+\Delta \tau)
      = \displaystyle\int G({\bf R} \leftarrow {\bf R'},\Delta \tau)
      \Psi({\bf R'}, \tau) \dd{\bf R'}.
      \end{equation}
      \item \green{Separable} ($\partial_\tau\mathcal{\hat H}=0$)
      \footnote{~~ $\{\Phi_i\} \rightarrow$ complete basis of eigenstates of the
      interacting problem.}
      \begin{equation}
      \Psi(0) = \sum_n c_n \Phi_n \implies
      \Psi(\tau) = \sum_n c_n \Phi_n\exp{\left[-(\mathcal{E}_n -
      E_T)\tau\right]}
      \end{equation}
    \end{itemize}
  \end{block}
\end{frame}

\begin{frame}{DMC - cont.}
  Effectively we take:
  \begin{equation}
  \lim_{\tau \rightarrow \infty} \Psi(\tau) \sim \Phi_{0},
  \end{equation}
  by having the DMC Green's function take configurations ${\bf
  R}'\rightarrow{\bf R}$, with caveats:
  \begin{itemize}
    \item \blue{Time steps}: know $G({\bf R} \leftarrow {\bf
    R'},\Delta\tau)$ in limit of small $\Delta
    \tau$.

    \item \blue{Population control}: number of walkers in DMC
    fluctuates. Control mechanism introduces a bias.

    \item \red{Finite-size (FS) effects}: extrapolation to TD limit
    a necessity.

    \item \red{Fixed-node approximation}: (non-local) antisymmetry enforced by
    (local) boundary condition ($\Psi=0$ surface is fixed).

  \end{itemize}
  \vfill
  \centering
  \green{Gaps}: expect some of these to matter less!
\end{frame}
% }}*

% }}*

\subsection{Excited-state QMC}

% *{{ Excited state DMC

\begin{frame}{Excited States}
Briefly:
\begin{itemize}
  \item  QP gap $\rightarrow E_{{\rm N}, {\rm N}\pm1}$ (VP on each
  \textit{ground} state).
  \item Ex. gap $\rightarrow$ \textit{may} have VP on $E^{+}$. \textit{May} only
  have at VMC level. FN-DMC VP obtained in special
  circumstances.~\footfullcite{Foulkes1999}
  \item \textcolor{gray}{(FN
  constraint means effective VP)}
\end{itemize}
\end{frame}

% \begin{frame}{Excited State DMC}{Wait a second... what I said applies to ground
% states.}
% \vspace{0cm}
% In special circumstances, what I have just said
% also applies to the ground state {\it of a given
% symmetry}.\footnotemark~Key idea concerns
% excited-states and generalised variational principles
% \only<2->{\footnotetext{~~Actually, the variational principle on ground states
% in fixed-node DMC is an excited state one, fermion GS transforms as 1D irrep.
% of permutation group.}}
%
%   \begin{itemize}
%   \pause
%     \item \red{Jargon}: If trial wfn. transforms as a 1D irrep. of the symmetry group
%     of the Hamiltonian, have a variational principle on states of that
%     symmetry (could be excited state).~\only<2->{\footfullcite{Foulkes1999}}
%     \pause
%     \item \green{Simple practical upshot}: the many-body Bloch conditions for periodic
%     calculations mean that states with definite crystal momentum belong to a 1D
%     irrep. (of the translation group).
%   \end{itemize}
% \vfill
% There are {\it many} other means of obtaining variational
% bounds.\only<3->{~\footfullcite{Zhao2016,MacDonald1933,Hipes2011}}
% \end{frame}

% }}* frame end

\section{Case studies}

% \subsection{Atoms and molecules}

% % *{{ Atoms and molecules
%
% \begin{frame}{Intrinsic accuracy: Ne atom}
%
% The neon atom ground state is (famously) closed shell. None of the cations
% are, and there's a potential for build-up of nodal\footnote{~~FN-error
% $\equiv$ error arising from (local) B.C. on $\Psi$ ($\Psi=0$ ``surface''
% fixed to please Pauli).}
% error in IP.~\footfullcite{Rasch2014,Kulahlioglu2014}
% \begin{equation}
% \text{IP}(n) = E(n) - E(n-1),
% \end{equation}
% Q: Does this happen?
% \begin{figure}[H]
%   \floatbox[{\capbeside\thisfloatsetup{capbesideposition={left, center},
%   capbesidewidth=0.5\textwidth}}]{figure}[\FBwidth]
%   {\caption{DMC and FCI correlation energies of atoms versus atomic
%   number (from Rasch {\it et al.}). NB 30mHa $\sim$ 0.8
%   eV.}\label{fig:nodal_error}}
%   {\includegraphics[width=0.5\textwidth]{figs/nodal_error.png}}
% \end{figure}
% \end{frame}
%
% \begin{frame}{Ne atom - cont.}
%   ``Exact'' $\implies$ expt. minus best est. relativistic correction.
%   \footfullcite{Chakravorty1993}
%   \begin{table}[H]
%   \centering
%   \begin{tabular}{c *{5}{d{3.5}}}
%   \multirow{2}{*}{$n$} & \multicolumn{5}{c}{IP($n$) (eV)} \\
%   & \mc{Exact}&\mc{SJ-VMC}&\mc{SJB-VMC}&\mc{SJ-DMC}&\mc{SJB-DMC}\\ \hline
%   1& 21.61333&22.08(2)&21.96(2) &21.72(1) &21.72(1) \\
%   2& 40.99110&41.48(2)&41.39(2) &41.10(1) &41.06(1) \\
%   3& 63.39913&63.44(2)&63.23(1) &63.35(2) &63.39(1) \\
%   4& 97.29312&97.91(2)&97.78(1) &97.75(2) &97.72(1) \\
%   5&126.28846&126.85(2)&126.72(2)&126.85(1)&126.79(1)\\
%   6&157.80001&158.43(2)&158.30(1)&158.25(2)&158.34(1)\\
%   7&207.04137&204.48(2)&204.56(1)&205.04(2)&205.26(1)\\
%   8&238.78949&238.10(1)&238.49(1)&238.70(2)&238.79(1)\\
%   \hline
%   \mc{\text{MAE}} & \mc{0\%} & 0.83\%    & 0.67\%    & 0.38\%    & 0.34\%
%   \end{tabular}
%   \end{table}
% \vspace{-0.2cm}
% So? IPs related to QP energies. More later!
% \end{frame}
%
% \begin{frame}{Molecules - dimers}
% \begin{itemize}
%   \item O$_2$: singlet/triplet, (near)degeneracy, multideterminants.
% \end{itemize}
% \begin{figure}[H]
%   \centering
%   \includegraphics[width=0.6\linewidth]{figs/o2_dimer/o2_dimer.pdf}
% \end{figure}
% \vfill
% \begin{itemize}
%   \item H$_2$: vibrations, quantum protons.
% \end{itemize}
% \begin{figure}[H]
%   \centering
%   \includegraphics[width=0.5\linewidth]{figs/h2_dimer/h2_dimer.pdf}
% \end{figure}
% \end{frame}
%
% \begin{frame}{Dimers (cont.)\footfullcite{Hunt2018}}
% \begin{figure}[H]
%   \floatbox[{\capbeside\thisfloatsetup{capbesideposition={left, center},
%   capbesidewidth=0.3\textwidth}}]{figure}[\FBwidth]
%   {\caption{O$_2$ and H$_2$ dimer results.}\label{fig:dimers}}
%   {\includegraphics[width=0.6\textwidth]{figs/dimers.png}}
% \end{figure}
%
% \begin{itemize}
%   \item For O$_2$, also determined singlet-triplet splitting of 1.62(2)
%   [0.20(3)] eV, c.f.
%   0.9773 eV.
%   \item $\mathcal{D}$/$\Sigma_i\mathcal{D}_i$ and ability of QMC
%   to treat on equal footing is key!
% \end{itemize}
% \end{frame}
%
% \begin{frame}{Molecules - nondimers}
% In terms of computational cost, QMC for small molecules (${\rm N}_e \sim 100$) is roughly on par with
% $GW$(-BSE). Recently, a slew of such studies have focussed on studying small
% molecules.
% \begin{figure}[H]
%   \centering
%   \includegraphics[width=0.7\linewidth]{figs/combo.png}
%   \caption{C$_{14}$H$_{10}$ $\rightarrow$ C$_6$N$_4$ $\rightarrow$ C$_7$H$_5$NS
%   $\rightarrow$ BF$_3$.}
% \label{fig:combo}
% \end{figure}
% \begin{itemize}
%   \item Results vary significantly, and are $G_0$, $W_0$, $v_{\text{XC}}$
%   \ldots dependent.
%   \item Is QMC a worthy competitor in the realm of small molecules?
% \end{itemize}
% Example use: $T(E)$ in molecular electronics, for example. Determines $I$-$V$
% characteristics of a molecular junction.
% \end{frame}
%
% \begin{frame}{Nondimers (cont.)}
% \begin{minipage}[t]{0.45\textwidth}
% Anthracene (C$_{14}$H$_{10}$):
% \begin{itemize}
%   \item $\Delta_{\text{Ex.}}^{S=0,1}=3.07(3),2.36(3)$ eV. (PAH \&
%   \green{Hund}!).
%   \item Expt. says 3.38-3.433,
%   1.84-1.85$^{\star}$ eV.
%   \item No comparable $GW$.\footnotemark \\
%   \hspace{2cm}$\downarrow$
% \end{itemize}
% \end{minipage}%
% \hfill
% \begin{minipage}[t]{0.45\textwidth}
% Benzothiazole (C$_7$H$_5$NS):
% \begin{itemize}
%   \item IP = $\underbrace{8.92(2)}_{\rm V}\rightarrow \underbrace{8.80(2)}_{\rm A}$ eV
%   \item Expt. says 8.72(5) eV [$\rm A$].
%   \item $GW$ says 8.2-8.5 eV [$\rm V$].
% \end{itemize}
% \end{minipage}%
% \footnotetext{~~Hummer \textit{et al.} studied mol.
% cry. anthracene: {\bf PRL 92(14)}, (2004). Unreferenced numbers found (w/ refs)
% in our article, excluded for space.}
% \vfill
% \begin{itemize}
%   \gitem QMC as tool for probing \blue{molecular excitons}. Relevant in
%   materials screening for biological imaging, for example. (FRET).
%   \bitem Relevant quantities are (really) resonance timescales.
% \end{itemize}
% \end{frame}
% % }}*

\subsection{Bulk solids}

% *{{ Bulk solids

\begin{frame}{Bulk solids}

We've studied Si, $\alpha$-SiO$_2$, and cubic BN in the current work. Previous
QMC studies had claimed success in evaluation of ``QMC band
structures'',~\footfullcite{Kent1998,Williamson1998}
minus discussions of:

\begin{minipage}[t]{0.35\textwidth}
\vspace{0.3cm}
\begin{itemize}
  \item Finite-size errors.
  \item Fixed-nodal errors.
  \item $\Delta_{\text{QP}}$ vs. $\Delta_{\text{Ex.}}$.
\end{itemize}
\vspace{0.3cm}
Will concentrate on Si here, exploring the above.
\end{minipage}%
\hfill
\begin{minipage}[t]{0.55\textwidth}
\begin{figure}[H]
  \centering
  \includegraphics[width=0.9\linewidth]{figs/si.png}
  %\caption{Si}
\label{fig:silicon}
\end{figure}
\end{minipage}%
\end{frame}

\begin{frame}{Bulk solids: FS errors}
\begin{itemize}
  \item Able only to simluate a finite \textit{chunk} of material
  (supercell), under PBCs. Excitations ``1/${\rm N}$'' effects. Need statistical
  accuracy + careful FS treatment.
\end{itemize}
\begin{figure}[H]
  \floatbox[{\capbeside\thisfloatsetup{capbesideposition={right, center},
  capbesidewidth=0.5\textwidth}}]{figure}[\FBwidth]
  {\caption{Uncorrected SJ-DMC gaps of Si. FS effect characteristic and
  quantifiable, largely from image-interactions.}\label{fig:si_gaps}}
  {\includegraphics[width=0.45\textwidth]{figs/si_gaps.pdf}}
\end{figure}
\begin{itemize}
  \item Then why do $\Delta_{\text{QP}}$ \& $\Delta_{\text{Ex.}}$ behave same?
\end{itemize}
\end{frame}

\begin{frame}{Bulk solids: Fixed-nodal error}
\begin{itemize}
  \item Probe with Backflow transformation:
  \begin{equation}
    {\bf r}_i \rightarrow {\bf x}_i = {\bf r}_i + {\boldsymbol \xi}_i({\bf R})
  \end{equation}
  which can change nodal surface.~\footfullcite{lopezrios2006}
  \item Tested $\Delta_{\text{QP/Ex}}(\Gamma_{\rm v} \rightarrow \Gamma_{\rm
  c})$ and $\Delta_{\text{QP}}(\Gamma_{\rm v} \overset{\sim}{\rightarrow}
  \text{CBM})$, in $2\times2\times2$ supercell.
  \item We find that backflow leads to a reduction in gaps, of at least 0.2
  eV, but upto 0.3--0.4 eV when one re-optimises ${\boldsymbol
  \xi}_i$.\footnote{~~C.f. controllable uncertainty: $\mathcal{O}(0.1\
  \text{eV})$ for each of pseudopotentials, statistics, NLO FS effects (?).}
\end{itemize}
\end{frame}

% \begin{frame}{Bulk solids: Fixed-nodal error (cont.)}
% \begin{block}{We find:}
%  \begin{itemize}
%    \item Inclusion of backflow correlations:
%    \begin{itemize}
%      \item Lowers $\Delta_{\text{QP}}(\Gamma_{\rm v} \rightarrow \Gamma_{\rm
%      c})$ by 0.13 eV.
%      \item Lowers $\Delta_{\text{Ex.}}(\Gamma_{\rm v} \rightarrow \Gamma_{\rm
%      c})$ by 0.12 eV.
%    \end{itemize}
%    \item Re-optimisation of excited-state backflow functions (${\boldsymbol
%    \xi}_i({\bf R})$) further lowers:
%    \begin{itemize}
%      \item $\Delta_{\text{QP}}(\Gamma_{\rm v} \rightarrow \Gamma_{\rm
%      c})$ by 0.18 eV.
%      \item $\Delta_{\text{QP}}(\Gamma_{\rm v} \overset{\sim}{\rightarrow}
%   \text{CBM})$ by 0.20 eV.
%    \end{itemize}
%  \end{itemize}
% \end{block}
%  \vfill
%  \red{Point:} FN errors \red{not} insignificant ($\gtrsim 0.2$ eV), but can be
%  remedied\footnote{~~Always safe for QP gaps. Sometimes safe for Ex. gaps.} w/ backflow.
% \end{frame}

%%  \begin{frame}{Bulk solids: $\Delta_{\text{QP}}$ and $\Delta_{\text{Ex.}}$}
%%  In bulk Si $E^{\rm X}_B \sim 15$ meV.
%%  \begin{itemize}
%%    \item Expect $\Delta_{\text{QP}} \sim \Delta_{\text{Ex.}}$ (in TD limit).
%%    \item (potentially) qualitatively different FS effects, however.
%%    \item \ldots but already seen that this doesn't happen in Si.
%%  \end{itemize}
%%  \vfill
%%  \begin{block}{Explanation:}
%%    Confine an exciton too tightly, $E_{\text{K}}$ dominates, no binding, behave
%%    effectively uncorrelated. Balance of supercell length scale and exciton
%%    length scale important.
%%  \vfill
%%    $\implies$ $a^{\star}_{\rm B}$ in Si is \blue{huge}. Cell never big enough to
%%    accomodate ${\rm X}$.
%%  \end{block}
%%  \end{frame}

% \begin{frame}{Bulk solids: the old calculations (briefly)}
% No $q_{\bf k} \implies \Delta \uparrow$ \hfill
% Bad wfn. $\implies \Delta \uparrow$ \hfill
% FS effects $\implies \Delta \downarrow$
% \begin{figure}[H]
%   \centering
%   \includegraphics[width=0.8\linewidth]{figs/perspective.jpg}
%   \caption{Perspective.}
% \label{fig:perspective}
% \end{figure}
% {\tiny \ldots and if you're trying to justify the need for many-body theory,
% you'd best not connect $\Delta_{\text{QP}}$ and
% $\Delta_{\text{Ex.}}$ by means of a M.-W. model!}
% \end{frame}
% }}*

\subsection{Phosphorene}

% *{{ Phosphorene

\begin{frame}{Phosphorene}
A direct gap 2D semiconductor, with large exciton binding
energy.\footfullcite{yang2015}

\begin{figure}[H]
  \centering
  \includegraphics[width=0.6\linewidth]{figs/phos_spec_paper.jpg}
  \caption{PL measurements of Phosphorene $n$-layers.}
\label{fig:phos_spec_paper}
\end{figure}

\begin{itemize}
  \item Do \red{not} expect $\Delta_{\text{QP}} \sim \Delta_{\text{Ex.}}$
  \item FS effects in $\Delta_{\text{QP/Ex.}}$ much more important.
\end{itemize}
\end{frame}

% \begin{frame}{Physics of FS effects in 2D}
% \begin{minipage}[t]{0.5\textwidth}
% \vfill
% \begin{center}
%   \begin{tikzpicture}
%     \node<1> (img1) {\includegraphics[width=.9\linewidth]{figs/per_exc_fig/free.pdf}};
%     \node<2> (img2) {\includegraphics[width=.9\linewidth]{figs/per_exc_fig/free_box.pdf}};
%     \node<3> (img3) {\includegraphics[width=.9\linewidth]{figs/per_exc_fig/per_emp.pdf}};
%     \node<4-> (img4) {\includegraphics[width=.9\linewidth]{figs/per_exc_fig/per_full.pdf}};
%   \end{tikzpicture}
% \vfill
% \end{center}
% \end{minipage}%
% \hfill
% \begin{minipage}[t]{0.5\textwidth}
% \begin{itemize}
% \item<1-> Want to model a free excitonic complex
% \item<2-> Supercell calculation (characteristic size $L$)
% \item<3-> Under periodic BCs
% \item<4-> Hence incurr an unphysical image-interaction
% \end{itemize}
% \end{minipage}%
% \vfill
% \begin{itemize}
% \item<5-> Need to remove $E_{\text{int}}$.
% \item<5-> Scaling arguments, or model calculations?
% \end{itemize}
% \end{frame}

\begin{frame}{Physics of FS effects in 2D}
\begin{minipage}[t]{0.5\textwidth}
\vfill
\begin{center}
  \begin{figure}[H]
    \centering
    \includegraphics[width=0.9\linewidth]{figs/per_exc_fig/per_full.pdf}
  \end{figure}
\vfill
\end{center}
\end{minipage}%
\hfill
\begin{minipage}[t]{0.5\textwidth}
\vspace{1.4cm}
\begin{itemize}
\item We want to model a \textit{free} excitonic complex
\item Perform supercell calculation (SC characteristic size $L$), subject to periodic BCs
\item Hence incurr an unphysical image-interaction
\end{itemize}
\end{minipage}%
\vfill
\begin{itemize}
\item Need to remove $E_{\text{int}}$. \red{How does it scale?}
\end{itemize}
\end{frame}

\begin{frame}{Physics of FS effects in 2D (cont.)}
\begin{figure}[H]
\centering
\begin{subfigure}{.5\textwidth}
  \centering
  \includegraphics[width=\linewidth]{figs/coul_origin.png}
  % \caption{$r_*=0$ (Coulomb).}
  \label{fig:sub1}
\end{subfigure}%
\begin{subfigure}{.5\textwidth}
  \centering
  \includegraphics[width=\linewidth]{figs/keld_rs_50_origin.png}
  % \caption{$r_*=50$ a.u. (Keldysh).}
  \label{fig:sub2}
\end{subfigure}
\vspace{-1cm}
\caption{Unscreened (left) and screened (right) field lines from point charges at $\rho=z=0$.}
\label{fig:field_lines}
\end{figure}
\vspace{-.3cm}
\begin{itemize}
  \item With 2D screening (Keldysh interaction), charge-quadrupole
  interaction\footnote{~~Note no dipole in inversion symmetric system!} leads to
  expected scaling which is $\mathcal{O}(L^{-2})$.% \footnote{~~(from taking sum
  % to be an integral - really only valid at large distances).}
\end{itemize}
\end{frame}

% \begin{frame}{Physics of FS effects in 2D (cont.)}
%
% \begin{itemize}
%   \item With 2D screening (Keldysh interaction), charge-quadrupole
%   interaction\footnote{~~Note no dipole in inversion symmetric system!} leads to
%   expected scaling which is $\mathcal{O}(L^{-2})$.\footnote{~~(from taking sum
%   to be an integral - really only valid at large distances).}
%
%   \item Model calculation: study periodic excitons directly (lattice sums over
%   screened interaction). \textbf{In progress.}\footnote{~~At last check, the
%   data I had were best fit by a model $\Delta(a)=\Delta(\infty)+C a^{-P}$, with
%   exponent $p$ of 2.0073\ldots}
% \end{itemize}
% \end{frame}

\begin{frame}{Physics of FS effects in 2D (cont.)}
\begin{block}{$\Delta_{\text{QP}}$}
  \begin{itemize}
    \item Similar image effects, easier to manage.
    \item Subtract single-particle $v_{\rm M}$ ($\mathcal{O}(L^{-1})$).
    \item From regularized lattice sum over screened interaction ($\sim$ Ewald
    sum).
  \end{itemize}
  \begin{equation}
    \sum_{\bf R} W({\bf r}-{\bf R}) \rightarrow \overbrace{\sum_{\bf R} V({\bf
    r}-{\bf R})}^{\green{Ewald}}
    + \underbrace{\sum_{\bf R} \delta V({\bf r}-{\bf R})}_{\blue{{\it safe}}}.
  \end{equation}
  \begin{itemize}
    \item ``\blue{{\it Safety}}'': $\displaystyle\lim_{r \rightarrow \infty} \delta
    V(r)=0$.% (screening has a range).
    % is realised if screening is irrelevant
    % at long-range in real-space.
  \end{itemize}
\end{block}
\end{frame}

\begin{frame}{Physics of FS effects in 2D (cont.)}
  \begin{itemize}
    \item FSE appear to scale as argued.
    \item Big gaps (\red{$\epsilon$!}),\footnote{~~Also, this is not due to FN
    error! Backflow has $\mathcal{O}(0.05\ \text{eV})$ effect here.} but good agreement
    w/ Gaufr\`{e}s \textit{et
    al.}\footnote{~~This result is unpublished, so far, but was presented at GW
    2018 by A. Loiseau. $\Delta_{\text{Ex}} = 1.95$ eV.}
    \item Phonon renormalisation $\sim 0.17$ eV @ 300K.\footnote{~~Via
    Tomeu Monserrat, also as yet unpublished.}
  \end{itemize}
\begin{figure}[H]
  \floatbox[{\capbeside\thisfloatsetup{capbesideposition={left, center},
  capbesidewidth=0.4\textwidth}}]{figure}[\FBwidth]
  {\caption{QMC energy gaps in phosphorene vs. system size.}\label{fig:phos}}
  {\includegraphics[width=0.5\textwidth]{figs/phos.pdf}}
\end{figure}
\end{frame}

\begin{frame}{Physics of FS effects (cont.)}
Another approach is to consider passivated (finite) clusters. Here FS effect
is kinetic in origin (confinement).\footfullcite{Frank2018}
\begin{itemize}
  \item FS converge faster (QP gap $\mathcal{O}(L^{-2})$ by default), {\bf but}\ldots
  \item State under study may not be relevant %not necessarily adiabatically connected to relevant
  \footfullcite{Drummond2005}\ldots
\end{itemize}
\begin{figure}[H]
  \centering
  \includegraphics[width=0.6\linewidth]{figs/diamondoid.pdf}
  \caption{Band charge densities in C$_{29}$H$_{36}$.}
\label{fig:diamondoid}
\end{figure}
\end{frame}

\begin{frame}{Another QMC study of Phosphorene}
Frank \textit{et al.} have also studied phosphorene. We're
dissatisfied with their approach. Why?
\begin{itemize}
  \item Used cluster calculations to argue scaling in bulk calculations.
  \item Calculated the \textit{wrong gap}:
\begin{figure}[H]
  \floatbox[{\capbeside\thisfloatsetup{capbesideposition={left, center},
  capbesidewidth=0.5\textwidth}}]{figure}[\FBwidth]
  {\caption{Excerpt from preprint.}\label{fig:frank}}
  {\includegraphics[width=0.5\textwidth]{figs/frank.png}}
\end{figure}
  \item Our excitonic gap (2.2(2) eV) agrees with their ``quasiparticle'' gap
  (2.4 eV) \smiley{}. \footnote{~~Guessed wrong scaling exponent (1/N) for QP gap, but
  this isn't a QP gap! Just so happen to have calculated and taken TD limit for
  an excitonic gap.  Assuming they've done the calculations correctly, a good
  test of our result!}
\end{itemize}
\end{frame}

%%  \begin{frame}{Nodal errors in Phosphorene}
%%  We've tested the inclusion of backflow functions in a small Phosphorene test
%%  system.
%%  \begin{itemize}
%%    \item $\Delta_{\text{QP}} $ lowered by $ 0.03(9)$ eV.
%%    \item $\Delta_{\text{Ex.}} $ lowered by $ 0.04(5)$ eV.
%%  \end{itemize}
%%  \begin{figure}[H]
%%    \floatbox[{\capbeside\thisfloatsetup{capbesideposition={left, center},
%%    capbesidewidth=0.5\textwidth}}]{figure}[\FBwidth]
%%    {\caption*{\red{Q} Why is backflow seemingly irrelevant here, when it
%%    was critical in Si? When, generally, do nodal errors tend to matter?}\label{fig:puzzled}}
%%    {\includegraphics[width=0.3\textwidth]{figs/thinking.png}}
%%  \end{figure}
%%  Backflow QMC has $\mathcal{O}({\rm N}^4_e)$ cost scaling. Further
%%  ``computational exploration'' unlikely to provide answer.
%%  \end{frame}
% }}*

\section{Conclusion}

% *{{ Conclusion
\begin{frame}{Conclusion}
\begin{itemize}
  \item QMC methods offer a direct, real-space approach to the many-body problem.
  \item They allow for accurate determination of energy gaps from
  first-principles in one, two and three-dimensional systems.
  \item They can be systematically extended, and treat various important pieces
  of physics \green{exactly}.% {\bf But}\ldots
\end{itemize}
\end{frame}

% \begin{frame}{Conclusion (cont.) - QMC wishlist}
%   QMC might benefit from improvements in:
%   \begin{itemize}
%     \item Trial wave function technology: size-consistent determinant
%     selection? (further) compact parametrisation of correlations?
%     \item Algorithms: cubic scaling isn't bad, can we get prefactor
%     down? Can we cut more corners with energy differences?
%     \item FS treatment: extensive literature on ground states, can we do better
%     for excitations? Are $\rho$-based electrostatics methods better?
%   \end{itemize}
% \end{frame}

% }}*

% *{{ Thanks!
\begin{frame}{Acknowledgements}

\begin{minipage}[t]{0.45\textwidth}
{\bf Lancaster:}
\begin{itemize}
  \item Neil Drummond
  \item Marcin Szyniszewski
\end{itemize}
\end{minipage}%
\hfill
\begin{minipage}[t]{0.45\textwidth}
{\bf Japan (JAIST):}
\begin{itemize}
  \item Ryo Maezono
  \item Genki Prayogo
\end{itemize}
\end{minipage}%
\vfill
\begin{center}
{\Large Thanks for your attention!}
\end{center}
\end{frame}
% }}*

% *{{ References
%\begin{frame}[t,allowframebreaks]{References}
%\printbibliography[]
%\end{frame}
% }}*

%%  \appendix
%%
%%  % *{{ Backups
%%  \begin{frame}[plain]{Backup I : The Keldysh interaction}
%%  Consider a planar sheet of (polarisable, 2D susc. $\kappa$) charge, at $z=0$, within a medium of
%%  dielectric constant $\epsilon$
%%  \begin{equation}
%%    \rho({\bf r})=\rho_{2}(x,y)\delta(z),
%%  \end{equation}
%%  Electric displacement field associated with sheet is
%%  \begin{align}
%%    {\bf D}(x,y,z) &= -\epsilon\nabla\phi(x,y,z)+{\bf P}_{\perp}(x,y)\delta(z),
%%    \nonumber \\
%%    &= -\epsilon\nabla\phi(x,y,z)+\kappa(\nabla\phi(x,y,0))\delta(z),
%%  \end{align}
%%  and Gauss' Law says $\nabla \cdot {\bf D}(x,y,z)=\rho(x,y,z)$.
%%  \end{frame}
%%  \begin{frame}[plain]{Backup I : The Keldysh interaction (cont.)}
%%  After FT and with some algebraic re-arrangement, leads to
%%  \begin{equation}
%%    \phi({\bf q},z=0)=\dfrac{\rho_{2}({\bf q})}{q(2\epsilon+\kappa q)}.
%%  \end{equation}
%%  with ${\bf q}$ the 2D momentum. Looks like
%%  \begin{equation}
%%  V(r) = \text{expression},
%%  \end{equation}
%%  in real-space. The ``Keldysh'' interaction,~\footfullcite{Keldysh1979} but which was known to
%%  Rytova~\footfullcite{Rytova1965coulomb,Rytova1967screened} some time
%%  earlier\ldots
%%  \end{frame}
%%
%%  \begin{frame}[plain]{Backup II: Variational Bounds}
%%  Excitonic promotion calculations: electron number fixed. Where states of
%%  definite crystal momentum can be chosen to populate the Slater determinant,
%%  trial state transforms (trivially) as 1D irrep. of group of translations (in a
%%  way that is basically as complicated as for 1D Bloch waves).
%%
%%  \vfill
%%
%%  Quasiparticle addition/removal: electron number varies. Key is that these are
%%  \textbf{not} bona fide excited states. QP energies formed from \textit{ground
%%  state} total energies of ${\rm N},{\rm N}\pm1$-electron systems.
%%  \end{frame}

%\begin{frame}[plain]{Spin-orbit effects in QMC}
%\end{frame}

%\begin{frame}[plain]{???}
%\end{frame}

% }}*

\end{document}
% }}*
